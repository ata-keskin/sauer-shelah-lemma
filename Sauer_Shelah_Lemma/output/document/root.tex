\documentclass[11pt,a4paper]{article}
\usepackage[T1]{fontenc}
\usepackage{isabelle,isabellesym}

% this should be the last package used
\usepackage{pdfsetup}

% urls in roman style, theory text in math-similar italics
\urlstyle{rm}
\isabellestyle{it}


\begin{document}

\title{Sauer-Shelah Lemma}
\author{Ata Keskin}
\maketitle

\begin{abstract}
	The Sauer-Shelah Lemma is a fundemental result in extremal set theory and combinatorics, that guarentees the existence of a set $T$ of size $k$
	which is shattered by a family of sets $\mathcal{F}$, if the cardinality of the family is greater than some bound dependent on $k$. A set $T$ is
	said to be shattered by a family $\mathcal{F}$ if every subset of $T$ can be obtained as an intersection of $T$ with some set $S \in \mathcal{F}$.
	The Sauer-Shelah Lemma has found use in diverse fields such as computational geometry, approximation algorithms and machine learning. In this entry
	we formalize the notion of shattering and prove the generalized and the standard version of the Sauer-Shelah Lemma. 
\end{abstract}

\tableofcontents

% include generated text of all theories
%
\begin{isabellebody}%
\setisabellecontext{Shattering}%
%
\isadelimdocument
%
\endisadelimdocument
%
\isatagdocument
%
\isamarkupsection{Definitions and lemmas about shattering%
}
\isamarkuptrue%
%
\endisatagdocument
{\isafolddocument}%
%
\isadelimdocument
%
\endisadelimdocument
%
\isadelimtheory
%
\endisadelimtheory
%
\isatagtheory
\isacommand{theory}\isamarkupfalse%
\ Shattering\isanewline
\ \ \isakeyword{imports}\ Main\isanewline
\isakeyword{begin}%
\endisatagtheory
{\isafoldtheory}%
%
\isadelimtheory
%
\endisadelimtheory
%
\isadelimdocument
%
\endisadelimdocument
%
\isatagdocument
%
\isamarkupsubsection{Intersection of a family of sets with a set%
}
\isamarkuptrue%
%
\endisatagdocument
{\isafolddocument}%
%
\isadelimdocument
%
\endisadelimdocument
\isacommand{abbreviation}\isamarkupfalse%
\ IntF\ {\isacharcolon}{\kern0pt}{\isacharcolon}{\kern0pt}\ {\isachardoublequoteopen}{\isacharprime}{\kern0pt}a\ set\ set\ {\isasymRightarrow}\ {\isacharprime}{\kern0pt}a\ set\ {\isasymRightarrow}\ {\isacharprime}{\kern0pt}a\ set\ set{\isachardoublequoteclose}\ {\isacharparenleft}{\kern0pt}\isakeyword{infixl}\ {\isachardoublequoteopen}{\isasyminter}{\isacharasterisk}{\kern0pt}{\isachardoublequoteclose}\ {\isadigit{6}}{\isadigit{0}}{\isacharparenright}{\kern0pt}\isanewline
\ \ \isakeyword{where}\ {\isachardoublequoteopen}F\ {\isasyminter}{\isacharasterisk}{\kern0pt}\ S\ {\isasymequiv}\ {\isacharparenleft}{\kern0pt}{\isacharparenleft}{\kern0pt}{\isasyminter}{\isacharparenright}{\kern0pt}\ S{\isacharparenright}{\kern0pt}\ {\isacharbackquote}{\kern0pt}\ F{\isachardoublequoteclose}\isanewline
\isanewline
\isacommand{lemma}\isamarkupfalse%
\ idem{\isacharunderscore}{\kern0pt}IntF{\isacharcolon}{\kern0pt}\isanewline
\ \ \isakeyword{assumes}\ {\isachardoublequoteopen}{\isasymUnion}A\ {\isasymsubseteq}\ Y{\isachardoublequoteclose}\isanewline
\ \ \isakeyword{shows}\ {\isachardoublequoteopen}A\ {\isasyminter}{\isacharasterisk}{\kern0pt}\ Y\ {\isacharequal}{\kern0pt}\ A{\isachardoublequoteclose}\isanewline
%
\isadelimproof
%
\endisadelimproof
%
\isatagproof
\isacommand{proof}\isamarkupfalse%
\ {\isacharminus}{\kern0pt}\isanewline
\ \ \isacommand{from}\isamarkupfalse%
\ assms\ \isacommand{have}\isamarkupfalse%
\ {\isachardoublequoteopen}A\ {\isasymsubseteq}\ A\ {\isasyminter}{\isacharasterisk}{\kern0pt}\ Y{\isachardoublequoteclose}\ \isacommand{by}\isamarkupfalse%
\ blast\isanewline
\ \ \isacommand{thus}\isamarkupfalse%
\ {\isacharquery}{\kern0pt}thesis\ \isacommand{by}\isamarkupfalse%
\ fastforce\isanewline
\isacommand{qed}\isamarkupfalse%
%
\endisatagproof
{\isafoldproof}%
%
\isadelimproof
\isanewline
%
\endisadelimproof
\isanewline
\isacommand{lemma}\isamarkupfalse%
\ subset{\isacharunderscore}{\kern0pt}IntF{\isacharcolon}{\kern0pt}\ \isanewline
\ \ \isakeyword{assumes}\ {\isachardoublequoteopen}A\ {\isasymsubseteq}\ B{\isachardoublequoteclose}\isanewline
\ \ \isakeyword{shows}\ {\isachardoublequoteopen}A\ {\isasyminter}{\isacharasterisk}{\kern0pt}\ X\ {\isasymsubseteq}\ B\ {\isasyminter}{\isacharasterisk}{\kern0pt}\ X{\isachardoublequoteclose}\isanewline
%
\isadelimproof
\ \ %
\endisadelimproof
%
\isatagproof
\isacommand{using}\isamarkupfalse%
\ assms\ \isacommand{by}\isamarkupfalse%
\ {\isacharparenleft}{\kern0pt}rule\ image{\isacharunderscore}{\kern0pt}mono{\isacharparenright}{\kern0pt}%
\endisatagproof
{\isafoldproof}%
%
\isadelimproof
\isanewline
%
\endisadelimproof
\isanewline
\isacommand{lemma}\isamarkupfalse%
\ Int{\isacharunderscore}{\kern0pt}IntF{\isacharcolon}{\kern0pt}\ {\isachardoublequoteopen}{\isacharparenleft}{\kern0pt}A\ {\isasyminter}{\isacharasterisk}{\kern0pt}\ Y{\isacharparenright}{\kern0pt}\ {\isasyminter}{\isacharasterisk}{\kern0pt}\ X\ {\isacharequal}{\kern0pt}\ A\ {\isasyminter}{\isacharasterisk}{\kern0pt}\ {\isacharparenleft}{\kern0pt}Y\ {\isasyminter}\ X{\isacharparenright}{\kern0pt}{\isachardoublequoteclose}\isanewline
%
\isadelimproof
%
\endisadelimproof
%
\isatagproof
\isacommand{proof}\isamarkupfalse%
\isanewline
\ \ \isacommand{show}\isamarkupfalse%
\ {\isachardoublequoteopen}A\ {\isasyminter}{\isacharasterisk}{\kern0pt}\ Y\ {\isasyminter}{\isacharasterisk}{\kern0pt}\ X\ {\isasymsubseteq}\ A\ {\isasyminter}{\isacharasterisk}{\kern0pt}\ {\isacharparenleft}{\kern0pt}Y\ {\isasyminter}\ X{\isacharparenright}{\kern0pt}{\isachardoublequoteclose}\isanewline
\ \ \isacommand{proof}\isamarkupfalse%
\isanewline
\ \ \ \ \isacommand{fix}\isamarkupfalse%
\ S\isanewline
\ \ \ \ \isacommand{assume}\isamarkupfalse%
\ {\isachardoublequoteopen}S\ {\isasymin}\ A\ {\isasyminter}{\isacharasterisk}{\kern0pt}\ Y\ {\isasyminter}{\isacharasterisk}{\kern0pt}\ X{\isachardoublequoteclose}\isanewline
\ \ \ \ \isacommand{then}\isamarkupfalse%
\ \isacommand{obtain}\isamarkupfalse%
\ a{\isacharunderscore}{\kern0pt}y\ \isakeyword{where}\ A{\isacharunderscore}{\kern0pt}Y{\isadigit{0}}{\isacharcolon}{\kern0pt}\ {\isachardoublequoteopen}a{\isacharunderscore}{\kern0pt}y\ {\isasymin}\ A\ {\isasyminter}{\isacharasterisk}{\kern0pt}\ Y{\isachardoublequoteclose}\ \isakeyword{and}\ A{\isacharunderscore}{\kern0pt}Y{\isadigit{1}}{\isacharcolon}{\kern0pt}\ {\isachardoublequoteopen}a{\isacharunderscore}{\kern0pt}y\ {\isasyminter}\ X\ {\isacharequal}{\kern0pt}\ S{\isachardoublequoteclose}\ \isacommand{by}\isamarkupfalse%
\ blast\isanewline
\ \ \ \ \isacommand{from}\isamarkupfalse%
\ A{\isacharunderscore}{\kern0pt}Y{\isadigit{0}}\ \isacommand{obtain}\isamarkupfalse%
\ a\ \isakeyword{where}\ A{\isadigit{0}}{\isacharcolon}{\kern0pt}\ {\isachardoublequoteopen}a\ {\isasymin}\ A{\isachardoublequoteclose}\ \isakeyword{and}\ A{\isadigit{1}}{\isacharcolon}{\kern0pt}\ {\isachardoublequoteopen}a\ {\isasyminter}\ Y\ {\isacharequal}{\kern0pt}\ a{\isacharunderscore}{\kern0pt}y{\isachardoublequoteclose}\ \isacommand{by}\isamarkupfalse%
\ blast\isanewline
\ \ \ \ \isacommand{from}\isamarkupfalse%
\ A{\isacharunderscore}{\kern0pt}Y{\isadigit{1}}\ A{\isadigit{1}}\ \isacommand{have}\isamarkupfalse%
\ {\isachardoublequoteopen}a\ {\isasyminter}\ {\isacharparenleft}{\kern0pt}Y\ {\isasyminter}\ X{\isacharparenright}{\kern0pt}\ {\isacharequal}{\kern0pt}\ S{\isachardoublequoteclose}\ \isacommand{by}\isamarkupfalse%
\ fast\isanewline
\ \ \ \ \isacommand{with}\isamarkupfalse%
\ A{\isadigit{0}}\ \isacommand{show}\isamarkupfalse%
\ {\isachardoublequoteopen}S\ {\isasymin}\ A\ {\isasyminter}{\isacharasterisk}{\kern0pt}\ {\isacharparenleft}{\kern0pt}Y\ {\isasyminter}\ X{\isacharparenright}{\kern0pt}{\isachardoublequoteclose}\ \isacommand{by}\isamarkupfalse%
\ blast\isanewline
\ \ \isacommand{qed}\isamarkupfalse%
\isanewline
\isacommand{next}\isamarkupfalse%
\isanewline
\ \ \isacommand{show}\isamarkupfalse%
\ {\isachardoublequoteopen}A\ {\isasyminter}{\isacharasterisk}{\kern0pt}\ {\isacharparenleft}{\kern0pt}Y\ {\isasyminter}\ X{\isacharparenright}{\kern0pt}\ {\isasymsubseteq}\ A\ {\isasyminter}{\isacharasterisk}{\kern0pt}\ Y\ {\isasyminter}{\isacharasterisk}{\kern0pt}\ X{\isachardoublequoteclose}\isanewline
\ \ \isacommand{proof}\isamarkupfalse%
\isanewline
\ \ \ \ \isacommand{fix}\isamarkupfalse%
\ S\isanewline
\ \ \ \ \isacommand{assume}\isamarkupfalse%
\ {\isachardoublequoteopen}S\ {\isasymin}\ A\ {\isasyminter}{\isacharasterisk}{\kern0pt}\ {\isacharparenleft}{\kern0pt}Y\ {\isasyminter}\ X{\isacharparenright}{\kern0pt}{\isachardoublequoteclose}\isanewline
\ \ \ \ \isacommand{then}\isamarkupfalse%
\ \isacommand{obtain}\isamarkupfalse%
\ a\ \isakeyword{where}\ A{\isadigit{0}}{\isacharcolon}{\kern0pt}\ {\isachardoublequoteopen}a\ {\isasymin}\ A{\isachardoublequoteclose}\ \isakeyword{and}\ A{\isadigit{1}}{\isacharcolon}{\kern0pt}\ {\isachardoublequoteopen}a\ {\isasyminter}\ {\isacharparenleft}{\kern0pt}Y\ {\isasyminter}\ X{\isacharparenright}{\kern0pt}\ {\isacharequal}{\kern0pt}\ S{\isachardoublequoteclose}\ \isacommand{by}\isamarkupfalse%
\ blast\isanewline
\ \ \ \ \isacommand{from}\isamarkupfalse%
\ A{\isadigit{0}}\ \isacommand{have}\isamarkupfalse%
\ {\isachardoublequoteopen}a\ {\isasyminter}\ Y\ {\isasymin}\ A\ {\isasyminter}{\isacharasterisk}{\kern0pt}\ Y{\isachardoublequoteclose}\ \isacommand{by}\isamarkupfalse%
\ blast\isanewline
\ \ \ \ \isacommand{with}\isamarkupfalse%
\ A{\isadigit{1}}\ \isacommand{show}\isamarkupfalse%
\ {\isachardoublequoteopen}S\ {\isasymin}\ {\isacharparenleft}{\kern0pt}A\ {\isasyminter}{\isacharasterisk}{\kern0pt}\ Y{\isacharparenright}{\kern0pt}\ {\isasyminter}{\isacharasterisk}{\kern0pt}\ X{\isachardoublequoteclose}\ \isacommand{by}\isamarkupfalse%
\ blast\isanewline
\ \ \isacommand{qed}\isamarkupfalse%
\isanewline
\isacommand{qed}\isamarkupfalse%
%
\endisatagproof
{\isafoldproof}%
%
\isadelimproof
%
\endisadelimproof
%
\begin{isamarkuptext}%
insert distributes over IntF%
\end{isamarkuptext}\isamarkuptrue%
\isacommand{lemma}\isamarkupfalse%
\ insert{\isacharunderscore}{\kern0pt}IntF{\isacharcolon}{\kern0pt}\ \isanewline
\ \ \isakeyword{shows}\ {\isachardoublequoteopen}insert\ x\ {\isacharbackquote}{\kern0pt}\ {\isacharparenleft}{\kern0pt}H\ {\isasyminter}{\isacharasterisk}{\kern0pt}\ S{\isacharparenright}{\kern0pt}\ {\isacharequal}{\kern0pt}\ {\isacharparenleft}{\kern0pt}insert\ x\ {\isacharbackquote}{\kern0pt}\ H{\isacharparenright}{\kern0pt}\ {\isasyminter}{\isacharasterisk}{\kern0pt}\ {\isacharparenleft}{\kern0pt}insert\ x\ S{\isacharparenright}{\kern0pt}{\isachardoublequoteclose}\isanewline
%
\isadelimproof
%
\endisadelimproof
%
\isatagproof
\isacommand{proof}\isamarkupfalse%
\isanewline
\ \ \isacommand{show}\isamarkupfalse%
\ {\isachardoublequoteopen}insert\ x\ {\isacharbackquote}{\kern0pt}\ {\isacharparenleft}{\kern0pt}H\ {\isasyminter}{\isacharasterisk}{\kern0pt}\ S{\isacharparenright}{\kern0pt}\ {\isasymsubseteq}\ {\isacharparenleft}{\kern0pt}insert\ x\ {\isacharbackquote}{\kern0pt}\ H{\isacharparenright}{\kern0pt}\ {\isasyminter}{\isacharasterisk}{\kern0pt}\ {\isacharparenleft}{\kern0pt}insert\ x\ S{\isacharparenright}{\kern0pt}{\isachardoublequoteclose}\isanewline
\ \ \isacommand{proof}\isamarkupfalse%
\isanewline
\ \ \ \ \isacommand{fix}\isamarkupfalse%
\ y{\isacharunderscore}{\kern0pt}x\isanewline
\ \ \ \ \isacommand{assume}\isamarkupfalse%
\ {\isachardoublequoteopen}y{\isacharunderscore}{\kern0pt}x\ {\isasymin}\ insert\ x\ {\isacharbackquote}{\kern0pt}\ {\isacharparenleft}{\kern0pt}H\ {\isasyminter}{\isacharasterisk}{\kern0pt}\ S{\isacharparenright}{\kern0pt}{\isachardoublequoteclose}\isanewline
\ \ \ \ \isacommand{then}\isamarkupfalse%
\ \isacommand{obtain}\isamarkupfalse%
\ y\ \isakeyword{where}\ {\isadigit{0}}{\isacharcolon}{\kern0pt}\ {\isachardoublequoteopen}y\ {\isasymin}\ {\isacharparenleft}{\kern0pt}H\ {\isasyminter}{\isacharasterisk}{\kern0pt}\ S{\isacharparenright}{\kern0pt}{\isachardoublequoteclose}\ \isakeyword{and}\ {\isadigit{1}}{\isacharcolon}{\kern0pt}\ {\isachardoublequoteopen}y{\isacharunderscore}{\kern0pt}x\ {\isacharequal}{\kern0pt}\ y\ {\isasymunion}\ {\isacharbraceleft}{\kern0pt}x{\isacharbraceright}{\kern0pt}{\isachardoublequoteclose}\ \isacommand{by}\isamarkupfalse%
\ blast\isanewline
\ \ \ \ \isacommand{from}\isamarkupfalse%
\ {\isadigit{0}}\ \isacommand{obtain}\isamarkupfalse%
\ yh\ \isakeyword{where}\ {\isadigit{2}}{\isacharcolon}{\kern0pt}\ {\isachardoublequoteopen}yh\ {\isasymin}\ H{\isachardoublequoteclose}\ \isakeyword{and}\ {\isadigit{3}}{\isacharcolon}{\kern0pt}\ {\isachardoublequoteopen}y\ {\isacharequal}{\kern0pt}\ yh\ {\isasyminter}\ S{\isachardoublequoteclose}\ \isacommand{by}\isamarkupfalse%
\ blast\isanewline
\ \ \ \ \isacommand{from}\isamarkupfalse%
\ {\isadigit{1}}\ {\isadigit{3}}\ \isacommand{have}\isamarkupfalse%
\ {\isachardoublequoteopen}y{\isacharunderscore}{\kern0pt}x\ {\isacharequal}{\kern0pt}\ {\isacharparenleft}{\kern0pt}yh\ {\isasymunion}\ {\isacharbraceleft}{\kern0pt}x{\isacharbraceright}{\kern0pt}{\isacharparenright}{\kern0pt}\ {\isasyminter}\ {\isacharparenleft}{\kern0pt}S\ {\isasymunion}\ {\isacharbraceleft}{\kern0pt}x{\isacharbraceright}{\kern0pt}{\isacharparenright}{\kern0pt}{\isachardoublequoteclose}\ \isacommand{by}\isamarkupfalse%
\ simp\isanewline
\ \ \ \ \isacommand{with}\isamarkupfalse%
\ {\isadigit{2}}\ \isacommand{show}\isamarkupfalse%
\ {\isachardoublequoteopen}y{\isacharunderscore}{\kern0pt}x\ {\isasymin}\ {\isacharparenleft}{\kern0pt}insert\ x\ {\isacharbackquote}{\kern0pt}\ H{\isacharparenright}{\kern0pt}\ {\isasyminter}{\isacharasterisk}{\kern0pt}\ {\isacharparenleft}{\kern0pt}insert\ x\ S{\isacharparenright}{\kern0pt}{\isachardoublequoteclose}\ \isacommand{by}\isamarkupfalse%
\ blast\isanewline
\ \ \isacommand{qed}\isamarkupfalse%
\isanewline
\isacommand{next}\isamarkupfalse%
\isanewline
\ \ \isacommand{show}\isamarkupfalse%
\ {\isachardoublequoteopen}insert\ x\ {\isacharbackquote}{\kern0pt}\ H\ {\isasyminter}{\isacharasterisk}{\kern0pt}\ {\isacharparenleft}{\kern0pt}insert\ x\ S{\isacharparenright}{\kern0pt}\ {\isasymsubseteq}\ insert\ x\ {\isacharbackquote}{\kern0pt}\ {\isacharparenleft}{\kern0pt}H\ {\isasyminter}{\isacharasterisk}{\kern0pt}\ S{\isacharparenright}{\kern0pt}{\isachardoublequoteclose}\isanewline
\ \ \isacommand{proof}\isamarkupfalse%
\isanewline
\ \ \ \ \isacommand{fix}\isamarkupfalse%
\ y{\isacharunderscore}{\kern0pt}x\isanewline
\ \ \ \ \isacommand{assume}\isamarkupfalse%
\ {\isachardoublequoteopen}y{\isacharunderscore}{\kern0pt}x\ {\isasymin}\ insert\ x\ {\isacharbackquote}{\kern0pt}\ H\ {\isasyminter}{\isacharasterisk}{\kern0pt}\ {\isacharparenleft}{\kern0pt}insert\ x\ S{\isacharparenright}{\kern0pt}{\isachardoublequoteclose}\isanewline
\ \ \ \ \isacommand{then}\isamarkupfalse%
\ \isacommand{obtain}\isamarkupfalse%
\ yh{\isacharunderscore}{\kern0pt}x\ \isakeyword{where}\ {\isadigit{0}}{\isacharcolon}{\kern0pt}\ {\isachardoublequoteopen}yh{\isacharunderscore}{\kern0pt}x\ {\isasymin}\ {\isacharparenleft}{\kern0pt}{\isasymlambda}Y{\isachardot}{\kern0pt}\ Y\ {\isasymunion}\ {\isacharbraceleft}{\kern0pt}x{\isacharbraceright}{\kern0pt}{\isacharparenright}{\kern0pt}\ {\isacharbackquote}{\kern0pt}\ H{\isachardoublequoteclose}\ \isakeyword{and}\ {\isadigit{1}}{\isacharcolon}{\kern0pt}\ {\isachardoublequoteopen}y{\isacharunderscore}{\kern0pt}x\ {\isacharequal}{\kern0pt}\ yh{\isacharunderscore}{\kern0pt}x\ {\isasyminter}\ {\isacharparenleft}{\kern0pt}S\ {\isasymunion}\ {\isacharbraceleft}{\kern0pt}x{\isacharbraceright}{\kern0pt}{\isacharparenright}{\kern0pt}{\isachardoublequoteclose}\ \isacommand{by}\isamarkupfalse%
\ blast\isanewline
\ \ \ \ \isacommand{from}\isamarkupfalse%
\ {\isadigit{0}}\ \isacommand{obtain}\isamarkupfalse%
\ yh\ \isakeyword{where}\ {\isadigit{2}}{\isacharcolon}{\kern0pt}\ {\isachardoublequoteopen}yh\ {\isasymin}\ H{\isachardoublequoteclose}\ \isakeyword{and}\ {\isadigit{3}}{\isacharcolon}{\kern0pt}\ {\isachardoublequoteopen}yh{\isacharunderscore}{\kern0pt}x\ {\isacharequal}{\kern0pt}\ yh\ {\isasymunion}\ {\isacharbraceleft}{\kern0pt}x{\isacharbraceright}{\kern0pt}{\isachardoublequoteclose}\ \isacommand{by}\isamarkupfalse%
\ blast\isanewline
\ \ \ \ \isacommand{from}\isamarkupfalse%
\ {\isadigit{1}}\ {\isadigit{3}}\ \isacommand{have}\isamarkupfalse%
\ {\isachardoublequoteopen}y{\isacharunderscore}{\kern0pt}x\ {\isacharequal}{\kern0pt}\ {\isacharparenleft}{\kern0pt}yh\ {\isasyminter}\ S{\isacharparenright}{\kern0pt}\ {\isasymunion}\ {\isacharbraceleft}{\kern0pt}x{\isacharbraceright}{\kern0pt}{\isachardoublequoteclose}\ \isacommand{by}\isamarkupfalse%
\ simp\isanewline
\ \ \ \ \isacommand{with}\isamarkupfalse%
\ {\isadigit{2}}\ \isacommand{show}\isamarkupfalse%
\ {\isachardoublequoteopen}y{\isacharunderscore}{\kern0pt}x\ {\isasymin}\ insert\ x\ {\isacharbackquote}{\kern0pt}\ {\isacharparenleft}{\kern0pt}H\ {\isasyminter}{\isacharasterisk}{\kern0pt}\ S{\isacharparenright}{\kern0pt}{\isachardoublequoteclose}\ \isacommand{by}\isamarkupfalse%
\ blast\isanewline
\ \ \isacommand{qed}\isamarkupfalse%
\isanewline
\isacommand{qed}\isamarkupfalse%
%
\endisatagproof
{\isafoldproof}%
%
\isadelimproof
%
\endisadelimproof
%
\isadelimdocument
%
\endisadelimdocument
%
\isatagdocument
%
\isamarkupsubsection{Definition of shattering%
}
\isamarkuptrue%
%
\endisatagdocument
{\isafolddocument}%
%
\isadelimdocument
%
\endisadelimdocument
\isacommand{abbreviation}\isamarkupfalse%
\ shatters\ {\isacharcolon}{\kern0pt}{\isacharcolon}{\kern0pt}\ {\isachardoublequoteopen}{\isacharprime}{\kern0pt}a\ set\ set\ {\isasymRightarrow}\ {\isacharprime}{\kern0pt}a\ set\ {\isasymRightarrow}\ bool{\isachardoublequoteclose}\ {\isacharparenleft}{\kern0pt}\isakeyword{infixl}\ {\isachardoublequoteopen}shatters{\isachardoublequoteclose}\ {\isadigit{7}}{\isadigit{0}}{\isacharparenright}{\kern0pt}\isanewline
\ \ \isakeyword{where}\ {\isachardoublequoteopen}H\ shatters\ A\ {\isasymequiv}\ H\ {\isasyminter}{\isacharasterisk}{\kern0pt}\ A\ {\isacharequal}{\kern0pt}\ Pow\ A{\isachardoublequoteclose}\isanewline
\isanewline
\isacommand{definition}\isamarkupfalse%
\ VC{\isacharunderscore}{\kern0pt}dim\ {\isacharcolon}{\kern0pt}{\isacharcolon}{\kern0pt}\ {\isachardoublequoteopen}{\isacharprime}{\kern0pt}a\ set\ set\ {\isasymRightarrow}\ nat{\isachardoublequoteclose}\isanewline
\ \ \isakeyword{where}\ {\isachardoublequoteopen}VC{\isacharunderscore}{\kern0pt}dim\ F\ {\isacharequal}{\kern0pt}\ Sup\ {\isacharbraceleft}{\kern0pt}card\ S\ {\isacharbar}{\kern0pt}\ S{\isachardot}{\kern0pt}\ F\ shatters\ S{\isacharbraceright}{\kern0pt}{\isachardoublequoteclose}\isanewline
\isanewline
\isacommand{definition}\isamarkupfalse%
\ shattered{\isacharunderscore}{\kern0pt}by\ {\isacharcolon}{\kern0pt}{\isacharcolon}{\kern0pt}\ {\isachardoublequoteopen}{\isacharprime}{\kern0pt}a\ set\ set\ {\isasymRightarrow}\ {\isacharprime}{\kern0pt}a\ set\ set{\isachardoublequoteclose}\isanewline
\ \ \isakeyword{where}\ {\isachardoublequoteopen}shattered{\isacharunderscore}{\kern0pt}by\ F\ {\isasymequiv}\ {\isacharbraceleft}{\kern0pt}A{\isachardot}{\kern0pt}\ F\ shatters\ A{\isacharbraceright}{\kern0pt}{\isachardoublequoteclose}\isanewline
\isanewline
\isacommand{lemma}\isamarkupfalse%
\ shattered{\isacharunderscore}{\kern0pt}by{\isacharunderscore}{\kern0pt}in{\isacharunderscore}{\kern0pt}Pow{\isacharcolon}{\kern0pt}\isanewline
\ \ \isakeyword{shows}\ {\isachardoublequoteopen}shattered{\isacharunderscore}{\kern0pt}by\ F\ {\isasymsubseteq}\ Pow\ {\isacharparenleft}{\kern0pt}{\isasymUnion}\ F{\isacharparenright}{\kern0pt}{\isachardoublequoteclose}\isanewline
%
\isadelimproof
\ \ %
\endisadelimproof
%
\isatagproof
\isacommand{unfolding}\isamarkupfalse%
\ shattered{\isacharunderscore}{\kern0pt}by{\isacharunderscore}{\kern0pt}def\ \isacommand{by}\isamarkupfalse%
\ blast%
\endisatagproof
{\isafoldproof}%
%
\isadelimproof
\isanewline
%
\endisadelimproof
\isanewline
\isacommand{lemma}\isamarkupfalse%
\ subset{\isacharunderscore}{\kern0pt}shatters{\isacharcolon}{\kern0pt}\isanewline
\ \ \isakeyword{assumes}\ {\isachardoublequoteopen}A\ {\isasymsubseteq}\ B{\isachardoublequoteclose}\ \isakeyword{and}\ {\isachardoublequoteopen}A\ shatters\ X{\isachardoublequoteclose}\isanewline
\ \ \isakeyword{shows}\ {\isachardoublequoteopen}B\ shatters\ X{\isachardoublequoteclose}\isanewline
%
\isadelimproof
%
\endisadelimproof
%
\isatagproof
\isacommand{proof}\isamarkupfalse%
\ {\isacharminus}{\kern0pt}\isanewline
\ \ \isacommand{from}\isamarkupfalse%
\ assms{\isacharparenleft}{\kern0pt}{\isadigit{1}}{\isacharparenright}{\kern0pt}\ \isacommand{have}\isamarkupfalse%
\ {\isachardoublequoteopen}A\ {\isasyminter}{\isacharasterisk}{\kern0pt}\ X\ {\isasymsubseteq}\ B\ {\isasyminter}{\isacharasterisk}{\kern0pt}\ X{\isachardoublequoteclose}\ \isacommand{by}\isamarkupfalse%
\ blast\isanewline
\ \ \isacommand{with}\isamarkupfalse%
\ assms{\isacharparenleft}{\kern0pt}{\isadigit{2}}{\isacharparenright}{\kern0pt}\ \isacommand{have}\isamarkupfalse%
\ {\isachardoublequoteopen}Pow\ X\ {\isasymsubseteq}\ B\ {\isasyminter}{\isacharasterisk}{\kern0pt}\ X{\isachardoublequoteclose}\ \ \isacommand{by}\isamarkupfalse%
\ presburger\isanewline
\ \ \isacommand{thus}\isamarkupfalse%
\ {\isacharquery}{\kern0pt}thesis\ \isacommand{by}\isamarkupfalse%
\ blast\isanewline
\isacommand{qed}\isamarkupfalse%
%
\endisatagproof
{\isafoldproof}%
%
\isadelimproof
\isanewline
%
\endisadelimproof
\isanewline
\isacommand{lemma}\isamarkupfalse%
\ supset{\isacharunderscore}{\kern0pt}shatters{\isacharcolon}{\kern0pt}\isanewline
\ \ \isakeyword{assumes}\ {\isachardoublequoteopen}Y\ {\isasymsubseteq}\ X{\isachardoublequoteclose}\ \isakeyword{and}\ {\isachardoublequoteopen}A\ shatters\ X{\isachardoublequoteclose}\isanewline
\ \ \isakeyword{shows}\ {\isachardoublequoteopen}A\ shatters\ Y{\isachardoublequoteclose}\isanewline
%
\isadelimproof
%
\endisadelimproof
%
\isatagproof
\isacommand{proof}\isamarkupfalse%
\ {\isacharminus}{\kern0pt}\isanewline
\ \ \isacommand{have}\isamarkupfalse%
\ h{\isacharcolon}{\kern0pt}\ {\isachardoublequoteopen}{\isasymUnion}{\isacharparenleft}{\kern0pt}Pow\ Y{\isacharparenright}{\kern0pt}\ {\isasymsubseteq}\ Y{\isachardoublequoteclose}\ \isacommand{by}\isamarkupfalse%
\ simp\isanewline
\ \ \isacommand{from}\isamarkupfalse%
\ assms\ \isacommand{have}\isamarkupfalse%
\ {\isadigit{0}}{\isacharcolon}{\kern0pt}\ {\isachardoublequoteopen}Pow\ Y\ {\isasymsubseteq}\ A\ {\isasyminter}{\isacharasterisk}{\kern0pt}\ X{\isachardoublequoteclose}\ \isacommand{by}\isamarkupfalse%
\ auto\isanewline
\ \ \isacommand{from}\isamarkupfalse%
\ subset{\isacharunderscore}{\kern0pt}IntF{\isacharbrackleft}{\kern0pt}OF\ {\isadigit{0}}{\isacharcomma}{\kern0pt}\ of\ Y{\isacharbrackright}{\kern0pt}\ Int{\isacharunderscore}{\kern0pt}IntF{\isacharbrackleft}{\kern0pt}of\ Y\ X\ A{\isacharbrackright}{\kern0pt}\ idem{\isacharunderscore}{\kern0pt}IntF{\isacharbrackleft}{\kern0pt}OF\ h{\isacharbrackright}{\kern0pt}\ \isacommand{have}\isamarkupfalse%
\ {\isachardoublequoteopen}Pow\ Y\ {\isasymsubseteq}\ A\ {\isasyminter}{\isacharasterisk}{\kern0pt}\ {\isacharparenleft}{\kern0pt}X\ {\isasyminter}\ Y{\isacharparenright}{\kern0pt}{\isachardoublequoteclose}\ \isacommand{by}\isamarkupfalse%
\ argo\isanewline
\ \ \isacommand{with}\isamarkupfalse%
\ Int{\isacharunderscore}{\kern0pt}absorb{\isadigit{2}}{\isacharbrackleft}{\kern0pt}OF\ assms{\isacharparenleft}{\kern0pt}{\isadigit{1}}{\isacharparenright}{\kern0pt}{\isacharbrackright}{\kern0pt}\ Int{\isacharunderscore}{\kern0pt}commute{\isacharbrackleft}{\kern0pt}of\ X\ Y{\isacharbrackright}{\kern0pt}\ \isacommand{have}\isamarkupfalse%
\ {\isachardoublequoteopen}Pow\ Y\ {\isasymsubseteq}\ A\ {\isasyminter}{\isacharasterisk}{\kern0pt}\ Y{\isachardoublequoteclose}\ \isacommand{by}\isamarkupfalse%
\ presburger\isanewline
\ \ \isacommand{then}\isamarkupfalse%
\ \isacommand{show}\isamarkupfalse%
\ {\isacharquery}{\kern0pt}thesis\ \isacommand{by}\isamarkupfalse%
\ fast\isanewline
\isacommand{qed}\isamarkupfalse%
%
\endisatagproof
{\isafoldproof}%
%
\isadelimproof
\isanewline
%
\endisadelimproof
\isanewline
\isacommand{lemma}\isamarkupfalse%
\ shatters{\isacharunderscore}{\kern0pt}empty{\isacharcolon}{\kern0pt}\isanewline
\ \ \isakeyword{assumes}\ {\isachardoublequoteopen}F\ {\isasymnoteq}\ {\isacharbraceleft}{\kern0pt}{\isacharbraceright}{\kern0pt}{\isachardoublequoteclose}\isanewline
\ \ \isakeyword{shows}\ {\isachardoublequoteopen}F\ shatters\ {\isacharbraceleft}{\kern0pt}{\isacharbraceright}{\kern0pt}{\isachardoublequoteclose}\ \isanewline
%
\isadelimproof
%
\endisadelimproof
%
\isatagproof
\isacommand{using}\isamarkupfalse%
\ assms\ \isacommand{by}\isamarkupfalse%
\ fastforce%
\endisatagproof
{\isafoldproof}%
%
\isadelimproof
\isanewline
%
\endisadelimproof
\isanewline
\isacommand{lemma}\isamarkupfalse%
\ subset{\isacharunderscore}{\kern0pt}shattered{\isacharunderscore}{\kern0pt}by{\isacharcolon}{\kern0pt}\isanewline
\ \ \isakeyword{assumes}\ {\isachardoublequoteopen}A\ {\isasymsubseteq}\ B{\isachardoublequoteclose}\isanewline
\ \ \isakeyword{shows}\ {\isachardoublequoteopen}shattered{\isacharunderscore}{\kern0pt}by\ A\ {\isasymsubseteq}\ shattered{\isacharunderscore}{\kern0pt}by\ B{\isachardoublequoteclose}\ \isanewline
%
\isadelimproof
%
\endisadelimproof
%
\isatagproof
\isacommand{unfolding}\isamarkupfalse%
\ shattered{\isacharunderscore}{\kern0pt}by{\isacharunderscore}{\kern0pt}def\ \isacommand{using}\isamarkupfalse%
\ subset{\isacharunderscore}{\kern0pt}shatters{\isacharbrackleft}{\kern0pt}OF\ assms{\isacharbrackright}{\kern0pt}\ \isacommand{by}\isamarkupfalse%
\ force%
\endisatagproof
{\isafoldproof}%
%
\isadelimproof
\isanewline
%
\endisadelimproof
\isanewline
\isacommand{lemma}\isamarkupfalse%
\ finite{\isacharunderscore}{\kern0pt}shattered{\isacharunderscore}{\kern0pt}by{\isacharcolon}{\kern0pt}\isanewline
\ \ \isakeyword{assumes}\ {\isachardoublequoteopen}finite\ {\isacharparenleft}{\kern0pt}{\isasymUnion}\ F{\isacharparenright}{\kern0pt}{\isachardoublequoteclose}\isanewline
\ \ \isakeyword{shows}\ {\isachardoublequoteopen}finite\ {\isacharparenleft}{\kern0pt}shattered{\isacharunderscore}{\kern0pt}by\ F{\isacharparenright}{\kern0pt}{\isachardoublequoteclose}\isanewline
%
\isadelimproof
\ \ %
\endisadelimproof
%
\isatagproof
\isacommand{using}\isamarkupfalse%
\ assms\ rev{\isacharunderscore}{\kern0pt}finite{\isacharunderscore}{\kern0pt}subset{\isacharbrackleft}{\kern0pt}OF\ {\isacharunderscore}{\kern0pt}\ shattered{\isacharunderscore}{\kern0pt}by{\isacharunderscore}{\kern0pt}in{\isacharunderscore}{\kern0pt}Pow{\isacharcomma}{\kern0pt}\ of\ F{\isacharbrackright}{\kern0pt}\ \isacommand{by}\isamarkupfalse%
\ fast%
\endisatagproof
{\isafoldproof}%
%
\isadelimproof
%
\endisadelimproof
%
\begin{isamarkuptext}%
The following example shows that requiring finiteness of a family of sets is not enough%
\end{isamarkuptext}\isamarkuptrue%
\isacommand{lemma}\isamarkupfalse%
\ {\isachardoublequoteopen}{\isasymexists}F{\isacharcolon}{\kern0pt}{\isacharcolon}{\kern0pt}nat\ set\ set{\isachardot}{\kern0pt}\ finite\ F\ {\isasymand}\ infinite\ {\isacharparenleft}{\kern0pt}shattered{\isacharunderscore}{\kern0pt}by\ F{\isacharparenright}{\kern0pt}{\isachardoublequoteclose}\isanewline
%
\isadelimproof
%
\endisadelimproof
%
\isatagproof
\isacommand{proof}\isamarkupfalse%
\ {\isacharminus}{\kern0pt}\ \ \ \ \ \ \ \ \ \ \ \isanewline
\ \ \isacommand{let}\isamarkupfalse%
\ {\isacharquery}{\kern0pt}F\ {\isacharequal}{\kern0pt}\ {\isachardoublequoteopen}{\isacharbraceleft}{\kern0pt}odd\ {\isacharminus}{\kern0pt}{\isacharbackquote}{\kern0pt}\ {\isacharbraceleft}{\kern0pt}True{\isacharbraceright}{\kern0pt}{\isacharcomma}{\kern0pt}\ odd\ {\isacharminus}{\kern0pt}{\isacharbackquote}{\kern0pt}\ {\isacharbraceleft}{\kern0pt}False{\isacharbraceright}{\kern0pt}{\isacharbraceright}{\kern0pt}{\isachardoublequoteclose}\isanewline
\ \ \isacommand{have}\isamarkupfalse%
\ {\isadigit{0}}{\isacharcolon}{\kern0pt}\ {\isachardoublequoteopen}finite\ {\isacharquery}{\kern0pt}F{\isachardoublequoteclose}\ \isacommand{by}\isamarkupfalse%
\ simp\isanewline
\isanewline
\ \ \isacommand{let}\isamarkupfalse%
\ {\isacharquery}{\kern0pt}f\ {\isacharequal}{\kern0pt}\ {\isachardoublequoteopen}{\isasymlambda}n{\isacharcolon}{\kern0pt}{\isacharcolon}{\kern0pt}nat{\isachardot}{\kern0pt}\ {\isacharbraceleft}{\kern0pt}n{\isacharbraceright}{\kern0pt}{\isachardoublequoteclose}\ \isanewline
\ \ \isacommand{let}\isamarkupfalse%
\ {\isacharquery}{\kern0pt}N\ {\isacharequal}{\kern0pt}\ {\isachardoublequoteopen}range\ {\isacharquery}{\kern0pt}f{\isachardoublequoteclose}\isanewline
\ \ \isacommand{have}\isamarkupfalse%
\ {\isachardoublequoteopen}inj\ {\isacharparenleft}{\kern0pt}{\isasymlambda}n{\isachardot}{\kern0pt}\ {\isacharbraceleft}{\kern0pt}n{\isacharbraceright}{\kern0pt}{\isacharparenright}{\kern0pt}{\isachardoublequoteclose}\ \isacommand{by}\isamarkupfalse%
\ simp\isanewline
\ \ \isacommand{with}\isamarkupfalse%
\ infinite{\isacharunderscore}{\kern0pt}iff{\isacharunderscore}{\kern0pt}countable{\isacharunderscore}{\kern0pt}subset{\isacharbrackleft}{\kern0pt}of\ {\isacharquery}{\kern0pt}N{\isacharbrackright}{\kern0pt}\ \isacommand{have}\isamarkupfalse%
\ infinite{\isacharunderscore}{\kern0pt}N{\isacharcolon}{\kern0pt}\ {\isachardoublequoteopen}infinite\ {\isacharquery}{\kern0pt}N{\isachardoublequoteclose}\ \isacommand{by}\isamarkupfalse%
\ blast\isanewline
\ \ \isacommand{have}\isamarkupfalse%
\ F{\isacharunderscore}{\kern0pt}shatters{\isacharunderscore}{\kern0pt}any{\isacharunderscore}{\kern0pt}singleton{\isacharcolon}{\kern0pt}\ {\isachardoublequoteopen}{\isacharquery}{\kern0pt}F\ shatters\ {\isacharbraceleft}{\kern0pt}n{\isacharcolon}{\kern0pt}{\isacharcolon}{\kern0pt}nat{\isacharbraceright}{\kern0pt}{\isachardoublequoteclose}\ \isakeyword{for}\ n\isanewline
\ \ \isacommand{proof}\isamarkupfalse%
\ {\isacharminus}{\kern0pt}\isanewline
\ \ \ \ \isacommand{have}\isamarkupfalse%
\ Pow{\isacharunderscore}{\kern0pt}n{\isacharcolon}{\kern0pt}\ {\isachardoublequoteopen}Pow\ {\isacharbraceleft}{\kern0pt}n{\isacharbraceright}{\kern0pt}\ {\isacharequal}{\kern0pt}\ {\isacharbraceleft}{\kern0pt}{\isacharbraceleft}{\kern0pt}n{\isacharbraceright}{\kern0pt}{\isacharcomma}{\kern0pt}\ {\isacharbraceleft}{\kern0pt}{\isacharbraceright}{\kern0pt}{\isacharbraceright}{\kern0pt}{\isachardoublequoteclose}\ \isacommand{by}\isamarkupfalse%
\ blast\isanewline
\ \ \ \ \isacommand{have}\isamarkupfalse%
\ {\isadigit{1}}{\isacharcolon}{\kern0pt}\ {\isachardoublequoteopen}Pow\ {\isacharbraceleft}{\kern0pt}n{\isacharbraceright}{\kern0pt}\ {\isasymsubseteq}\ {\isacharquery}{\kern0pt}F\ {\isasyminter}{\isacharasterisk}{\kern0pt}\ {\isacharbraceleft}{\kern0pt}n{\isacharbraceright}{\kern0pt}{\isachardoublequoteclose}\ \isanewline
\ \ \ \ \isacommand{proof}\isamarkupfalse%
\ {\isacharparenleft}{\kern0pt}cases\ {\isachardoublequoteopen}odd\ n{\isachardoublequoteclose}{\isacharparenright}{\kern0pt}\isanewline
\ \ \ \ \ \ \isacommand{case}\isamarkupfalse%
\ True\isanewline
\ \ \ \ \ \ \isacommand{from}\isamarkupfalse%
\ True\ \isacommand{have}\isamarkupfalse%
\ {\isachardoublequoteopen}{\isacharparenleft}{\kern0pt}odd\ {\isacharminus}{\kern0pt}{\isacharbackquote}{\kern0pt}\ {\isacharbraceleft}{\kern0pt}False{\isacharbraceright}{\kern0pt}{\isacharparenright}{\kern0pt}\ {\isasyminter}\ {\isacharbraceleft}{\kern0pt}n{\isacharbraceright}{\kern0pt}\ {\isacharequal}{\kern0pt}\ {\isacharbraceleft}{\kern0pt}{\isacharbraceright}{\kern0pt}{\isachardoublequoteclose}\ \isacommand{by}\isamarkupfalse%
\ blast\isanewline
\ \ \ \ \ \ \isacommand{hence}\isamarkupfalse%
\ {\isadigit{0}}{\isacharcolon}{\kern0pt}\ {\isachardoublequoteopen}{\isacharbraceleft}{\kern0pt}{\isacharbraceright}{\kern0pt}\ {\isasymin}\ {\isacharquery}{\kern0pt}F\ {\isasyminter}{\isacharasterisk}{\kern0pt}\ {\isacharbraceleft}{\kern0pt}n{\isacharbraceright}{\kern0pt}{\isachardoublequoteclose}\ \ \isacommand{by}\isamarkupfalse%
\ blast\isanewline
\ \ \ \ \ \ \isacommand{from}\isamarkupfalse%
\ True\ \isacommand{have}\isamarkupfalse%
\ {\isachardoublequoteopen}{\isacharparenleft}{\kern0pt}odd\ {\isacharminus}{\kern0pt}{\isacharbackquote}{\kern0pt}\ {\isacharbraceleft}{\kern0pt}True{\isacharbraceright}{\kern0pt}{\isacharparenright}{\kern0pt}\ {\isasyminter}\ {\isacharbraceleft}{\kern0pt}n{\isacharbraceright}{\kern0pt}\ {\isacharequal}{\kern0pt}\ {\isacharbraceleft}{\kern0pt}n{\isacharbraceright}{\kern0pt}{\isachardoublequoteclose}\ \isacommand{by}\isamarkupfalse%
\ blast\isanewline
\ \ \ \ \ \ \isacommand{hence}\isamarkupfalse%
\ {\isadigit{1}}{\isacharcolon}{\kern0pt}\ {\isachardoublequoteopen}{\isacharbraceleft}{\kern0pt}n{\isacharbraceright}{\kern0pt}\ {\isasymin}\ {\isacharquery}{\kern0pt}F\ {\isasyminter}{\isacharasterisk}{\kern0pt}\ {\isacharbraceleft}{\kern0pt}n{\isacharbraceright}{\kern0pt}{\isachardoublequoteclose}\ \ \isacommand{by}\isamarkupfalse%
\ blast\isanewline
\ \ \ \ \ \ \isacommand{from}\isamarkupfalse%
\ {\isadigit{0}}\ {\isadigit{1}}\ Pow{\isacharunderscore}{\kern0pt}n\ \isacommand{show}\isamarkupfalse%
\ {\isacharquery}{\kern0pt}thesis\ \isacommand{by}\isamarkupfalse%
\ simp\isanewline
\ \ \ \ \isacommand{next}\isamarkupfalse%
\isanewline
\ \ \ \ \ \ \isacommand{case}\isamarkupfalse%
\ False\isanewline
\ \ \ \ \ \ \isacommand{from}\isamarkupfalse%
\ False\ \isacommand{have}\isamarkupfalse%
\ {\isachardoublequoteopen}{\isacharparenleft}{\kern0pt}odd\ {\isacharminus}{\kern0pt}{\isacharbackquote}{\kern0pt}\ {\isacharbraceleft}{\kern0pt}True{\isacharbraceright}{\kern0pt}{\isacharparenright}{\kern0pt}\ {\isasyminter}\ {\isacharbraceleft}{\kern0pt}n{\isacharbraceright}{\kern0pt}\ {\isacharequal}{\kern0pt}\ {\isacharbraceleft}{\kern0pt}{\isacharbraceright}{\kern0pt}{\isachardoublequoteclose}\ \isacommand{by}\isamarkupfalse%
\ blast\isanewline
\ \ \ \ \ \ \isacommand{hence}\isamarkupfalse%
\ {\isadigit{0}}{\isacharcolon}{\kern0pt}\ {\isachardoublequoteopen}{\isacharbraceleft}{\kern0pt}{\isacharbraceright}{\kern0pt}\ {\isasymin}\ {\isacharquery}{\kern0pt}F\ {\isasyminter}{\isacharasterisk}{\kern0pt}\ {\isacharbraceleft}{\kern0pt}n{\isacharbraceright}{\kern0pt}{\isachardoublequoteclose}\ \isacommand{by}\isamarkupfalse%
\ blast\isanewline
\ \ \ \ \ \ \isacommand{from}\isamarkupfalse%
\ False\ \isacommand{have}\isamarkupfalse%
\ {\isachardoublequoteopen}{\isacharparenleft}{\kern0pt}odd\ {\isacharminus}{\kern0pt}{\isacharbackquote}{\kern0pt}\ {\isacharbraceleft}{\kern0pt}False{\isacharbraceright}{\kern0pt}{\isacharparenright}{\kern0pt}\ {\isasyminter}\ {\isacharbraceleft}{\kern0pt}n{\isacharbraceright}{\kern0pt}\ {\isacharequal}{\kern0pt}\ {\isacharbraceleft}{\kern0pt}n{\isacharbraceright}{\kern0pt}{\isachardoublequoteclose}\ \isacommand{by}\isamarkupfalse%
\ blast\isanewline
\ \ \ \ \ \ \isacommand{hence}\isamarkupfalse%
\ {\isadigit{1}}{\isacharcolon}{\kern0pt}\ {\isachardoublequoteopen}{\isacharbraceleft}{\kern0pt}n{\isacharbraceright}{\kern0pt}\ {\isasymin}\ {\isacharquery}{\kern0pt}F\ {\isasyminter}{\isacharasterisk}{\kern0pt}\ {\isacharbraceleft}{\kern0pt}n{\isacharbraceright}{\kern0pt}{\isachardoublequoteclose}\ \isacommand{by}\isamarkupfalse%
\ blast\isanewline
\ \ \ \ \ \ \isacommand{from}\isamarkupfalse%
\ {\isadigit{0}}\ {\isadigit{1}}\ Pow{\isacharunderscore}{\kern0pt}n\ \isacommand{show}\isamarkupfalse%
\ {\isacharquery}{\kern0pt}thesis\ \isacommand{by}\isamarkupfalse%
\ simp\isanewline
\ \ \ \ \isacommand{qed}\isamarkupfalse%
\isanewline
\ \ \ \ \isacommand{thus}\isamarkupfalse%
\ {\isacharquery}{\kern0pt}thesis\ \isacommand{by}\isamarkupfalse%
\ fastforce\isanewline
\ \ \isacommand{qed}\isamarkupfalse%
\isanewline
\ \ \isacommand{then}\isamarkupfalse%
\ \isacommand{have}\isamarkupfalse%
\ {\isachardoublequoteopen}{\isacharquery}{\kern0pt}N\ {\isasymsubseteq}\ shattered{\isacharunderscore}{\kern0pt}by\ {\isacharquery}{\kern0pt}F{\isachardoublequoteclose}\ \isacommand{unfolding}\isamarkupfalse%
\ shattered{\isacharunderscore}{\kern0pt}by{\isacharunderscore}{\kern0pt}def\ \isacommand{by}\isamarkupfalse%
\ force\isanewline
\ \ \isacommand{from}\isamarkupfalse%
\ {\isadigit{0}}\ infinite{\isacharunderscore}{\kern0pt}super{\isacharbrackleft}{\kern0pt}OF\ this\ infinite{\isacharunderscore}{\kern0pt}N{\isacharbrackright}{\kern0pt}\ \isacommand{show}\isamarkupfalse%
\ {\isacharquery}{\kern0pt}thesis\ \isacommand{by}\isamarkupfalse%
\ blast\isanewline
\isacommand{qed}\isamarkupfalse%
%
\endisatagproof
{\isafoldproof}%
%
\isadelimproof
\isanewline
%
\endisadelimproof
\isanewline
%
\isadelimtheory
\isanewline
%
\endisadelimtheory
%
\isatagtheory
\isacommand{end}\isamarkupfalse%
%
\endisatagtheory
{\isafoldtheory}%
%
\isadelimtheory
%
\endisadelimtheory
%
\end{isabellebody}%
\endinput
%:%file=Shattering.tex%:%
%:%11=5%:%
%:%27=7%:%
%:%28=7%:%
%:%29=8%:%
%:%30=9%:%
%:%44=11%:%
%:%54=13%:%
%:%55=13%:%
%:%56=14%:%
%:%57=15%:%
%:%58=16%:%
%:%59=16%:%
%:%60=17%:%
%:%61=18%:%
%:%68=19%:%
%:%69=19%:%
%:%70=20%:%
%:%71=20%:%
%:%72=20%:%
%:%73=20%:%
%:%74=21%:%
%:%75=21%:%
%:%76=21%:%
%:%77=22%:%
%:%83=22%:%
%:%86=23%:%
%:%87=24%:%
%:%88=24%:%
%:%89=25%:%
%:%90=26%:%
%:%93=27%:%
%:%97=27%:%
%:%98=27%:%
%:%99=27%:%
%:%104=27%:%
%:%107=28%:%
%:%108=29%:%
%:%109=29%:%
%:%116=30%:%
%:%117=30%:%
%:%118=31%:%
%:%119=31%:%
%:%120=32%:%
%:%121=32%:%
%:%122=33%:%
%:%123=33%:%
%:%124=34%:%
%:%125=34%:%
%:%126=35%:%
%:%127=35%:%
%:%128=35%:%
%:%129=35%:%
%:%130=36%:%
%:%131=36%:%
%:%132=36%:%
%:%133=36%:%
%:%134=37%:%
%:%135=37%:%
%:%136=37%:%
%:%137=37%:%
%:%138=38%:%
%:%139=38%:%
%:%140=38%:%
%:%141=38%:%
%:%142=39%:%
%:%143=39%:%
%:%144=40%:%
%:%145=40%:%
%:%146=41%:%
%:%147=41%:%
%:%148=42%:%
%:%149=42%:%
%:%150=43%:%
%:%151=43%:%
%:%152=44%:%
%:%153=44%:%
%:%154=45%:%
%:%155=45%:%
%:%156=45%:%
%:%157=45%:%
%:%158=46%:%
%:%159=46%:%
%:%160=46%:%
%:%161=46%:%
%:%162=47%:%
%:%163=47%:%
%:%164=47%:%
%:%165=47%:%
%:%166=48%:%
%:%167=48%:%
%:%168=49%:%
%:%178=51%:%
%:%180=52%:%
%:%181=52%:%
%:%182=53%:%
%:%189=54%:%
%:%190=54%:%
%:%191=55%:%
%:%192=55%:%
%:%193=56%:%
%:%194=56%:%
%:%195=57%:%
%:%196=57%:%
%:%197=58%:%
%:%198=58%:%
%:%199=59%:%
%:%200=59%:%
%:%201=59%:%
%:%202=59%:%
%:%203=60%:%
%:%204=60%:%
%:%205=60%:%
%:%206=60%:%
%:%207=61%:%
%:%208=61%:%
%:%209=61%:%
%:%210=61%:%
%:%211=62%:%
%:%212=62%:%
%:%213=62%:%
%:%214=62%:%
%:%215=63%:%
%:%216=63%:%
%:%217=64%:%
%:%218=64%:%
%:%219=65%:%
%:%220=65%:%
%:%221=66%:%
%:%222=66%:%
%:%223=67%:%
%:%224=67%:%
%:%225=68%:%
%:%226=68%:%
%:%227=69%:%
%:%228=69%:%
%:%229=69%:%
%:%230=69%:%
%:%231=70%:%
%:%232=70%:%
%:%233=70%:%
%:%234=70%:%
%:%235=71%:%
%:%236=71%:%
%:%237=71%:%
%:%238=71%:%
%:%239=72%:%
%:%240=72%:%
%:%241=72%:%
%:%242=72%:%
%:%243=73%:%
%:%244=73%:%
%:%245=74%:%
%:%260=76%:%
%:%270=78%:%
%:%271=78%:%
%:%272=79%:%
%:%273=80%:%
%:%274=81%:%
%:%275=81%:%
%:%276=82%:%
%:%277=83%:%
%:%278=84%:%
%:%279=84%:%
%:%280=85%:%
%:%281=86%:%
%:%282=87%:%
%:%283=87%:%
%:%284=88%:%
%:%287=89%:%
%:%291=89%:%
%:%292=89%:%
%:%293=89%:%
%:%298=89%:%
%:%301=90%:%
%:%302=91%:%
%:%303=91%:%
%:%304=92%:%
%:%305=93%:%
%:%312=94%:%
%:%313=94%:%
%:%314=95%:%
%:%315=95%:%
%:%316=95%:%
%:%317=95%:%
%:%318=96%:%
%:%319=96%:%
%:%320=96%:%
%:%321=96%:%
%:%322=97%:%
%:%323=97%:%
%:%324=97%:%
%:%325=98%:%
%:%331=98%:%
%:%334=99%:%
%:%335=100%:%
%:%336=100%:%
%:%337=101%:%
%:%338=102%:%
%:%345=103%:%
%:%346=103%:%
%:%347=104%:%
%:%348=104%:%
%:%349=104%:%
%:%350=105%:%
%:%351=105%:%
%:%352=105%:%
%:%353=105%:%
%:%354=106%:%
%:%355=106%:%
%:%356=106%:%
%:%357=106%:%
%:%358=107%:%
%:%359=107%:%
%:%360=107%:%
%:%361=107%:%
%:%362=108%:%
%:%363=108%:%
%:%364=108%:%
%:%365=108%:%
%:%366=109%:%
%:%372=109%:%
%:%375=110%:%
%:%376=111%:%
%:%377=111%:%
%:%378=112%:%
%:%379=113%:%
%:%386=114%:%
%:%387=114%:%
%:%388=114%:%
%:%393=114%:%
%:%396=115%:%
%:%397=116%:%
%:%398=116%:%
%:%399=117%:%
%:%400=118%:%
%:%407=119%:%
%:%408=119%:%
%:%409=119%:%
%:%410=119%:%
%:%415=119%:%
%:%418=120%:%
%:%419=121%:%
%:%420=121%:%
%:%421=122%:%
%:%422=123%:%
%:%425=124%:%
%:%429=124%:%
%:%430=124%:%
%:%431=124%:%
%:%440=126%:%
%:%442=127%:%
%:%443=127%:%
%:%450=128%:%
%:%451=128%:%
%:%452=129%:%
%:%453=129%:%
%:%454=130%:%
%:%455=130%:%
%:%456=130%:%
%:%457=131%:%
%:%458=132%:%
%:%459=132%:%
%:%460=133%:%
%:%461=133%:%
%:%462=134%:%
%:%463=134%:%
%:%464=134%:%
%:%465=135%:%
%:%466=135%:%
%:%467=135%:%
%:%468=135%:%
%:%469=136%:%
%:%470=136%:%
%:%471=137%:%
%:%472=137%:%
%:%473=138%:%
%:%474=138%:%
%:%475=138%:%
%:%476=139%:%
%:%477=139%:%
%:%478=140%:%
%:%479=140%:%
%:%480=141%:%
%:%481=141%:%
%:%482=142%:%
%:%483=142%:%
%:%484=142%:%
%:%485=142%:%
%:%486=143%:%
%:%487=143%:%
%:%488=143%:%
%:%489=144%:%
%:%490=144%:%
%:%491=144%:%
%:%492=144%:%
%:%493=145%:%
%:%494=145%:%
%:%495=145%:%
%:%496=146%:%
%:%497=146%:%
%:%498=146%:%
%:%499=146%:%
%:%500=147%:%
%:%501=147%:%
%:%502=148%:%
%:%503=148%:%
%:%504=149%:%
%:%505=149%:%
%:%506=149%:%
%:%507=149%:%
%:%508=150%:%
%:%509=150%:%
%:%510=150%:%
%:%511=151%:%
%:%512=151%:%
%:%513=151%:%
%:%514=151%:%
%:%515=152%:%
%:%516=152%:%
%:%517=152%:%
%:%518=153%:%
%:%519=153%:%
%:%520=153%:%
%:%521=153%:%
%:%522=154%:%
%:%523=154%:%
%:%524=155%:%
%:%525=155%:%
%:%526=155%:%
%:%527=156%:%
%:%528=156%:%
%:%529=157%:%
%:%530=157%:%
%:%531=157%:%
%:%532=157%:%
%:%533=157%:%
%:%534=158%:%
%:%535=158%:%
%:%536=158%:%
%:%537=158%:%
%:%538=159%:%
%:%544=159%:%
%:%547=160%:%
%:%550=161%:%
%:%555=162%:%

%
\begin{isabellebody}%
\setisabellecontext{Card{\isacharunderscore}{\kern0pt}Lemmas}%
%
\isadelimdocument
%
\endisadelimdocument
%
\isatagdocument
%
\isamarkupsection{Lemmas involving the cardinality of sets%
}
\isamarkuptrue%
%
\endisatagdocument
{\isafolddocument}%
%
\isadelimdocument
%
\endisadelimdocument
%
\begin{isamarkuptext}%
In this section, we prove some lemmas that make use of the term \isa{card} or provide bounds for it.%
\end{isamarkuptext}\isamarkuptrue%
%
\isadelimtheory
%
\endisadelimtheory
%
\isatagtheory
\isacommand{theory}\isamarkupfalse%
\ Card{\isacharunderscore}{\kern0pt}Lemmas\isanewline
\ \ \isakeyword{imports}\ Main\isanewline
\isakeyword{begin}%
\endisatagtheory
{\isafoldtheory}%
%
\isadelimtheory
%
\endisadelimtheory
\isanewline
\isanewline
\isacommand{lemma}\isamarkupfalse%
\ card{\isacharunderscore}{\kern0pt}Int{\isacharunderscore}{\kern0pt}copy{\isacharcolon}{\kern0pt}\isanewline
\ \ \isakeyword{assumes}\ {\isachardoublequoteopen}finite\ X{\isachardoublequoteclose}\ \isakeyword{and}\ {\isachardoublequoteopen}A\ {\isasymunion}\ B\ {\isasymsubseteq}\ X{\isachardoublequoteclose}\ \isakeyword{and}\ {\isachardoublequoteopen}{\isasymexists}f{\isachardot}{\kern0pt}\ inj{\isacharunderscore}{\kern0pt}on\ f\ {\isacharparenleft}{\kern0pt}A\ {\isasyminter}\ B{\isacharparenright}{\kern0pt}\ {\isasymand}\ {\isacharparenleft}{\kern0pt}A\ {\isasymunion}\ B{\isacharparenright}{\kern0pt}\ {\isasyminter}\ {\isacharparenleft}{\kern0pt}f\ {\isacharbackquote}{\kern0pt}\ {\isacharparenleft}{\kern0pt}A\ {\isasyminter}\ B{\isacharparenright}{\kern0pt}{\isacharparenright}{\kern0pt}\ {\isacharequal}{\kern0pt}\ {\isacharbraceleft}{\kern0pt}{\isacharbraceright}{\kern0pt}\ {\isasymand}\ f\ {\isacharbackquote}{\kern0pt}\ {\isacharparenleft}{\kern0pt}A\ {\isasyminter}\ B{\isacharparenright}{\kern0pt}\ {\isasymsubseteq}\ X{\isachardoublequoteclose}\isanewline
\ \ \isakeyword{shows}\ {\isachardoublequoteopen}card\ A\ {\isacharplus}{\kern0pt}\ card\ B\ {\isasymle}\ card\ X{\isachardoublequoteclose}\isanewline
%
\isadelimproof
%
\endisadelimproof
%
\isatagproof
\isacommand{proof}\isamarkupfalse%
\ {\isacharminus}{\kern0pt}\isanewline
\ \ \isacommand{from}\isamarkupfalse%
\ rev{\isacharunderscore}{\kern0pt}finite{\isacharunderscore}{\kern0pt}subset{\isacharbrackleft}{\kern0pt}OF\ assms{\isacharparenleft}{\kern0pt}{\isadigit{1}}{\isacharparenright}{\kern0pt}{\isacharcomma}{\kern0pt}\ of\ A{\isacharbrackright}{\kern0pt}\ rev{\isacharunderscore}{\kern0pt}finite{\isacharunderscore}{\kern0pt}subset{\isacharbrackleft}{\kern0pt}OF\ assms{\isacharparenleft}{\kern0pt}{\isadigit{1}}{\isacharparenright}{\kern0pt}{\isacharcomma}{\kern0pt}\ of\ B{\isacharbrackright}{\kern0pt}\ assms{\isacharparenleft}{\kern0pt}{\isadigit{2}}{\isacharparenright}{\kern0pt}\ \isanewline
\ \ \isacommand{have}\isamarkupfalse%
\ finite{\isacharunderscore}{\kern0pt}A{\isacharcolon}{\kern0pt}\ {\isachardoublequoteopen}finite\ A{\isachardoublequoteclose}\ \isakeyword{and}\ finite{\isacharunderscore}{\kern0pt}B{\isacharcolon}{\kern0pt}\ {\isachardoublequoteopen}finite\ B{\isachardoublequoteclose}\ \isacommand{by}\isamarkupfalse%
\ blast{\isacharplus}{\kern0pt}\isanewline
\ \ \isacommand{then}\isamarkupfalse%
\ \isacommand{have}\isamarkupfalse%
\ finite{\isacharunderscore}{\kern0pt}A{\isacharunderscore}{\kern0pt}Un{\isacharunderscore}{\kern0pt}B{\isacharcolon}{\kern0pt}\ {\isachardoublequoteopen}finite\ {\isacharparenleft}{\kern0pt}A\ {\isasymunion}\ B{\isacharparenright}{\kern0pt}{\isachardoublequoteclose}\ \isakeyword{and}\ finite{\isacharunderscore}{\kern0pt}A{\isacharunderscore}{\kern0pt}Int{\isacharunderscore}{\kern0pt}B{\isacharcolon}{\kern0pt}\ {\isachardoublequoteopen}finite\ {\isacharparenleft}{\kern0pt}A\ {\isasyminter}\ B{\isacharparenright}{\kern0pt}{\isachardoublequoteclose}\ \isacommand{by}\isamarkupfalse%
\ blast{\isacharplus}{\kern0pt}\isanewline
\ \ \isacommand{from}\isamarkupfalse%
\ assms{\isacharparenleft}{\kern0pt}{\isadigit{3}}{\isacharparenright}{\kern0pt}\ \isacommand{obtain}\isamarkupfalse%
\ f\ \isakeyword{where}\ f{\isacharunderscore}{\kern0pt}inj{\isacharunderscore}{\kern0pt}on{\isacharcolon}{\kern0pt}\ {\isachardoublequoteopen}inj{\isacharunderscore}{\kern0pt}on\ f\ {\isacharparenleft}{\kern0pt}A\ {\isasyminter}\ B{\isacharparenright}{\kern0pt}{\isachardoublequoteclose}\ \isanewline
\ \ \ \ \ \ \ \ \ \ \ \ \ \ \ \ \ \ \ \ \ \ \ \ \ \ \ \isakeyword{and}\ f{\isacharunderscore}{\kern0pt}disjnt{\isacharcolon}{\kern0pt}\ {\isachardoublequoteopen}{\isacharparenleft}{\kern0pt}A\ {\isasymunion}\ B{\isacharparenright}{\kern0pt}\ {\isasyminter}\ {\isacharparenleft}{\kern0pt}f\ {\isacharbackquote}{\kern0pt}\ {\isacharparenleft}{\kern0pt}A\ {\isasyminter}\ B{\isacharparenright}{\kern0pt}{\isacharparenright}{\kern0pt}\ {\isacharequal}{\kern0pt}\ {\isacharbraceleft}{\kern0pt}{\isacharbraceright}{\kern0pt}{\isachardoublequoteclose}\ \isanewline
\ \ \ \ \ \ \ \ \ \ \ \ \ \ \ \ \ \ \ \ \ \ \ \ \ \ \ \isakeyword{and}\ f{\isacharunderscore}{\kern0pt}imj{\isacharunderscore}{\kern0pt}in{\isacharcolon}{\kern0pt}\ {\isachardoublequoteopen}f\ {\isacharbackquote}{\kern0pt}\ {\isacharparenleft}{\kern0pt}A\ {\isasyminter}\ B{\isacharparenright}{\kern0pt}\ {\isasymsubseteq}\ X{\isachardoublequoteclose}\ \isacommand{by}\isamarkupfalse%
\ blast\isanewline
\ \ \isacommand{from}\isamarkupfalse%
\ finite{\isacharunderscore}{\kern0pt}A{\isacharunderscore}{\kern0pt}Int{\isacharunderscore}{\kern0pt}B\ \isacommand{have}\isamarkupfalse%
\ finite{\isacharunderscore}{\kern0pt}f{\isacharunderscore}{\kern0pt}img{\isacharcolon}{\kern0pt}\ {\isachardoublequoteopen}finite\ {\isacharparenleft}{\kern0pt}f\ {\isacharbackquote}{\kern0pt}\ {\isacharparenleft}{\kern0pt}A\ {\isasyminter}\ B{\isacharparenright}{\kern0pt}{\isacharparenright}{\kern0pt}{\isachardoublequoteclose}\ \isacommand{by}\isamarkupfalse%
\ blast\isanewline
\ \ \isacommand{from}\isamarkupfalse%
\ assms{\isacharparenleft}{\kern0pt}{\isadigit{2}}{\isacharparenright}{\kern0pt}\ f{\isacharunderscore}{\kern0pt}imj{\isacharunderscore}{\kern0pt}in\ \isacommand{have}\isamarkupfalse%
\ union{\isacharunderscore}{\kern0pt}in{\isacharcolon}{\kern0pt}\ {\isachardoublequoteopen}{\isacharparenleft}{\kern0pt}A\ {\isasymunion}\ B{\isacharparenright}{\kern0pt}\ {\isasymunion}\ f\ {\isacharbackquote}{\kern0pt}\ {\isacharparenleft}{\kern0pt}A\ {\isasyminter}\ B{\isacharparenright}{\kern0pt}\ {\isasymsubseteq}\ X{\isachardoublequoteclose}\ \isacommand{by}\isamarkupfalse%
\ blast\isanewline
\ \ \isanewline
\ \ \isacommand{from}\isamarkupfalse%
\ card{\isacharunderscore}{\kern0pt}Un{\isacharunderscore}{\kern0pt}Int{\isacharbrackleft}{\kern0pt}OF\ finite{\isacharunderscore}{\kern0pt}A\ finite{\isacharunderscore}{\kern0pt}B{\isacharbrackright}{\kern0pt}\ \isacommand{have}\isamarkupfalse%
\ {\isachardoublequoteopen}card\ A\ {\isacharplus}{\kern0pt}\ card\ B\ {\isacharequal}{\kern0pt}\ card\ {\isacharparenleft}{\kern0pt}A\ {\isasymunion}\ B{\isacharparenright}{\kern0pt}\ {\isacharplus}{\kern0pt}\ card\ {\isacharparenleft}{\kern0pt}A\ {\isasyminter}\ B{\isacharparenright}{\kern0pt}{\isachardoublequoteclose}\ \isacommand{{\isachardot}{\kern0pt}}\isamarkupfalse%
\isanewline
\ \ \isacommand{also}\isamarkupfalse%
\ \isacommand{from}\isamarkupfalse%
\ card{\isacharunderscore}{\kern0pt}image{\isacharbrackleft}{\kern0pt}OF\ f{\isacharunderscore}{\kern0pt}inj{\isacharunderscore}{\kern0pt}on{\isacharbrackright}{\kern0pt}\ \isacommand{have}\isamarkupfalse%
\ {\isachardoublequoteopen}{\isachardot}{\kern0pt}{\isachardot}{\kern0pt}{\isachardot}{\kern0pt}\ {\isacharequal}{\kern0pt}\ card\ {\isacharparenleft}{\kern0pt}A\ {\isasymunion}\ B{\isacharparenright}{\kern0pt}\ {\isacharplus}{\kern0pt}\ card\ {\isacharparenleft}{\kern0pt}f\ {\isacharbackquote}{\kern0pt}\ {\isacharparenleft}{\kern0pt}A\ {\isasyminter}\ B{\isacharparenright}{\kern0pt}{\isacharparenright}{\kern0pt}{\isachardoublequoteclose}\ \isacommand{by}\isamarkupfalse%
\ presburger\isanewline
\ \ \isacommand{also}\isamarkupfalse%
\ \isacommand{from}\isamarkupfalse%
\ card{\isacharunderscore}{\kern0pt}Un{\isacharunderscore}{\kern0pt}disjoint{\isacharbrackleft}{\kern0pt}OF\ finite{\isacharunderscore}{\kern0pt}A{\isacharunderscore}{\kern0pt}Un{\isacharunderscore}{\kern0pt}B\ finite{\isacharunderscore}{\kern0pt}f{\isacharunderscore}{\kern0pt}img\ f{\isacharunderscore}{\kern0pt}disjnt{\isacharbrackright}{\kern0pt}\ \isacommand{have}\isamarkupfalse%
\ {\isachardoublequoteopen}{\isachardot}{\kern0pt}{\isachardot}{\kern0pt}{\isachardot}{\kern0pt}\ {\isacharequal}{\kern0pt}\ card\ {\isacharparenleft}{\kern0pt}{\isacharparenleft}{\kern0pt}A\ {\isasymunion}\ B{\isacharparenright}{\kern0pt}\ {\isasymunion}\ f\ {\isacharbackquote}{\kern0pt}\ {\isacharparenleft}{\kern0pt}A\ {\isasyminter}\ B{\isacharparenright}{\kern0pt}{\isacharparenright}{\kern0pt}{\isachardoublequoteclose}\ \isacommand{by}\isamarkupfalse%
\ argo\isanewline
\ \ \isacommand{also}\isamarkupfalse%
\ \isacommand{from}\isamarkupfalse%
\ card{\isacharunderscore}{\kern0pt}mono{\isacharbrackleft}{\kern0pt}OF\ assms{\isacharparenleft}{\kern0pt}{\isadigit{1}}{\isacharparenright}{\kern0pt}\ union{\isacharunderscore}{\kern0pt}in{\isacharbrackright}{\kern0pt}\ \isacommand{have}\isamarkupfalse%
\ {\isachardoublequoteopen}{\isachardot}{\kern0pt}{\isachardot}{\kern0pt}{\isachardot}{\kern0pt}\ {\isasymle}\ card\ X{\isachardoublequoteclose}\ \isacommand{by}\isamarkupfalse%
\ blast\isanewline
\ \ \isacommand{finally}\isamarkupfalse%
\ \isacommand{show}\isamarkupfalse%
\ {\isacharquery}{\kern0pt}thesis\ \isacommand{{\isachardot}{\kern0pt}}\isamarkupfalse%
\isanewline
\isacommand{qed}\isamarkupfalse%
%
\endisatagproof
{\isafoldproof}%
%
\isadelimproof
\isanewline
%
\endisadelimproof
\isanewline
\isacommand{lemma}\isamarkupfalse%
\ card{\isacharunderscore}{\kern0pt}ge{\isacharunderscore}{\kern0pt}{\isadigit{0}}{\isacharcolon}{\kern0pt}\isanewline
\ \ \isakeyword{assumes}\ {\isachardoublequoteopen}A\ {\isasymnoteq}\ {\isacharbraceleft}{\kern0pt}{\isacharbraceright}{\kern0pt}{\isachardoublequoteclose}\ \isakeyword{and}\ {\isachardoublequoteopen}finite\ A{\isachardoublequoteclose}\isanewline
\ \ \isakeyword{shows}\ {\isachardoublequoteopen}{\isadigit{0}}\ {\isacharless}{\kern0pt}\ card\ A{\isachardoublequoteclose}\isanewline
%
\isadelimproof
%
\endisadelimproof
%
\isatagproof
\isacommand{proof}\isamarkupfalse%
\ {\isacharminus}{\kern0pt}\isanewline
\ \ \isacommand{from}\isamarkupfalse%
\ assms{\isacharparenleft}{\kern0pt}{\isadigit{1}}{\isacharparenright}{\kern0pt}\ \isacommand{have}\isamarkupfalse%
\ {\isachardoublequoteopen}{\isacharbraceleft}{\kern0pt}{\isacharbraceright}{\kern0pt}\ {\isasymsubset}\ A{\isachardoublequoteclose}\ \isacommand{by}\isamarkupfalse%
\ blast\isanewline
\ \ \isacommand{from}\isamarkupfalse%
\ psubset{\isacharunderscore}{\kern0pt}card{\isacharunderscore}{\kern0pt}mono{\isacharbrackleft}{\kern0pt}OF\ assms{\isacharparenleft}{\kern0pt}{\isadigit{2}}{\isacharparenright}{\kern0pt}\ this{\isacharbrackright}{\kern0pt}\ \isacommand{show}\isamarkupfalse%
\ {\isacharquery}{\kern0pt}thesis\ \isacommand{by}\isamarkupfalse%
\ force\isanewline
\isacommand{qed}\isamarkupfalse%
%
\endisatagproof
{\isafoldproof}%
%
\isadelimproof
\isanewline
%
\endisadelimproof
\isanewline
\isacommand{lemma}\isamarkupfalse%
\ finite{\isacharunderscore}{\kern0pt}diff{\isacharunderscore}{\kern0pt}not{\isacharunderscore}{\kern0pt}empty{\isacharcolon}{\kern0pt}\ \isanewline
\ \ \isakeyword{assumes}\ {\isachardoublequoteopen}finite\ Y{\isachardoublequoteclose}\ \isakeyword{and}\ {\isachardoublequoteopen}card\ Y\ {\isacharless}{\kern0pt}\ card\ X{\isachardoublequoteclose}\isanewline
\ \ \isakeyword{shows}\ {\isachardoublequoteopen}X\ {\isacharminus}{\kern0pt}\ Y\ {\isasymnoteq}\ {\isacharbraceleft}{\kern0pt}{\isacharbraceright}{\kern0pt}{\isachardoublequoteclose}\isanewline
%
\isadelimproof
%
\endisadelimproof
%
\isatagproof
\isacommand{proof}\isamarkupfalse%
\isanewline
\ \ \isacommand{assume}\isamarkupfalse%
\ {\isachardoublequoteopen}X\ {\isacharminus}{\kern0pt}\ Y\ {\isacharequal}{\kern0pt}\ {\isacharbraceleft}{\kern0pt}{\isacharbraceright}{\kern0pt}{\isachardoublequoteclose}\isanewline
\ \ \isacommand{hence}\isamarkupfalse%
\ {\isachardoublequoteopen}X\ {\isasymsubseteq}\ Y{\isachardoublequoteclose}\ \isacommand{by}\isamarkupfalse%
\ simp\isanewline
\ \ \isacommand{from}\isamarkupfalse%
\ card{\isacharunderscore}{\kern0pt}mono{\isacharbrackleft}{\kern0pt}OF\ assms{\isacharparenleft}{\kern0pt}{\isadigit{1}}{\isacharparenright}{\kern0pt}\ this{\isacharbrackright}{\kern0pt}\ assms{\isacharparenleft}{\kern0pt}{\isadigit{2}}{\isacharparenright}{\kern0pt}\ \isacommand{show}\isamarkupfalse%
\ False\ \isacommand{by}\isamarkupfalse%
\ linarith\isanewline
\isacommand{qed}\isamarkupfalse%
%
\endisatagproof
{\isafoldproof}%
%
\isadelimproof
\isanewline
%
\endisadelimproof
\isanewline
\isacommand{lemma}\isamarkupfalse%
\ obtain{\isacharunderscore}{\kern0pt}difference{\isacharunderscore}{\kern0pt}element{\isacharcolon}{\kern0pt}\isanewline
\ \ \isakeyword{fixes}\ F\ {\isacharcolon}{\kern0pt}{\isacharcolon}{\kern0pt}\ {\isachardoublequoteopen}{\isacharprime}{\kern0pt}a\ set\ set{\isachardoublequoteclose}\isanewline
\ \ \isakeyword{assumes}\ {\isachardoublequoteopen}{\isadigit{2}}\ {\isasymle}\ card\ F{\isachardoublequoteclose}\isanewline
\ \ \isakeyword{obtains}\ {\isachardoublequoteopen}x{\isachardoublequoteclose}\ \isakeyword{where}\ {\isachardoublequoteopen}x{\isasymin}\ {\isasymUnion}F{\isachardoublequoteclose}\ {\isachardoublequoteopen}x\ {\isasymnotin}\ {\isasymInter}F{\isachardoublequoteclose}\isanewline
%
\isadelimproof
%
\endisadelimproof
%
\isatagproof
\isacommand{proof}\isamarkupfalse%
\ {\isacharminus}{\kern0pt}\isanewline
\ \ \isacommand{from}\isamarkupfalse%
\ assms\ card{\isacharunderscore}{\kern0pt}le{\isacharunderscore}{\kern0pt}Suc{\isacharunderscore}{\kern0pt}iff{\isacharbrackleft}{\kern0pt}of\ {\isadigit{1}}\ F{\isacharbrackright}{\kern0pt}\ \isacommand{obtain}\isamarkupfalse%
\ A\ F{\isacharprime}{\kern0pt}\ \isakeyword{where}\ {\isadigit{0}}{\isacharcolon}{\kern0pt}\ {\isachardoublequoteopen}F\ {\isacharequal}{\kern0pt}\ insert\ A\ F{\isacharprime}{\kern0pt}{\isachardoublequoteclose}\ \isakeyword{and}\ {\isadigit{1}}{\isacharcolon}{\kern0pt}\ {\isachardoublequoteopen}A\ {\isasymnotin}\ F{\isacharprime}{\kern0pt}{\isachardoublequoteclose}\ \isakeyword{and}\ {\isadigit{2}}{\isacharcolon}{\kern0pt}\ {\isachardoublequoteopen}{\isadigit{1}}\ {\isasymle}\ card\ F{\isacharprime}{\kern0pt}{\isachardoublequoteclose}\ \isacommand{by}\isamarkupfalse%
\ auto\isanewline
\ \ \isacommand{from}\isamarkupfalse%
\ {\isadigit{2}}\ card{\isacharunderscore}{\kern0pt}le{\isacharunderscore}{\kern0pt}Suc{\isacharunderscore}{\kern0pt}iff{\isacharbrackleft}{\kern0pt}of\ {\isadigit{0}}\ F{\isacharprime}{\kern0pt}{\isacharbrackright}{\kern0pt}\ \isacommand{obtain}\isamarkupfalse%
\ B\ F{\isacharprime}{\kern0pt}{\isacharprime}{\kern0pt}\ \isakeyword{where}\ {\isadigit{3}}{\isacharcolon}{\kern0pt}\ {\isachardoublequoteopen}F{\isacharprime}{\kern0pt}\ {\isacharequal}{\kern0pt}\ insert\ B\ F{\isacharprime}{\kern0pt}{\isacharprime}{\kern0pt}{\isachardoublequoteclose}\ \isacommand{by}\isamarkupfalse%
\ auto\isanewline
\ \ \isacommand{from}\isamarkupfalse%
\ {\isadigit{1}}\ {\isadigit{3}}\ \isacommand{have}\isamarkupfalse%
\ A{\isacharunderscore}{\kern0pt}noteq{\isacharunderscore}{\kern0pt}B{\isacharcolon}{\kern0pt}\ {\isachardoublequoteopen}A\ {\isasymnoteq}\ B{\isachardoublequoteclose}\ \isacommand{by}\isamarkupfalse%
\ blast\isanewline
\ \ \isacommand{from}\isamarkupfalse%
\ {\isadigit{0}}\ {\isadigit{3}}\ \isacommand{have}\isamarkupfalse%
\ A{\isacharunderscore}{\kern0pt}in{\isacharunderscore}{\kern0pt}F{\isacharcolon}{\kern0pt}\ {\isachardoublequoteopen}A\ {\isasymin}\ F{\isachardoublequoteclose}\ \isakeyword{and}\ B{\isacharunderscore}{\kern0pt}in{\isacharunderscore}{\kern0pt}F{\isacharcolon}{\kern0pt}\ {\isachardoublequoteopen}B\ {\isasymin}\ F{\isachardoublequoteclose}\ \isacommand{by}\isamarkupfalse%
\ blast{\isacharplus}{\kern0pt}\isanewline
\ \ \isacommand{from}\isamarkupfalse%
\ A{\isacharunderscore}{\kern0pt}noteq{\isacharunderscore}{\kern0pt}B\ \isacommand{have}\isamarkupfalse%
\ {\isachardoublequoteopen}{\isacharparenleft}{\kern0pt}A\ {\isacharminus}{\kern0pt}\ B{\isacharparenright}{\kern0pt}\ {\isasymunion}\ {\isacharparenleft}{\kern0pt}B\ {\isacharminus}{\kern0pt}\ A{\isacharparenright}{\kern0pt}\ {\isasymnoteq}\ {\isacharbraceleft}{\kern0pt}{\isacharbraceright}{\kern0pt}{\isachardoublequoteclose}\ \isacommand{by}\isamarkupfalse%
\ simp\isanewline
\ \ \isacommand{with}\isamarkupfalse%
\ A{\isacharunderscore}{\kern0pt}in{\isacharunderscore}{\kern0pt}F\ B{\isacharunderscore}{\kern0pt}in{\isacharunderscore}{\kern0pt}F\ that\ \isacommand{show}\isamarkupfalse%
\ thesis\ \isacommand{by}\isamarkupfalse%
\ blast\isanewline
\isacommand{qed}\isamarkupfalse%
%
\endisatagproof
{\isafoldproof}%
%
\isadelimproof
\isanewline
%
\endisadelimproof
%
\isadelimtheory
\isanewline
%
\endisadelimtheory
%
\isatagtheory
\isacommand{end}\isamarkupfalse%
%
\endisatagtheory
{\isafoldtheory}%
%
\isadelimtheory
%
\endisadelimtheory
%
\end{isabellebody}%
\endinput
%:%file=Card_Lemmas.tex%:%
%:%11=5%:%
%:%23=7%:%
%:%31=9%:%
%:%32=9%:%
%:%33=10%:%
%:%34=11%:%
%:%41=11%:%
%:%42=12%:%
%:%43=13%:%
%:%44=13%:%
%:%45=14%:%
%:%46=15%:%
%:%53=16%:%
%:%54=16%:%
%:%55=17%:%
%:%56=17%:%
%:%57=18%:%
%:%58=18%:%
%:%59=18%:%
%:%60=19%:%
%:%61=19%:%
%:%62=19%:%
%:%63=19%:%
%:%64=20%:%
%:%65=20%:%
%:%66=20%:%
%:%67=21%:%
%:%68=22%:%
%:%69=22%:%
%:%70=23%:%
%:%71=23%:%
%:%72=23%:%
%:%73=23%:%
%:%74=24%:%
%:%75=24%:%
%:%76=24%:%
%:%77=24%:%
%:%78=25%:%
%:%79=26%:%
%:%80=26%:%
%:%81=26%:%
%:%82=26%:%
%:%83=27%:%
%:%84=27%:%
%:%85=27%:%
%:%86=27%:%
%:%87=27%:%
%:%88=28%:%
%:%89=28%:%
%:%90=28%:%
%:%91=28%:%
%:%92=28%:%
%:%93=29%:%
%:%94=29%:%
%:%95=29%:%
%:%96=29%:%
%:%97=29%:%
%:%98=30%:%
%:%99=30%:%
%:%100=30%:%
%:%101=30%:%
%:%102=31%:%
%:%108=31%:%
%:%111=32%:%
%:%112=33%:%
%:%113=33%:%
%:%114=34%:%
%:%115=35%:%
%:%122=36%:%
%:%123=36%:%
%:%124=37%:%
%:%125=37%:%
%:%126=37%:%
%:%127=37%:%
%:%128=38%:%
%:%129=38%:%
%:%130=38%:%
%:%131=38%:%
%:%132=39%:%
%:%138=39%:%
%:%141=40%:%
%:%142=41%:%
%:%143=41%:%
%:%144=42%:%
%:%145=43%:%
%:%152=44%:%
%:%153=44%:%
%:%154=45%:%
%:%155=45%:%
%:%156=46%:%
%:%157=46%:%
%:%158=46%:%
%:%159=47%:%
%:%160=47%:%
%:%161=47%:%
%:%162=47%:%
%:%163=48%:%
%:%169=48%:%
%:%172=49%:%
%:%173=50%:%
%:%174=50%:%
%:%175=51%:%
%:%176=52%:%
%:%177=53%:%
%:%184=54%:%
%:%185=54%:%
%:%186=55%:%
%:%187=55%:%
%:%188=55%:%
%:%189=55%:%
%:%190=56%:%
%:%191=56%:%
%:%192=56%:%
%:%193=56%:%
%:%194=57%:%
%:%195=57%:%
%:%196=57%:%
%:%197=57%:%
%:%198=58%:%
%:%199=58%:%
%:%200=58%:%
%:%201=58%:%
%:%202=59%:%
%:%203=59%:%
%:%204=59%:%
%:%205=59%:%
%:%206=60%:%
%:%207=60%:%
%:%208=60%:%
%:%209=60%:%
%:%210=61%:%
%:%216=61%:%
%:%221=62%:%
%:%226=63%:%

%
\begin{isabellebody}%
\setisabellecontext{Binomial{\isacharunderscore}{\kern0pt}Lemmas}%
%
\isadelimdocument
%
\endisadelimdocument
%
\isatagdocument
%
\isamarkupsection{Lemmas involving the binomial coefficient%
}
\isamarkuptrue%
%
\endisatagdocument
{\isafolddocument}%
%
\isadelimdocument
%
\endisadelimdocument
%
\begin{isamarkuptext}%
In this section we prove lemmas that use the term for the binomial coefficient \isa{choose}.%
\end{isamarkuptext}\isamarkuptrue%
%
\isadelimtheory
%
\endisadelimtheory
%
\isatagtheory
\isacommand{theory}\isamarkupfalse%
\ Binomial{\isacharunderscore}{\kern0pt}Lemmas\isanewline
\ \ \isakeyword{imports}\ Main\isanewline
\isakeyword{begin}%
\endisatagtheory
{\isafoldtheory}%
%
\isadelimtheory
%
\endisadelimtheory
\isanewline
\isanewline
\isacommand{lemma}\isamarkupfalse%
\ choose{\isacharunderscore}{\kern0pt}mono{\isacharcolon}{\kern0pt}\isanewline
\ \ \isakeyword{assumes}\ {\isachardoublequoteopen}x\ {\isasymle}\ y{\isachardoublequoteclose}\isanewline
\ \ \isakeyword{shows}\ {\isachardoublequoteopen}x\ choose\ n\ {\isasymle}\ y\ choose\ n{\isachardoublequoteclose}\isanewline
%
\isadelimproof
%
\endisadelimproof
%
\isatagproof
\isacommand{proof}\isamarkupfalse%
\ {\isacharminus}{\kern0pt}\isanewline
\ \ \isacommand{have}\isamarkupfalse%
\ {\isachardoublequoteopen}finite\ {\isacharbraceleft}{\kern0pt}{\isadigit{0}}{\isachardot}{\kern0pt}{\isachardot}{\kern0pt}{\isacharless}{\kern0pt}y{\isacharbraceright}{\kern0pt}{\isachardoublequoteclose}\ \isacommand{by}\isamarkupfalse%
\ blast\isanewline
\ \ \isacommand{with}\isamarkupfalse%
\ finite{\isacharunderscore}{\kern0pt}Pow{\isacharunderscore}{\kern0pt}iff{\isacharbrackleft}{\kern0pt}of\ {\isachardoublequoteopen}{\isacharbraceleft}{\kern0pt}{\isadigit{0}}{\isachardot}{\kern0pt}{\isachardot}{\kern0pt}{\isacharless}{\kern0pt}y{\isacharbraceright}{\kern0pt}{\isachardoublequoteclose}{\isacharbrackright}{\kern0pt}\ \isacommand{have}\isamarkupfalse%
\ finiteness{\isacharcolon}{\kern0pt}\ {\isachardoublequoteopen}finite\ {\isacharbraceleft}{\kern0pt}K\ {\isasymin}\ Pow\ {\isacharbraceleft}{\kern0pt}{\isadigit{0}}{\isachardot}{\kern0pt}{\isachardot}{\kern0pt}{\isacharless}{\kern0pt}y{\isacharbraceright}{\kern0pt}{\isachardot}{\kern0pt}\ card\ K\ {\isacharequal}{\kern0pt}\ n{\isacharbraceright}{\kern0pt}{\isachardoublequoteclose}\ \isacommand{by}\isamarkupfalse%
\ simp\isanewline
\ \ \isacommand{from}\isamarkupfalse%
\ assms\ \isacommand{have}\isamarkupfalse%
\ {\isachardoublequoteopen}Pow\ {\isacharbraceleft}{\kern0pt}{\isadigit{0}}{\isachardot}{\kern0pt}{\isachardot}{\kern0pt}{\isacharless}{\kern0pt}x{\isacharbraceright}{\kern0pt}\ {\isasymsubseteq}\ Pow\ {\isacharbraceleft}{\kern0pt}{\isadigit{0}}{\isachardot}{\kern0pt}{\isachardot}{\kern0pt}{\isacharless}{\kern0pt}y{\isacharbraceright}{\kern0pt}{\isachardoublequoteclose}\ \isacommand{by}\isamarkupfalse%
\ force\isanewline
\ \ \isacommand{then}\isamarkupfalse%
\ \isacommand{have}\isamarkupfalse%
\ {\isachardoublequoteopen}{\isacharbraceleft}{\kern0pt}K\ {\isasymin}\ Pow\ {\isacharbraceleft}{\kern0pt}{\isadigit{0}}{\isachardot}{\kern0pt}{\isachardot}{\kern0pt}{\isacharless}{\kern0pt}x{\isacharbraceright}{\kern0pt}{\isachardot}{\kern0pt}\ card\ K\ {\isacharequal}{\kern0pt}\ n{\isacharbraceright}{\kern0pt}\ {\isasymsubseteq}\ {\isacharbraceleft}{\kern0pt}K\ {\isasymin}\ Pow\ {\isacharbraceleft}{\kern0pt}{\isadigit{0}}{\isachardot}{\kern0pt}{\isachardot}{\kern0pt}{\isacharless}{\kern0pt}y{\isacharbraceright}{\kern0pt}{\isachardot}{\kern0pt}\ card\ K\ {\isacharequal}{\kern0pt}\ n{\isacharbraceright}{\kern0pt}{\isachardoublequoteclose}\ \isacommand{by}\isamarkupfalse%
\ blast\isanewline
\ \ \isacommand{from}\isamarkupfalse%
\ card{\isacharunderscore}{\kern0pt}mono{\isacharbrackleft}{\kern0pt}OF\ finiteness\ this{\isacharbrackright}{\kern0pt}\ \isacommand{show}\isamarkupfalse%
\ {\isacharquery}{\kern0pt}thesis\ \isacommand{unfolding}\isamarkupfalse%
\ binomial{\isacharunderscore}{\kern0pt}def\ \isacommand{{\isachardot}{\kern0pt}}\isamarkupfalse%
\isanewline
\isacommand{qed}\isamarkupfalse%
%
\endisatagproof
{\isafoldproof}%
%
\isadelimproof
\isanewline
%
\endisadelimproof
\isanewline
\isacommand{lemma}\isamarkupfalse%
\ choose{\isacharunderscore}{\kern0pt}row{\isacharunderscore}{\kern0pt}sum{\isacharunderscore}{\kern0pt}set{\isacharcolon}{\kern0pt}\isanewline
\ \ \isakeyword{assumes}\ {\isachardoublequoteopen}finite\ {\isacharparenleft}{\kern0pt}{\isasymUnion}F{\isacharparenright}{\kern0pt}{\isachardoublequoteclose}\isanewline
\ \ \isakeyword{shows}\ {\isachardoublequoteopen}card\ {\isacharbraceleft}{\kern0pt}S{\isachardot}{\kern0pt}\ S\ {\isasymsubseteq}\ {\isasymUnion}F\ {\isasymand}\ card\ S\ {\isasymle}\ k{\isacharbraceright}{\kern0pt}\ {\isacharequal}{\kern0pt}\ {\isacharparenleft}{\kern0pt}{\isasymSum}i{\isasymle}k{\isachardot}{\kern0pt}\ card\ {\isacharparenleft}{\kern0pt}{\isasymUnion}\ F{\isacharparenright}{\kern0pt}\ choose\ i{\isacharparenright}{\kern0pt}{\isachardoublequoteclose}\isanewline
%
\isadelimproof
%
\endisadelimproof
%
\isatagproof
\isacommand{proof}\isamarkupfalse%
\ {\isacharparenleft}{\kern0pt}induction\ k{\isacharparenright}{\kern0pt}\isanewline
\ \ \isacommand{case}\isamarkupfalse%
\ {\isadigit{0}}\isanewline
\ \ \isacommand{from}\isamarkupfalse%
\ rev{\isacharunderscore}{\kern0pt}finite{\isacharunderscore}{\kern0pt}subset{\isacharbrackleft}{\kern0pt}OF\ assms{\isacharbrackright}{\kern0pt}\ \isacommand{have}\isamarkupfalse%
\ {\isachardoublequoteopen}S\ {\isasymsubseteq}\ {\isasymUnion}F\ {\isasymand}\ card\ S\ {\isasymle}\ {\isadigit{0}}\ {\isasymlongleftrightarrow}\ S\ {\isacharequal}{\kern0pt}\ {\isacharbraceleft}{\kern0pt}{\isacharbraceright}{\kern0pt}{\isachardoublequoteclose}\ \isakeyword{for}\ S\ \isacommand{by}\isamarkupfalse%
\ fastforce\isanewline
\ \ \isacommand{then}\isamarkupfalse%
\ \isacommand{show}\isamarkupfalse%
\ {\isacharquery}{\kern0pt}case\ \isacommand{by}\isamarkupfalse%
\ simp\isanewline
\isacommand{next}\isamarkupfalse%
\isanewline
\ \ \isacommand{case}\isamarkupfalse%
\ {\isacharparenleft}{\kern0pt}Suc\ k{\isacharparenright}{\kern0pt}\isanewline
\ \ \isacommand{let}\isamarkupfalse%
\ {\isacharquery}{\kern0pt}FS\ {\isacharequal}{\kern0pt}\ {\isachardoublequoteopen}{\isacharbraceleft}{\kern0pt}S{\isachardot}{\kern0pt}\ S\ {\isasymsubseteq}\ {\isasymUnion}\ F\ {\isasymand}\ card\ S\ {\isasymle}\ Suc\ k{\isacharbraceright}{\kern0pt}{\isachardoublequoteclose}\isanewline
\ \ \isakeyword{and}\ {\isacharquery}{\kern0pt}F{\isacharunderscore}{\kern0pt}Asm\ {\isacharequal}{\kern0pt}\ {\isachardoublequoteopen}{\isacharbraceleft}{\kern0pt}S{\isachardot}{\kern0pt}\ S\ {\isasymsubseteq}\ {\isasymUnion}\ F\ {\isasymand}\ card\ S\ {\isasymle}\ k{\isacharbraceright}{\kern0pt}{\isachardoublequoteclose}\ \isanewline
\ \ \isakeyword{and}\ {\isacharquery}{\kern0pt}F{\isacharunderscore}{\kern0pt}Step\ {\isacharequal}{\kern0pt}\ {\isachardoublequoteopen}{\isacharbraceleft}{\kern0pt}S{\isachardot}{\kern0pt}\ S\ {\isasymsubseteq}\ {\isasymUnion}\ F\ {\isasymand}\ card\ S\ {\isacharequal}{\kern0pt}\ Suc\ k{\isacharbraceright}{\kern0pt}{\isachardoublequoteclose}\isanewline
\isanewline
\ \ \isacommand{from}\isamarkupfalse%
\ finite{\isacharunderscore}{\kern0pt}Pow{\isacharunderscore}{\kern0pt}iff{\isacharbrackleft}{\kern0pt}of\ {\isachardoublequoteopen}{\isasymUnion}F{\isachardoublequoteclose}{\isacharbrackright}{\kern0pt}\ assms\ \isacommand{have}\isamarkupfalse%
\ finite{\isacharunderscore}{\kern0pt}Pow{\isacharunderscore}{\kern0pt}Un{\isacharunderscore}{\kern0pt}F{\isacharcolon}{\kern0pt}\ {\isachardoublequoteopen}finite\ {\isacharparenleft}{\kern0pt}Pow\ {\isacharparenleft}{\kern0pt}{\isasymUnion}\ F{\isacharparenright}{\kern0pt}{\isacharparenright}{\kern0pt}{\isachardoublequoteclose}\ \isacommand{{\isachardot}{\kern0pt}{\isachardot}{\kern0pt}}\isamarkupfalse%
\isanewline
\ \ \isacommand{have}\isamarkupfalse%
\ {\isachardoublequoteopen}{\isacharquery}{\kern0pt}F{\isacharunderscore}{\kern0pt}Asm\ {\isasymsubseteq}\ Pow\ {\isacharparenleft}{\kern0pt}{\isasymUnion}\ F{\isacharparenright}{\kern0pt}{\isachardoublequoteclose}\ \isakeyword{and}\ {\isachardoublequoteopen}{\isacharquery}{\kern0pt}F{\isacharunderscore}{\kern0pt}Step\ {\isasymsubseteq}\ Pow\ {\isacharparenleft}{\kern0pt}{\isasymUnion}\ F{\isacharparenright}{\kern0pt}{\isachardoublequoteclose}\ \isacommand{by}\isamarkupfalse%
\ fast{\isacharplus}{\kern0pt}\isanewline
\ \ \isacommand{with}\isamarkupfalse%
\ rev{\isacharunderscore}{\kern0pt}finite{\isacharunderscore}{\kern0pt}subset{\isacharbrackleft}{\kern0pt}OF\ finite{\isacharunderscore}{\kern0pt}Pow{\isacharunderscore}{\kern0pt}Un{\isacharunderscore}{\kern0pt}F{\isacharbrackright}{\kern0pt}\ \isacommand{have}\isamarkupfalse%
\ finite{\isacharunderscore}{\kern0pt}F{\isacharunderscore}{\kern0pt}Asm{\isacharcolon}{\kern0pt}\ {\isachardoublequoteopen}finite\ {\isacharquery}{\kern0pt}F{\isacharunderscore}{\kern0pt}Asm{\isachardoublequoteclose}\ \isakeyword{and}\ finite{\isacharunderscore}{\kern0pt}F{\isacharunderscore}{\kern0pt}Step{\isacharcolon}{\kern0pt}\ {\isachardoublequoteopen}finite\ {\isacharquery}{\kern0pt}F{\isacharunderscore}{\kern0pt}Step{\isachardoublequoteclose}\ \isacommand{by}\isamarkupfalse%
\ presburger{\isacharplus}{\kern0pt}\isanewline
\isanewline
\ \ \isacommand{have}\isamarkupfalse%
\ F{\isacharunderscore}{\kern0pt}Un{\isacharcolon}{\kern0pt}\ {\isachardoublequoteopen}{\isacharquery}{\kern0pt}FS\ {\isacharequal}{\kern0pt}\ {\isacharquery}{\kern0pt}F{\isacharunderscore}{\kern0pt}Asm\ {\isasymunion}\ {\isacharquery}{\kern0pt}F{\isacharunderscore}{\kern0pt}Step{\isachardoublequoteclose}\ \ \isakeyword{and}\ F{\isacharunderscore}{\kern0pt}disjoint{\isacharcolon}{\kern0pt}\ {\isachardoublequoteopen}{\isacharquery}{\kern0pt}F{\isacharunderscore}{\kern0pt}Asm\ {\isasyminter}\ {\isacharquery}{\kern0pt}F{\isacharunderscore}{\kern0pt}Step\ {\isacharequal}{\kern0pt}\ {\isacharbraceleft}{\kern0pt}{\isacharbraceright}{\kern0pt}{\isachardoublequoteclose}\ \isacommand{by}\isamarkupfalse%
\ fastforce{\isacharplus}{\kern0pt}\isanewline
\ \ \isacommand{from}\isamarkupfalse%
\ card{\isacharunderscore}{\kern0pt}Un{\isacharunderscore}{\kern0pt}disjoint{\isacharbrackleft}{\kern0pt}OF\ finite{\isacharunderscore}{\kern0pt}F{\isacharunderscore}{\kern0pt}Asm\ finite{\isacharunderscore}{\kern0pt}F{\isacharunderscore}{\kern0pt}Step\ F{\isacharunderscore}{\kern0pt}disjoint{\isacharbrackright}{\kern0pt}\ F{\isacharunderscore}{\kern0pt}Un\ \isacommand{have}\isamarkupfalse%
\ {\isachardoublequoteopen}card\ {\isacharquery}{\kern0pt}FS\ {\isacharequal}{\kern0pt}\ card\ {\isacharquery}{\kern0pt}F{\isacharunderscore}{\kern0pt}Asm\ {\isacharplus}{\kern0pt}\ card\ {\isacharquery}{\kern0pt}F{\isacharunderscore}{\kern0pt}Step{\isachardoublequoteclose}\ \isacommand{by}\isamarkupfalse%
\ argo\isanewline
\ \ \isacommand{also}\isamarkupfalse%
\ \isacommand{from}\isamarkupfalse%
\ Suc\ \isacommand{have}\isamarkupfalse%
\ {\isachardoublequoteopen}{\isachardot}{\kern0pt}{\isachardot}{\kern0pt}{\isachardot}{\kern0pt}\ {\isacharequal}{\kern0pt}\ {\isacharparenleft}{\kern0pt}{\isasymSum}i{\isasymle}k{\isachardot}{\kern0pt}\ card\ {\isacharparenleft}{\kern0pt}{\isasymUnion}\ F{\isacharparenright}{\kern0pt}\ choose\ i{\isacharparenright}{\kern0pt}\ {\isacharplus}{\kern0pt}\ card\ {\isacharquery}{\kern0pt}F{\isacharunderscore}{\kern0pt}Step{\isachardoublequoteclose}\ \isacommand{by}\isamarkupfalse%
\ argo\isanewline
\ \ \isacommand{also}\isamarkupfalse%
\ \isacommand{from}\isamarkupfalse%
\ n{\isacharunderscore}{\kern0pt}subsets{\isacharbrackleft}{\kern0pt}OF\ assms{\isacharcomma}{\kern0pt}\ of\ {\isachardoublequoteopen}Suc\ k{\isachardoublequoteclose}{\isacharbrackright}{\kern0pt}\ \isacommand{have}\isamarkupfalse%
\ {\isachardoublequoteopen}{\isachardot}{\kern0pt}{\isachardot}{\kern0pt}{\isachardot}{\kern0pt}\ {\isacharequal}{\kern0pt}\ {\isacharparenleft}{\kern0pt}{\isasymSum}i{\isasymle}Suc\ k{\isachardot}{\kern0pt}\ card\ {\isacharparenleft}{\kern0pt}{\isasymUnion}\ F{\isacharparenright}{\kern0pt}\ choose\ i{\isacharparenright}{\kern0pt}{\isachardoublequoteclose}\ \isacommand{by}\isamarkupfalse%
\ force\isanewline
\ \ \isacommand{finally}\isamarkupfalse%
\ \isacommand{show}\isamarkupfalse%
\ {\isacharquery}{\kern0pt}case\ \isacommand{by}\isamarkupfalse%
\ blast\isanewline
\isacommand{qed}\isamarkupfalse%
%
\endisatagproof
{\isafoldproof}%
%
\isadelimproof
\isanewline
%
\endisadelimproof
%
\isadelimtheory
\isanewline
%
\endisadelimtheory
%
\isatagtheory
\isacommand{end}\isamarkupfalse%
%
\endisatagtheory
{\isafoldtheory}%
%
\isadelimtheory
%
\endisadelimtheory
%
\end{isabellebody}%
\endinput
%:%file=Binomial_Lemmas.tex%:%
%:%11=5%:%
%:%23=7%:%
%:%31=9%:%
%:%32=9%:%
%:%33=10%:%
%:%34=11%:%
%:%41=11%:%
%:%42=12%:%
%:%43=13%:%
%:%44=13%:%
%:%45=14%:%
%:%46=15%:%
%:%53=16%:%
%:%54=16%:%
%:%55=17%:%
%:%56=17%:%
%:%57=17%:%
%:%58=18%:%
%:%59=18%:%
%:%60=18%:%
%:%61=18%:%
%:%62=19%:%
%:%63=19%:%
%:%64=19%:%
%:%65=19%:%
%:%66=20%:%
%:%67=20%:%
%:%68=20%:%
%:%69=20%:%
%:%70=21%:%
%:%71=21%:%
%:%72=21%:%
%:%73=21%:%
%:%74=21%:%
%:%75=22%:%
%:%81=22%:%
%:%84=23%:%
%:%85=24%:%
%:%86=24%:%
%:%87=25%:%
%:%88=26%:%
%:%95=27%:%
%:%96=27%:%
%:%97=28%:%
%:%98=28%:%
%:%99=29%:%
%:%100=29%:%
%:%101=29%:%
%:%102=29%:%
%:%103=30%:%
%:%104=30%:%
%:%105=30%:%
%:%106=30%:%
%:%107=31%:%
%:%108=31%:%
%:%109=32%:%
%:%110=32%:%
%:%111=33%:%
%:%112=33%:%
%:%113=34%:%
%:%114=35%:%
%:%115=36%:%
%:%116=37%:%
%:%117=37%:%
%:%118=37%:%
%:%119=37%:%
%:%120=38%:%
%:%121=38%:%
%:%122=38%:%
%:%123=39%:%
%:%124=39%:%
%:%125=39%:%
%:%126=39%:%
%:%127=40%:%
%:%128=41%:%
%:%129=41%:%
%:%130=41%:%
%:%131=42%:%
%:%132=42%:%
%:%133=42%:%
%:%134=42%:%
%:%135=43%:%
%:%136=43%:%
%:%137=43%:%
%:%138=43%:%
%:%139=43%:%
%:%140=44%:%
%:%141=44%:%
%:%142=44%:%
%:%143=44%:%
%:%144=44%:%
%:%145=45%:%
%:%146=45%:%
%:%147=45%:%
%:%148=45%:%
%:%149=46%:%
%:%155=46%:%
%:%160=47%:%
%:%165=48%:%

%
\begin{isabellebody}%
\setisabellecontext{Sauer{\isacharunderscore}{\kern0pt}Shelah{\isacharunderscore}{\kern0pt}Lemma}%
%
\isadelimdocument
%
\endisadelimdocument
%
\isatagdocument
%
\isamarkupsection{Sauer-Shelah Lemma%
}
\isamarkuptrue%
%
\endisatagdocument
{\isafolddocument}%
%
\isadelimdocument
%
\endisadelimdocument
%
\isadelimtheory
%
\endisadelimtheory
%
\isatagtheory
\isacommand{theory}\isamarkupfalse%
\ Sauer{\isacharunderscore}{\kern0pt}Shelah{\isacharunderscore}{\kern0pt}Lemma\isanewline
\ \ \isakeyword{imports}\ Main\ Shattering\ Card{\isacharunderscore}{\kern0pt}Lemmas\ Binomial{\isacharunderscore}{\kern0pt}Lemmas\isanewline
\isakeyword{begin}%
\endisatagtheory
{\isafoldtheory}%
%
\isadelimtheory
%
\endisadelimtheory
%
\isadelimdocument
%
\endisadelimdocument
%
\isatagdocument
%
\isamarkupsubsection{Generalized Sauer-Shelah Lemma%
}
\isamarkuptrue%
%
\endisatagdocument
{\isafolddocument}%
%
\isadelimdocument
%
\endisadelimdocument
%
\begin{isamarkuptext}%
To prove the Sauer-Shelah Lemma, we will first prove a slightly stronger fact that every family
      \isa{F} shatters at least as many sets as \isa{card\ F}. We first fix an element \isa{x\ {\isasymin}\ {\isasymUnion}\ F}
      and consider the subfamily \isa{F{\isadigit{0}}} of sets in the family not containing it. By induction \isa{F{\isadigit{0}}} 
      shatters at least as many elements of \isa{F} as \isa{card\ F{\isadigit{0}}}. 
      Next we consider the subfamily \isa{F{\isadigit{1}}} of sets in the family that contain \isa{x}.
      Again, by induction \isa{F{\isadigit{1}}} shatters as many elements of \isa{F} as its cardinality. 
      The number of elements of \isa{F} shattered by \isa{F{\isadigit{0}}} and \isa{F{\isadigit{1}}} sum up to at least 
      \isa{card\ F{\isadigit{0}}\ {\isacharplus}{\kern0pt}\ card\ F{\isadigit{1}}\ {\isacharequal}{\kern0pt}\ card\ F}. When a set \isa{S\ {\isasymin}\ F} is shattered by only one of the two subfamilies, say \isa{F{\isadigit{0}}}, 
      it contributes one unit to the set \isa{shattered{\isacharunderscore}{\kern0pt}by\ F{\isadigit{0}}} and to \isa{shattered{\isacharunderscore}{\kern0pt}by\ F}. However when the set is shattered by 
      both subfamilies, both \isa{S} and \isa{S\ {\isasymunion}\ {\isacharbraceleft}{\kern0pt}x{\isacharbraceright}{\kern0pt}} are in \isa{shattered{\isacharunderscore}{\kern0pt}by\ F}, so \isa{S} contributes two units
      to \isa{shattered{\isacharunderscore}{\kern0pt}by\ F{\isadigit{0}}\ {\isasymunion}\ shattered{\isacharunderscore}{\kern0pt}by\ F{\isadigit{1}}}. Therefore, the cardinality of \isa{shattered{\isacharunderscore}{\kern0pt}by\ F} 
      is at least equal to the cardinality of \isa{shattered{\isacharunderscore}{\kern0pt}by\ F{\isadigit{0}}\ {\isasymunion}\ shattered{\isacharunderscore}{\kern0pt}by\ F{\isadigit{1}}}, which is at least \isa{card\ F}.%
\end{isamarkuptext}\isamarkuptrue%
\isacommand{lemma}\isamarkupfalse%
\ sauer{\isacharunderscore}{\kern0pt}shelah{\isacharunderscore}{\kern0pt}{\isadigit{0}}{\isacharcolon}{\kern0pt}\isanewline
\ \ \isakeyword{fixes}\ F\ {\isacharcolon}{\kern0pt}{\isacharcolon}{\kern0pt}\ {\isachardoublequoteopen}{\isacharprime}{\kern0pt}a\ set\ set{\isachardoublequoteclose}\isanewline
\ \ \isakeyword{shows}\ {\isachardoublequoteopen}finite\ {\isacharparenleft}{\kern0pt}{\isasymUnion}\ F{\isacharparenright}{\kern0pt}\ {\isasymLongrightarrow}\ card\ F\ {\isasymle}\ card\ {\isacharparenleft}{\kern0pt}shattered{\isacharunderscore}{\kern0pt}by\ F{\isacharparenright}{\kern0pt}{\isachardoublequoteclose}\isanewline
%
\isadelimproof
%
\endisadelimproof
%
\isatagproof
\isacommand{proof}\isamarkupfalse%
\ {\isacharparenleft}{\kern0pt}induction\ F\ rule{\isacharcolon}{\kern0pt}\ measure{\isacharunderscore}{\kern0pt}induct{\isacharunderscore}{\kern0pt}rule{\isacharbrackleft}{\kern0pt}of\ {\isachardoublequoteopen}card{\isachardoublequoteclose}{\isacharbrackright}{\kern0pt}{\isacharparenright}{\kern0pt}\isanewline
\ \ \isacommand{case}\isamarkupfalse%
\ {\isacharparenleft}{\kern0pt}less\ F{\isacharparenright}{\kern0pt}\isanewline
\ \ \isacommand{note}\isamarkupfalse%
\ finite{\isacharunderscore}{\kern0pt}F\ {\isacharequal}{\kern0pt}\ finite{\isacharunderscore}{\kern0pt}UnionD{\isacharbrackleft}{\kern0pt}OF\ less{\isacharparenleft}{\kern0pt}{\isadigit{2}}{\isacharparenright}{\kern0pt}{\isacharbrackright}{\kern0pt}\isanewline
\ \ \isacommand{note}\isamarkupfalse%
\ finite{\isacharunderscore}{\kern0pt}shF\ {\isacharequal}{\kern0pt}\ finite{\isacharunderscore}{\kern0pt}shattered{\isacharunderscore}{\kern0pt}by{\isacharbrackleft}{\kern0pt}OF\ less{\isacharparenleft}{\kern0pt}{\isadigit{2}}{\isacharparenright}{\kern0pt}{\isacharbrackright}{\kern0pt}\isanewline
\ \ \isacommand{show}\isamarkupfalse%
\ {\isacharquery}{\kern0pt}case\isanewline
\ \ \isacommand{proof}\isamarkupfalse%
\ {\isacharparenleft}{\kern0pt}cases\ {\isachardoublequoteopen}{\isadigit{2}}\ {\isasymle}\ card\ F{\isachardoublequoteclose}{\isacharparenright}{\kern0pt}\isanewline
\ \ \ \ \isacommand{case}\isamarkupfalse%
\ True\isanewline
\ \ \ \ \isacommand{from}\isamarkupfalse%
\ obtain{\isacharunderscore}{\kern0pt}difference{\isacharunderscore}{\kern0pt}element{\isacharbrackleft}{\kern0pt}OF\ True{\isacharbrackright}{\kern0pt}\ \isanewline
\ \ \ \ \isacommand{obtain}\isamarkupfalse%
\ x\ {\isacharcolon}{\kern0pt}{\isacharcolon}{\kern0pt}\ {\isacharprime}{\kern0pt}a\ \isakeyword{where}\ x{\isacharunderscore}{\kern0pt}in{\isacharunderscore}{\kern0pt}Union{\isacharunderscore}{\kern0pt}F{\isacharcolon}{\kern0pt}\ {\isachardoublequoteopen}x\ {\isasymin}\ {\isasymUnion}F{\isachardoublequoteclose}\ \isanewline
\ \ \ \ \ \ \ \ \ \ \ \ \ \ \ \ \ \ \ \ \ \isakeyword{and}\ x{\isacharunderscore}{\kern0pt}not{\isacharunderscore}{\kern0pt}in{\isacharunderscore}{\kern0pt}Int{\isacharunderscore}{\kern0pt}F{\isacharcolon}{\kern0pt}\ {\isachardoublequoteopen}x\ {\isasymnotin}\ {\isasymInter}F{\isachardoublequoteclose}\ \isacommand{by}\isamarkupfalse%
\ blast%
\begin{isamarkuptext}%
Define F0 as the subfamily of F containing sets that don't contain \isa{x}.%
\end{isamarkuptext}\isamarkuptrue%
\ \ \ \ \isacommand{let}\isamarkupfalse%
\ {\isacharquery}{\kern0pt}F{\isadigit{0}}\ {\isacharequal}{\kern0pt}\ {\isachardoublequoteopen}{\isacharbraceleft}{\kern0pt}S\ {\isasymin}\ F{\isachardot}{\kern0pt}\ x\ {\isasymnotin}\ S{\isacharbraceright}{\kern0pt}{\isachardoublequoteclose}\isanewline
\ \ \ \ \isacommand{from}\isamarkupfalse%
\ x{\isacharunderscore}{\kern0pt}in{\isacharunderscore}{\kern0pt}Union{\isacharunderscore}{\kern0pt}F\ \isacommand{have}\isamarkupfalse%
\ F{\isadigit{0}}{\isacharunderscore}{\kern0pt}psubset{\isacharunderscore}{\kern0pt}F{\isacharcolon}{\kern0pt}\ {\isachardoublequoteopen}{\isacharquery}{\kern0pt}F{\isadigit{0}}\ {\isasymsubset}\ F{\isachardoublequoteclose}\ \isacommand{by}\isamarkupfalse%
\ blast\isanewline
\ \ \ \ \isacommand{from}\isamarkupfalse%
\ F{\isadigit{0}}{\isacharunderscore}{\kern0pt}psubset{\isacharunderscore}{\kern0pt}F\ \isacommand{have}\isamarkupfalse%
\ F{\isadigit{0}}{\isacharunderscore}{\kern0pt}in{\isacharunderscore}{\kern0pt}F{\isacharcolon}{\kern0pt}\ {\isachardoublequoteopen}{\isacharquery}{\kern0pt}F{\isadigit{0}}\ {\isasymsubseteq}\ F{\isachardoublequoteclose}\ \isacommand{by}\isamarkupfalse%
\ blast\isanewline
\ \ \ \ \isacommand{from}\isamarkupfalse%
\ subset{\isacharunderscore}{\kern0pt}shattered{\isacharunderscore}{\kern0pt}by{\isacharbrackleft}{\kern0pt}OF\ F{\isadigit{0}}{\isacharunderscore}{\kern0pt}in{\isacharunderscore}{\kern0pt}F{\isacharbrackright}{\kern0pt}\ \isacommand{have}\isamarkupfalse%
\ shF{\isadigit{0}}{\isacharunderscore}{\kern0pt}subset{\isacharunderscore}{\kern0pt}shF{\isacharcolon}{\kern0pt}\ {\isachardoublequoteopen}shattered{\isacharunderscore}{\kern0pt}by\ {\isacharquery}{\kern0pt}F{\isadigit{0}}\ {\isasymsubseteq}\ shattered{\isacharunderscore}{\kern0pt}by\ F{\isachardoublequoteclose}\ \isacommand{{\isachardot}{\kern0pt}}\isamarkupfalse%
\isanewline
\ \ \ \ \isacommand{from}\isamarkupfalse%
\ F{\isadigit{0}}{\isacharunderscore}{\kern0pt}in{\isacharunderscore}{\kern0pt}F\ \isacommand{have}\isamarkupfalse%
\ Un{\isacharunderscore}{\kern0pt}F{\isadigit{0}}{\isacharunderscore}{\kern0pt}in{\isacharunderscore}{\kern0pt}Un{\isacharunderscore}{\kern0pt}F{\isacharcolon}{\kern0pt}{\isachardoublequoteopen}{\isasymUnion}\ {\isacharquery}{\kern0pt}F{\isadigit{0}}\ {\isasymsubseteq}\ {\isasymUnion}\ F{\isachardoublequoteclose}\ \isacommand{by}\isamarkupfalse%
\ blast%
\begin{isamarkuptext}%
F0 shatters at least as many sets as \isa{card\ F{\isadigit{0}}} by the induction hypothesis%
\end{isamarkuptext}\isamarkuptrue%
\ \ \ \ \isacommand{note}\isamarkupfalse%
\ IH{\isacharunderscore}{\kern0pt}F{\isadigit{0}}\ {\isacharequal}{\kern0pt}\ less{\isacharparenleft}{\kern0pt}{\isadigit{1}}{\isacharparenright}{\kern0pt}{\isacharbrackleft}{\kern0pt}OF\ psubset{\isacharunderscore}{\kern0pt}card{\isacharunderscore}{\kern0pt}mono{\isacharbrackleft}{\kern0pt}OF\ finite{\isacharunderscore}{\kern0pt}F\ F{\isadigit{0}}{\isacharunderscore}{\kern0pt}psubset{\isacharunderscore}{\kern0pt}F{\isacharbrackright}{\kern0pt}\ rev{\isacharunderscore}{\kern0pt}finite{\isacharunderscore}{\kern0pt}subset{\isacharbrackleft}{\kern0pt}OF\ less{\isacharparenleft}{\kern0pt}{\isadigit{2}}{\isacharparenright}{\kern0pt}\ Un{\isacharunderscore}{\kern0pt}F{\isadigit{0}}{\isacharunderscore}{\kern0pt}in{\isacharunderscore}{\kern0pt}Un{\isacharunderscore}{\kern0pt}F{\isacharbrackright}{\kern0pt}{\isacharbrackright}{\kern0pt}%
\begin{isamarkuptext}%
Define F1 as the subfamily of F containing sets that contain \isa{x}%
\end{isamarkuptext}\isamarkuptrue%
\ \ \ \ \isacommand{let}\isamarkupfalse%
\ {\isacharquery}{\kern0pt}F{\isadigit{1}}\ {\isacharequal}{\kern0pt}\ {\isachardoublequoteopen}{\isacharbraceleft}{\kern0pt}S\ {\isasymin}\ F{\isachardot}{\kern0pt}\ x\ {\isasymin}\ S{\isacharbraceright}{\kern0pt}{\isachardoublequoteclose}\isanewline
\ \ \ \ \isacommand{from}\isamarkupfalse%
\ x{\isacharunderscore}{\kern0pt}not{\isacharunderscore}{\kern0pt}in{\isacharunderscore}{\kern0pt}Int{\isacharunderscore}{\kern0pt}F\ \isacommand{have}\isamarkupfalse%
\ F{\isadigit{1}}{\isacharunderscore}{\kern0pt}psubset{\isacharunderscore}{\kern0pt}F{\isacharcolon}{\kern0pt}\ {\isachardoublequoteopen}{\isacharquery}{\kern0pt}F{\isadigit{1}}\ {\isasymsubset}\ F{\isachardoublequoteclose}\ \isacommand{by}\isamarkupfalse%
\ blast\isanewline
\ \ \ \ \isacommand{from}\isamarkupfalse%
\ F{\isadigit{1}}{\isacharunderscore}{\kern0pt}psubset{\isacharunderscore}{\kern0pt}F\ \isacommand{have}\isamarkupfalse%
\ F{\isadigit{1}}{\isacharunderscore}{\kern0pt}in{\isacharunderscore}{\kern0pt}F{\isacharcolon}{\kern0pt}\ {\isachardoublequoteopen}{\isacharquery}{\kern0pt}F{\isadigit{1}}\ {\isasymsubseteq}\ F{\isachardoublequoteclose}\ \isacommand{by}\isamarkupfalse%
\ blast\isanewline
\ \ \ \ \isacommand{from}\isamarkupfalse%
\ subset{\isacharunderscore}{\kern0pt}shattered{\isacharunderscore}{\kern0pt}by{\isacharbrackleft}{\kern0pt}OF\ F{\isadigit{1}}{\isacharunderscore}{\kern0pt}in{\isacharunderscore}{\kern0pt}F{\isacharbrackright}{\kern0pt}\ \isacommand{have}\isamarkupfalse%
\ shF{\isadigit{1}}{\isacharunderscore}{\kern0pt}subset{\isacharunderscore}{\kern0pt}shF{\isacharcolon}{\kern0pt}\ {\isachardoublequoteopen}shattered{\isacharunderscore}{\kern0pt}by\ {\isacharquery}{\kern0pt}F{\isadigit{1}}\ {\isasymsubseteq}\ shattered{\isacharunderscore}{\kern0pt}by\ F{\isachardoublequoteclose}\ \isacommand{{\isachardot}{\kern0pt}}\isamarkupfalse%
\isanewline
\ \ \ \ \isacommand{from}\isamarkupfalse%
\ F{\isadigit{1}}{\isacharunderscore}{\kern0pt}in{\isacharunderscore}{\kern0pt}F\ \isacommand{have}\isamarkupfalse%
\ Un{\isacharunderscore}{\kern0pt}F{\isadigit{1}}{\isacharunderscore}{\kern0pt}in{\isacharunderscore}{\kern0pt}Un{\isacharunderscore}{\kern0pt}F{\isacharcolon}{\kern0pt}{\isachardoublequoteopen}{\isasymUnion}\ {\isacharquery}{\kern0pt}F{\isadigit{1}}\ {\isasymsubseteq}\ {\isasymUnion}\ F{\isachardoublequoteclose}\ \isacommand{by}\isamarkupfalse%
\ blast%
\begin{isamarkuptext}%
F1 shatters at least as many sets as \isa{card\ F{\isadigit{1}}} by the induction hypothesis%
\end{isamarkuptext}\isamarkuptrue%
\ \ \ \ \isacommand{note}\isamarkupfalse%
\ IH{\isacharunderscore}{\kern0pt}F{\isadigit{1}}\ {\isacharequal}{\kern0pt}\ less{\isacharparenleft}{\kern0pt}{\isadigit{1}}{\isacharparenright}{\kern0pt}{\isacharbrackleft}{\kern0pt}OF\ psubset{\isacharunderscore}{\kern0pt}card{\isacharunderscore}{\kern0pt}mono{\isacharbrackleft}{\kern0pt}OF\ finite{\isacharunderscore}{\kern0pt}F\ F{\isadigit{1}}{\isacharunderscore}{\kern0pt}psubset{\isacharunderscore}{\kern0pt}F{\isacharbrackright}{\kern0pt}\ rev{\isacharunderscore}{\kern0pt}finite{\isacharunderscore}{\kern0pt}subset{\isacharbrackleft}{\kern0pt}OF\ less{\isacharparenleft}{\kern0pt}{\isadigit{2}}{\isacharparenright}{\kern0pt}\ Un{\isacharunderscore}{\kern0pt}F{\isadigit{1}}{\isacharunderscore}{\kern0pt}in{\isacharunderscore}{\kern0pt}Un{\isacharunderscore}{\kern0pt}F{\isacharbrackright}{\kern0pt}{\isacharbrackright}{\kern0pt}\isanewline
\isanewline
\ \ \ \ \isacommand{from}\isamarkupfalse%
\ shF{\isadigit{0}}{\isacharunderscore}{\kern0pt}subset{\isacharunderscore}{\kern0pt}shF\ shF{\isadigit{1}}{\isacharunderscore}{\kern0pt}subset{\isacharunderscore}{\kern0pt}shF\ \isanewline
\ \ \ \ \isacommand{have}\isamarkupfalse%
\ shattered{\isacharunderscore}{\kern0pt}subset{\isacharcolon}{\kern0pt}\ {\isachardoublequoteopen}{\isacharparenleft}{\kern0pt}shattered{\isacharunderscore}{\kern0pt}by\ {\isacharquery}{\kern0pt}F{\isadigit{0}}{\isacharparenright}{\kern0pt}\ {\isasymunion}\ {\isacharparenleft}{\kern0pt}shattered{\isacharunderscore}{\kern0pt}by\ {\isacharquery}{\kern0pt}F{\isadigit{1}}{\isacharparenright}{\kern0pt}\ {\isasymsubseteq}\ shattered{\isacharunderscore}{\kern0pt}by\ F{\isachardoublequoteclose}\ \isacommand{by}\isamarkupfalse%
\ simp%
\begin{isamarkuptext}%
There is a set with the same cardinality as the intersection of 
        \isa{shattered{\isacharunderscore}{\kern0pt}by\ F{\isadigit{0}}} and \isa{shattered{\isacharunderscore}{\kern0pt}by\ F{\isadigit{1}}} which is disjoint from their union, 
        which is also contained in \isa{shattered{\isacharunderscore}{\kern0pt}by\ F}.%
\end{isamarkuptext}\isamarkuptrue%
\ \ \ \ \isacommand{have}\isamarkupfalse%
\ f{\isacharunderscore}{\kern0pt}copies{\isacharunderscore}{\kern0pt}the{\isacharunderscore}{\kern0pt}intersection{\isacharcolon}{\kern0pt}\isanewline
\ \ \ \ \ \ {\isachardoublequoteopen}{\isasymexists}f{\isachardot}{\kern0pt}\ inj{\isacharunderscore}{\kern0pt}on\ f\ {\isacharparenleft}{\kern0pt}shattered{\isacharunderscore}{\kern0pt}by\ {\isacharquery}{\kern0pt}F{\isadigit{0}}\ {\isasyminter}\ shattered{\isacharunderscore}{\kern0pt}by\ {\isacharquery}{\kern0pt}F{\isadigit{1}}{\isacharparenright}{\kern0pt}\ {\isasymand}\isanewline
\ \ \ \ \ \ \ {\isacharparenleft}{\kern0pt}shattered{\isacharunderscore}{\kern0pt}by\ {\isacharquery}{\kern0pt}F{\isadigit{0}}\ {\isasymunion}\ shattered{\isacharunderscore}{\kern0pt}by\ {\isacharquery}{\kern0pt}F{\isadigit{1}}{\isacharparenright}{\kern0pt}\ {\isasyminter}\ {\isacharparenleft}{\kern0pt}f\ {\isacharbackquote}{\kern0pt}\ {\isacharparenleft}{\kern0pt}shattered{\isacharunderscore}{\kern0pt}by\ {\isacharquery}{\kern0pt}F{\isadigit{0}}\ {\isasyminter}\ shattered{\isacharunderscore}{\kern0pt}by\ {\isacharquery}{\kern0pt}F{\isadigit{1}}{\isacharparenright}{\kern0pt}{\isacharparenright}{\kern0pt}\ {\isacharequal}{\kern0pt}\ {\isacharbraceleft}{\kern0pt}{\isacharbraceright}{\kern0pt}\ {\isasymand}\isanewline
\ \ \ \ \ \ \ f\ {\isacharbackquote}{\kern0pt}\ {\isacharparenleft}{\kern0pt}shattered{\isacharunderscore}{\kern0pt}by\ {\isacharquery}{\kern0pt}F{\isadigit{0}}\ {\isasyminter}\ shattered{\isacharunderscore}{\kern0pt}by\ {\isacharquery}{\kern0pt}F{\isadigit{1}}{\isacharparenright}{\kern0pt}\ {\isasymsubseteq}\ shattered{\isacharunderscore}{\kern0pt}by\ F{\isachardoublequoteclose}\isanewline
\ \ \ \ \isacommand{proof}\isamarkupfalse%
\isanewline
\ \ \ \ \ \ \isacommand{have}\isamarkupfalse%
\ x{\isacharunderscore}{\kern0pt}not{\isacharunderscore}{\kern0pt}in{\isacharunderscore}{\kern0pt}shattered{\isacharcolon}{\kern0pt}\ {\isachardoublequoteopen}{\isasymforall}S{\isasymin}{\isacharparenleft}{\kern0pt}shattered{\isacharunderscore}{\kern0pt}by\ {\isacharquery}{\kern0pt}F{\isadigit{0}}{\isacharparenright}{\kern0pt}\ {\isasymunion}\ {\isacharparenleft}{\kern0pt}shattered{\isacharunderscore}{\kern0pt}by\ {\isacharquery}{\kern0pt}F{\isadigit{1}}{\isacharparenright}{\kern0pt}{\isachardot}{\kern0pt}\ x\ {\isasymnotin}\ S{\isachardoublequoteclose}\ \isacommand{unfolding}\isamarkupfalse%
\ shattered{\isacharunderscore}{\kern0pt}by{\isacharunderscore}{\kern0pt}def\ \isacommand{by}\isamarkupfalse%
\ blast%
\begin{isamarkuptext}%
This set is precisely the image of the intersection under \isa{insert\ x}.%
\end{isamarkuptext}\isamarkuptrue%
\ \ \ \ \ \ \isacommand{let}\isamarkupfalse%
\ {\isacharquery}{\kern0pt}f\ {\isacharequal}{\kern0pt}\ {\isachardoublequoteopen}insert\ x{\isachardoublequoteclose}\isanewline
\ \ \ \ \ \ \isacommand{have}\isamarkupfalse%
\ {\isadigit{0}}{\isacharcolon}{\kern0pt}\ {\isachardoublequoteopen}inj{\isacharunderscore}{\kern0pt}on\ {\isacharquery}{\kern0pt}f\ {\isacharparenleft}{\kern0pt}shattered{\isacharunderscore}{\kern0pt}by\ {\isacharquery}{\kern0pt}F{\isadigit{0}}\ {\isasyminter}\ shattered{\isacharunderscore}{\kern0pt}by\ {\isacharquery}{\kern0pt}F{\isadigit{1}}{\isacharparenright}{\kern0pt}{\isachardoublequoteclose}\isanewline
\ \ \ \ \ \ \isacommand{proof}\isamarkupfalse%
\isanewline
\ \ \ \ \ \ \ \ \isacommand{fix}\isamarkupfalse%
\ X\ Y\isanewline
\ \ \ \ \ \ \ \ \isacommand{assume}\isamarkupfalse%
\ x{\isadigit{0}}{\isacharcolon}{\kern0pt}\ {\isachardoublequoteopen}X\ {\isasymin}\ {\isacharparenleft}{\kern0pt}shattered{\isacharunderscore}{\kern0pt}by\ {\isacharquery}{\kern0pt}F{\isadigit{0}}\ {\isasyminter}\ shattered{\isacharunderscore}{\kern0pt}by\ {\isacharquery}{\kern0pt}F{\isadigit{1}}{\isacharparenright}{\kern0pt}{\isachardoublequoteclose}\ \isakeyword{and}\ y{\isadigit{0}}{\isacharcolon}{\kern0pt}\ {\isachardoublequoteopen}Y\ {\isasymin}\ {\isacharparenleft}{\kern0pt}shattered{\isacharunderscore}{\kern0pt}by\ {\isacharquery}{\kern0pt}F{\isadigit{0}}\ {\isasyminter}\ shattered{\isacharunderscore}{\kern0pt}by\ {\isacharquery}{\kern0pt}F{\isadigit{1}}{\isacharparenright}{\kern0pt}{\isachardoublequoteclose}\isanewline
\ \ \ \ \ \ \ \ \ \ \ \ \ \ \ \isakeyword{and}\ {\isadigit{0}}{\isacharcolon}{\kern0pt}\ {\isachardoublequoteopen}{\isacharquery}{\kern0pt}f\ X\ {\isacharequal}{\kern0pt}\ {\isacharquery}{\kern0pt}f\ Y{\isachardoublequoteclose}\isanewline
\ \ \ \ \ \ \ \ \isacommand{from}\isamarkupfalse%
\ x{\isacharunderscore}{\kern0pt}not{\isacharunderscore}{\kern0pt}in{\isacharunderscore}{\kern0pt}shattered\ x{\isadigit{0}}\ \isacommand{have}\isamarkupfalse%
\ {\isachardoublequoteopen}X\ {\isacharequal}{\kern0pt}\ {\isacharquery}{\kern0pt}f\ X\ {\isacharminus}{\kern0pt}\ {\isacharbraceleft}{\kern0pt}x{\isacharbraceright}{\kern0pt}{\isachardoublequoteclose}\ \isacommand{by}\isamarkupfalse%
\ blast\isanewline
\ \ \ \ \ \ \ \ \isacommand{also}\isamarkupfalse%
\ \isacommand{from}\isamarkupfalse%
\ {\isadigit{0}}\ \isacommand{have}\isamarkupfalse%
\ {\isachardoublequoteopen}{\isachardot}{\kern0pt}{\isachardot}{\kern0pt}{\isachardot}{\kern0pt}\ {\isacharequal}{\kern0pt}\ {\isacharquery}{\kern0pt}f\ Y\ {\isacharminus}{\kern0pt}\ {\isacharbraceleft}{\kern0pt}x{\isacharbraceright}{\kern0pt}{\isachardoublequoteclose}\ \isacommand{by}\isamarkupfalse%
\ argo\isanewline
\ \ \ \ \ \ \ \ \isacommand{also}\isamarkupfalse%
\ \isacommand{from}\isamarkupfalse%
\ x{\isacharunderscore}{\kern0pt}not{\isacharunderscore}{\kern0pt}in{\isacharunderscore}{\kern0pt}shattered\ y{\isadigit{0}}\ \isacommand{have}\isamarkupfalse%
\ {\isachardoublequoteopen}{\isachardot}{\kern0pt}{\isachardot}{\kern0pt}{\isachardot}{\kern0pt}\ {\isacharequal}{\kern0pt}\ Y{\isachardoublequoteclose}\ \isacommand{by}\isamarkupfalse%
\ blast\isanewline
\ \ \ \ \ \ \ \ \isacommand{finally}\isamarkupfalse%
\ \isacommand{show}\isamarkupfalse%
\ {\isachardoublequoteopen}X\ {\isacharequal}{\kern0pt}\ Y{\isachardoublequoteclose}\ \isacommand{{\isachardot}{\kern0pt}}\isamarkupfalse%
\isanewline
\ \ \ \ \ \ \isacommand{qed}\isamarkupfalse%
%
\begin{isamarkuptext}%
The set is disjoint from the union.%
\end{isamarkuptext}\isamarkuptrue%
\ \ \ \ \ \ \isacommand{have}\isamarkupfalse%
\ {\isadigit{1}}{\isacharcolon}{\kern0pt}\ {\isachardoublequoteopen}{\isacharparenleft}{\kern0pt}shattered{\isacharunderscore}{\kern0pt}by\ {\isacharquery}{\kern0pt}F{\isadigit{0}}\ {\isasymunion}\ shattered{\isacharunderscore}{\kern0pt}by\ {\isacharquery}{\kern0pt}F{\isadigit{1}}{\isacharparenright}{\kern0pt}\ {\isasyminter}\ {\isacharquery}{\kern0pt}f\ {\isacharbackquote}{\kern0pt}\ {\isacharparenleft}{\kern0pt}shattered{\isacharunderscore}{\kern0pt}by\ {\isacharquery}{\kern0pt}F{\isadigit{0}}\ {\isasyminter}\ shattered{\isacharunderscore}{\kern0pt}by\ {\isacharquery}{\kern0pt}F{\isadigit{1}}{\isacharparenright}{\kern0pt}\ {\isacharequal}{\kern0pt}\ {\isacharbraceleft}{\kern0pt}{\isacharbraceright}{\kern0pt}{\isachardoublequoteclose}\isanewline
\ \ \ \ \ \ \isacommand{proof}\isamarkupfalse%
\ {\isacharparenleft}{\kern0pt}rule\ ccontr{\isacharparenright}{\kern0pt}\isanewline
\ \ \ \ \ \ \ \ \isacommand{assume}\isamarkupfalse%
\ {\isachardoublequoteopen}{\isacharparenleft}{\kern0pt}shattered{\isacharunderscore}{\kern0pt}by\ {\isacharquery}{\kern0pt}F{\isadigit{0}}\ {\isasymunion}\ shattered{\isacharunderscore}{\kern0pt}by\ {\isacharquery}{\kern0pt}F{\isadigit{1}}{\isacharparenright}{\kern0pt}\ {\isasyminter}\ {\isacharquery}{\kern0pt}f\ {\isacharbackquote}{\kern0pt}\ {\isacharparenleft}{\kern0pt}shattered{\isacharunderscore}{\kern0pt}by\ {\isacharquery}{\kern0pt}F{\isadigit{0}}\ {\isasyminter}\ shattered{\isacharunderscore}{\kern0pt}by\ {\isacharquery}{\kern0pt}F{\isadigit{1}}{\isacharparenright}{\kern0pt}\ {\isasymnoteq}\ {\isacharbraceleft}{\kern0pt}{\isacharbraceright}{\kern0pt}{\isachardoublequoteclose}\isanewline
\ \ \ \ \ \ \ \ \isacommand{then}\isamarkupfalse%
\ \isacommand{obtain}\isamarkupfalse%
\ S\ \isakeyword{where}\ {\isadigit{1}}{\isadigit{0}}{\isacharcolon}{\kern0pt}\ {\isachardoublequoteopen}S\ {\isasymin}\ {\isacharparenleft}{\kern0pt}shattered{\isacharunderscore}{\kern0pt}by\ {\isacharquery}{\kern0pt}F{\isadigit{0}}\ {\isasymunion}\ shattered{\isacharunderscore}{\kern0pt}by\ {\isacharquery}{\kern0pt}F{\isadigit{1}}{\isacharparenright}{\kern0pt}{\isachardoublequoteclose}\ \isanewline
\ \ \ \ \ \ \ \ \ \ \ \ \ \ \ \ \ \ \ \ \ \ \ \ \isakeyword{and}\ {\isadigit{1}}{\isadigit{1}}{\isacharcolon}{\kern0pt}\ {\isachardoublequoteopen}S\ {\isasymin}\ {\isacharquery}{\kern0pt}f\ {\isacharbackquote}{\kern0pt}\ {\isacharparenleft}{\kern0pt}shattered{\isacharunderscore}{\kern0pt}by\ {\isacharquery}{\kern0pt}F{\isadigit{0}}\ {\isasyminter}\ shattered{\isacharunderscore}{\kern0pt}by\ {\isacharquery}{\kern0pt}F{\isadigit{1}}{\isacharparenright}{\kern0pt}{\isachardoublequoteclose}\ \isacommand{by}\isamarkupfalse%
\ auto\isanewline
\ \ \ \ \ \ \ \ \isacommand{from}\isamarkupfalse%
\ {\isadigit{1}}{\isadigit{0}}\ x{\isacharunderscore}{\kern0pt}not{\isacharunderscore}{\kern0pt}in{\isacharunderscore}{\kern0pt}shattered\ \isacommand{have}\isamarkupfalse%
\ {\isachardoublequoteopen}x\ {\isasymnotin}\ S{\isachardoublequoteclose}\ \isacommand{by}\isamarkupfalse%
\ blast\isanewline
\ \ \ \ \ \ \ \ \isacommand{with}\isamarkupfalse%
\ {\isadigit{1}}{\isadigit{1}}\ \isacommand{show}\isamarkupfalse%
\ {\isachardoublequoteopen}False{\isachardoublequoteclose}\ \isacommand{by}\isamarkupfalse%
\ blast\isanewline
\ \ \ \ \ \ \isacommand{qed}\isamarkupfalse%
%
\begin{isamarkuptext}%
This set is also in \isa{shattered{\isacharunderscore}{\kern0pt}by\ F}.%
\end{isamarkuptext}\isamarkuptrue%
\ \ \ \ \ \ \isacommand{have}\isamarkupfalse%
\ {\isadigit{2}}{\isacharcolon}{\kern0pt}\ {\isachardoublequoteopen}{\isacharquery}{\kern0pt}f\ {\isacharbackquote}{\kern0pt}\ {\isacharparenleft}{\kern0pt}shattered{\isacharunderscore}{\kern0pt}by\ {\isacharquery}{\kern0pt}F{\isadigit{0}}\ {\isasyminter}\ shattered{\isacharunderscore}{\kern0pt}by\ {\isacharquery}{\kern0pt}F{\isadigit{1}}{\isacharparenright}{\kern0pt}\ {\isasymsubseteq}\ shattered{\isacharunderscore}{\kern0pt}by\ F{\isachardoublequoteclose}\isanewline
\ \ \ \ \ \ \isacommand{proof}\isamarkupfalse%
\ \isanewline
\ \ \ \ \ \ \ \ \isacommand{fix}\isamarkupfalse%
\ S{\isacharunderscore}{\kern0pt}x\isanewline
\ \ \ \ \ \ \ \ \isacommand{assume}\isamarkupfalse%
\ {\isachardoublequoteopen}S{\isacharunderscore}{\kern0pt}x\ {\isasymin}\ {\isacharquery}{\kern0pt}f\ {\isacharbackquote}{\kern0pt}\ {\isacharparenleft}{\kern0pt}shattered{\isacharunderscore}{\kern0pt}by\ {\isacharquery}{\kern0pt}F{\isadigit{0}}\ {\isasyminter}\ shattered{\isacharunderscore}{\kern0pt}by\ {\isacharquery}{\kern0pt}F{\isadigit{1}}{\isacharparenright}{\kern0pt}{\isachardoublequoteclose}\isanewline
\ \ \ \ \ \ \ \ \isacommand{then}\isamarkupfalse%
\ \isacommand{obtain}\isamarkupfalse%
\ S\ \isakeyword{where}\ {\isadigit{2}}{\isadigit{0}}{\isacharcolon}{\kern0pt}\ {\isachardoublequoteopen}S\ {\isasymin}\ shattered{\isacharunderscore}{\kern0pt}by\ {\isacharquery}{\kern0pt}F{\isadigit{0}}{\isachardoublequoteclose}\ \isanewline
\ \ \ \ \ \ \ \ \ \ \ \ \ \ \ \ \ \ \ \ \ \ \ \ \isakeyword{and}\ {\isadigit{2}}{\isadigit{1}}{\isacharcolon}{\kern0pt}\ {\isachardoublequoteopen}S\ {\isasymin}\ shattered{\isacharunderscore}{\kern0pt}by\ {\isacharquery}{\kern0pt}F{\isadigit{1}}{\isachardoublequoteclose}\ \isanewline
\ \ \ \ \ \ \ \ \ \ \ \ \ \ \ \ \ \ \ \ \ \ \ \ \isakeyword{and}\ {\isadigit{2}}{\isadigit{2}}{\isacharcolon}{\kern0pt}\ {\isachardoublequoteopen}S{\isacharunderscore}{\kern0pt}x\ {\isacharequal}{\kern0pt}\ {\isacharquery}{\kern0pt}f\ S{\isachardoublequoteclose}\ \isacommand{by}\isamarkupfalse%
\ blast\isanewline
\ \ \ \ \ \ \ \ \isacommand{from}\isamarkupfalse%
\ x{\isacharunderscore}{\kern0pt}not{\isacharunderscore}{\kern0pt}in{\isacharunderscore}{\kern0pt}shattered\ {\isadigit{2}}{\isadigit{0}}\ \isacommand{have}\isamarkupfalse%
\ x{\isacharunderscore}{\kern0pt}not{\isacharunderscore}{\kern0pt}in{\isacharunderscore}{\kern0pt}S{\isacharcolon}{\kern0pt}\ {\isachardoublequoteopen}x\ {\isasymnotin}\ S{\isachardoublequoteclose}\ \isacommand{by}\isamarkupfalse%
\ blast\isanewline
\isanewline
\ \ \ \ \ \ \ \ \isacommand{from}\isamarkupfalse%
\ {\isadigit{2}}{\isadigit{2}}\ Pow{\isacharunderscore}{\kern0pt}insert{\isacharbrackleft}{\kern0pt}of\ x\ S{\isacharbrackright}{\kern0pt}\ \isacommand{have}\isamarkupfalse%
\ {\isachardoublequoteopen}Pow\ S{\isacharunderscore}{\kern0pt}x\ {\isacharequal}{\kern0pt}\ Pow\ S\ {\isasymunion}\ {\isacharquery}{\kern0pt}f\ {\isacharbackquote}{\kern0pt}\ Pow\ S{\isachardoublequoteclose}\ \isacommand{by}\isamarkupfalse%
\ fast\isanewline
\ \ \ \ \ \ \ \ \isacommand{also}\isamarkupfalse%
\ \isacommand{from}\isamarkupfalse%
\ {\isadigit{2}}{\isadigit{0}}\ \isacommand{have}\isamarkupfalse%
\ {\isachardoublequoteopen}{\isachardot}{\kern0pt}{\isachardot}{\kern0pt}{\isachardot}{\kern0pt}\ {\isacharequal}{\kern0pt}\ {\isacharparenleft}{\kern0pt}{\isacharquery}{\kern0pt}F{\isadigit{0}}\ {\isasyminter}{\isacharasterisk}{\kern0pt}\ S{\isacharparenright}{\kern0pt}\ {\isasymunion}\ {\isacharparenleft}{\kern0pt}{\isacharquery}{\kern0pt}f\ {\isacharbackquote}{\kern0pt}\ Pow\ S{\isacharparenright}{\kern0pt}{\isachardoublequoteclose}\ \isacommand{unfolding}\isamarkupfalse%
\ shattered{\isacharunderscore}{\kern0pt}by{\isacharunderscore}{\kern0pt}def\ \isacommand{by}\isamarkupfalse%
\ blast\isanewline
\ \ \ \ \ \ \ \ \isacommand{also}\isamarkupfalse%
\ \isacommand{from}\isamarkupfalse%
\ {\isadigit{2}}{\isadigit{1}}\ \isacommand{have}\isamarkupfalse%
\ {\isachardoublequoteopen}{\isachardot}{\kern0pt}{\isachardot}{\kern0pt}{\isachardot}{\kern0pt}\ {\isacharequal}{\kern0pt}\ {\isacharparenleft}{\kern0pt}{\isacharquery}{\kern0pt}F{\isadigit{0}}\ {\isasyminter}{\isacharasterisk}{\kern0pt}\ S{\isacharparenright}{\kern0pt}\ {\isasymunion}\ {\isacharparenleft}{\kern0pt}{\isacharquery}{\kern0pt}f\ {\isacharbackquote}{\kern0pt}\ {\isacharparenleft}{\kern0pt}{\isacharquery}{\kern0pt}F{\isadigit{1}}\ {\isasyminter}{\isacharasterisk}{\kern0pt}\ S{\isacharparenright}{\kern0pt}{\isacharparenright}{\kern0pt}{\isachardoublequoteclose}\ \isacommand{unfolding}\isamarkupfalse%
\ shattered{\isacharunderscore}{\kern0pt}by{\isacharunderscore}{\kern0pt}def\ \isacommand{by}\isamarkupfalse%
\ force\isanewline
\ \ \ \ \ \ \ \ \isacommand{also}\isamarkupfalse%
\ \isacommand{from}\isamarkupfalse%
\ insert{\isacharunderscore}{\kern0pt}IntF{\isacharbrackleft}{\kern0pt}of\ x\ S\ {\isacharquery}{\kern0pt}F{\isadigit{1}}{\isacharbrackright}{\kern0pt}\ \isacommand{have}\isamarkupfalse%
\ {\isachardoublequoteopen}{\isachardot}{\kern0pt}{\isachardot}{\kern0pt}{\isachardot}{\kern0pt}\ {\isacharequal}{\kern0pt}\ {\isacharparenleft}{\kern0pt}{\isacharquery}{\kern0pt}F{\isadigit{0}}\ {\isasyminter}{\isacharasterisk}{\kern0pt}\ S{\isacharparenright}{\kern0pt}\ {\isasymunion}\ {\isacharparenleft}{\kern0pt}{\isacharquery}{\kern0pt}f\ {\isacharbackquote}{\kern0pt}\ {\isacharquery}{\kern0pt}F{\isadigit{1}}\ {\isasyminter}{\isacharasterisk}{\kern0pt}\ {\isacharparenleft}{\kern0pt}{\isacharquery}{\kern0pt}f\ S{\isacharparenright}{\kern0pt}{\isacharparenright}{\kern0pt}{\isachardoublequoteclose}\ \isacommand{by}\isamarkupfalse%
\ argo\isanewline
\ \ \ \ \ \ \ \ \isacommand{also}\isamarkupfalse%
\ \isacommand{from}\isamarkupfalse%
\ {\isadigit{2}}{\isadigit{2}}\ \isacommand{have}\isamarkupfalse%
\ {\isachardoublequoteopen}{\isachardot}{\kern0pt}{\isachardot}{\kern0pt}{\isachardot}{\kern0pt}\ {\isacharequal}{\kern0pt}\ {\isacharparenleft}{\kern0pt}{\isacharquery}{\kern0pt}F{\isadigit{0}}\ {\isasyminter}{\isacharasterisk}{\kern0pt}\ S{\isacharparenright}{\kern0pt}\ {\isasymunion}\ {\isacharparenleft}{\kern0pt}{\isacharquery}{\kern0pt}F{\isadigit{1}}\ {\isasyminter}{\isacharasterisk}{\kern0pt}\ S{\isacharunderscore}{\kern0pt}x{\isacharparenright}{\kern0pt}{\isachardoublequoteclose}\ \isacommand{by}\isamarkupfalse%
\ blast\isanewline
\ \ \ \ \ \ \ \ \isacommand{also}\isamarkupfalse%
\ \isacommand{from}\isamarkupfalse%
\ {\isadigit{2}}{\isadigit{2}}\ \isacommand{have}\isamarkupfalse%
\ {\isachardoublequoteopen}{\isachardot}{\kern0pt}{\isachardot}{\kern0pt}{\isachardot}{\kern0pt}\ {\isacharequal}{\kern0pt}\ {\isacharparenleft}{\kern0pt}{\isacharquery}{\kern0pt}F{\isadigit{0}}\ {\isasyminter}{\isacharasterisk}{\kern0pt}\ S{\isacharunderscore}{\kern0pt}x{\isacharparenright}{\kern0pt}\ {\isasymunion}\ {\isacharparenleft}{\kern0pt}{\isacharquery}{\kern0pt}F{\isadigit{1}}\ {\isasyminter}{\isacharasterisk}{\kern0pt}\ S{\isacharunderscore}{\kern0pt}x{\isacharparenright}{\kern0pt}{\isachardoublequoteclose}\ \isacommand{by}\isamarkupfalse%
\ blast\isanewline
\ \ \ \ \ \ \ \ \isacommand{also}\isamarkupfalse%
\ \isacommand{from}\isamarkupfalse%
\ subset{\isacharunderscore}{\kern0pt}IntF{\isacharbrackleft}{\kern0pt}OF\ F{\isadigit{0}}{\isacharunderscore}{\kern0pt}in{\isacharunderscore}{\kern0pt}F{\isacharcomma}{\kern0pt}\ of\ S{\isacharunderscore}{\kern0pt}x{\isacharbrackright}{\kern0pt}\ subset{\isacharunderscore}{\kern0pt}IntF{\isacharbrackleft}{\kern0pt}OF\ F{\isadigit{1}}{\isacharunderscore}{\kern0pt}in{\isacharunderscore}{\kern0pt}F{\isacharcomma}{\kern0pt}\ of\ S{\isacharunderscore}{\kern0pt}x{\isacharbrackright}{\kern0pt}\ \isacommand{have}\isamarkupfalse%
\ {\isachardoublequoteopen}{\isachardot}{\kern0pt}{\isachardot}{\kern0pt}{\isachardot}{\kern0pt}\ {\isasymsubseteq}\ {\isacharparenleft}{\kern0pt}F\ {\isasyminter}{\isacharasterisk}{\kern0pt}\ S{\isacharunderscore}{\kern0pt}x{\isacharparenright}{\kern0pt}{\isachardoublequoteclose}\ \isacommand{by}\isamarkupfalse%
\ blast\isanewline
\ \ \ \ \ \ \ \ \isacommand{finally}\isamarkupfalse%
\ \isacommand{have}\isamarkupfalse%
\ {\isachardoublequoteopen}Pow\ S{\isacharunderscore}{\kern0pt}x\ {\isasymsubseteq}\ {\isacharparenleft}{\kern0pt}F\ {\isasyminter}{\isacharasterisk}{\kern0pt}\ S{\isacharunderscore}{\kern0pt}x{\isacharparenright}{\kern0pt}{\isachardoublequoteclose}\ \isacommand{{\isachardot}{\kern0pt}}\isamarkupfalse%
\isanewline
\ \ \ \ \ \ \ \ \isacommand{thus}\isamarkupfalse%
\ {\isachardoublequoteopen}S{\isacharunderscore}{\kern0pt}x\ {\isasymin}\ shattered{\isacharunderscore}{\kern0pt}by\ F{\isachardoublequoteclose}\ \isacommand{unfolding}\isamarkupfalse%
\ shattered{\isacharunderscore}{\kern0pt}by{\isacharunderscore}{\kern0pt}def\ \isacommand{by}\isamarkupfalse%
\ blast\isanewline
\ \ \ \ \ \ \isacommand{qed}\isamarkupfalse%
\isanewline
\isanewline
\ \ \ \ \ \ \isacommand{from}\isamarkupfalse%
\ {\isadigit{0}}\ {\isadigit{1}}\ {\isadigit{2}}\ \isacommand{show}\isamarkupfalse%
\ {\isachardoublequoteopen}inj{\isacharunderscore}{\kern0pt}on\ {\isacharquery}{\kern0pt}f\ {\isacharparenleft}{\kern0pt}shattered{\isacharunderscore}{\kern0pt}by\ {\isacharquery}{\kern0pt}F{\isadigit{0}}\ {\isasyminter}\ shattered{\isacharunderscore}{\kern0pt}by\ {\isacharquery}{\kern0pt}F{\isadigit{1}}{\isacharparenright}{\kern0pt}\ {\isasymand}\isanewline
\ \ \ \ \ \ \ \ {\isacharparenleft}{\kern0pt}shattered{\isacharunderscore}{\kern0pt}by\ {\isacharquery}{\kern0pt}F{\isadigit{0}}\ {\isasymunion}\ shattered{\isacharunderscore}{\kern0pt}by\ {\isacharquery}{\kern0pt}F{\isadigit{1}}{\isacharparenright}{\kern0pt}\ {\isasyminter}\ {\isacharparenleft}{\kern0pt}{\isacharquery}{\kern0pt}f\ {\isacharbackquote}{\kern0pt}\ {\isacharparenleft}{\kern0pt}shattered{\isacharunderscore}{\kern0pt}by\ {\isacharquery}{\kern0pt}F{\isadigit{0}}\ {\isasyminter}\ shattered{\isacharunderscore}{\kern0pt}by\ {\isacharquery}{\kern0pt}F{\isadigit{1}}{\isacharparenright}{\kern0pt}{\isacharparenright}{\kern0pt}\ {\isacharequal}{\kern0pt}\ {\isacharbraceleft}{\kern0pt}{\isacharbraceright}{\kern0pt}\ {\isasymand}\isanewline
\ \ \ \ \ \ \ \ {\isacharquery}{\kern0pt}f\ {\isacharbackquote}{\kern0pt}\ {\isacharparenleft}{\kern0pt}shattered{\isacharunderscore}{\kern0pt}by\ {\isacharquery}{\kern0pt}F{\isadigit{0}}\ {\isasyminter}\ shattered{\isacharunderscore}{\kern0pt}by\ {\isacharquery}{\kern0pt}F{\isadigit{1}}{\isacharparenright}{\kern0pt}\ {\isasymsubseteq}\ shattered{\isacharunderscore}{\kern0pt}by\ F{\isachardoublequoteclose}\ \isacommand{by}\isamarkupfalse%
\ blast\isanewline
\ \ \ \ \isacommand{qed}\isamarkupfalse%
\isanewline
\isanewline
\ \ \ \ \isacommand{have}\isamarkupfalse%
\ F{\isadigit{0}}{\isacharunderscore}{\kern0pt}union{\isacharunderscore}{\kern0pt}F{\isadigit{1}}{\isacharunderscore}{\kern0pt}is{\isacharunderscore}{\kern0pt}F{\isacharcolon}{\kern0pt}\ {\isachardoublequoteopen}{\isacharquery}{\kern0pt}F{\isadigit{0}}\ {\isasymunion}\ {\isacharquery}{\kern0pt}F{\isadigit{1}}\ {\isacharequal}{\kern0pt}\ F{\isachardoublequoteclose}\ \isacommand{by}\isamarkupfalse%
\ fastforce\isanewline
\ \ \ \ \isacommand{from}\isamarkupfalse%
\ finite{\isacharunderscore}{\kern0pt}F\ \isacommand{have}\isamarkupfalse%
\ finite{\isacharunderscore}{\kern0pt}F{\isadigit{0}}{\isacharcolon}{\kern0pt}\ {\isachardoublequoteopen}finite\ {\isacharquery}{\kern0pt}F{\isadigit{0}}{\isachardoublequoteclose}\ \isakeyword{and}\ finite{\isacharunderscore}{\kern0pt}F{\isadigit{1}}{\isacharcolon}{\kern0pt}\ {\isachardoublequoteopen}finite\ {\isacharquery}{\kern0pt}F{\isadigit{1}}{\isachardoublequoteclose}\ \isacommand{by}\isamarkupfalse%
\ fastforce{\isacharplus}{\kern0pt}\isanewline
\ \ \ \ \isacommand{have}\isamarkupfalse%
\ disjoint{\isacharunderscore}{\kern0pt}F{\isadigit{0}}{\isacharunderscore}{\kern0pt}F{\isadigit{1}}{\isacharcolon}{\kern0pt}\ {\isachardoublequoteopen}{\isacharquery}{\kern0pt}F{\isadigit{0}}\ {\isasyminter}\ {\isacharquery}{\kern0pt}F{\isadigit{1}}\ {\isacharequal}{\kern0pt}\ {\isacharbraceleft}{\kern0pt}{\isacharbraceright}{\kern0pt}{\isachardoublequoteclose}\ \isacommand{by}\isamarkupfalse%
\ fastforce%
\begin{isamarkuptext}%
Thus we have the following lower bound on the cardinality of \isa{shattered{\isacharunderscore}{\kern0pt}by\ F}%
\end{isamarkuptext}\isamarkuptrue%
\ \ \ \ \isacommand{from}\isamarkupfalse%
\ F{\isadigit{0}}{\isacharunderscore}{\kern0pt}union{\isacharunderscore}{\kern0pt}F{\isadigit{1}}{\isacharunderscore}{\kern0pt}is{\isacharunderscore}{\kern0pt}F\ card{\isacharunderscore}{\kern0pt}Un{\isacharunderscore}{\kern0pt}disjoint{\isacharbrackleft}{\kern0pt}OF\ finite{\isacharunderscore}{\kern0pt}F{\isadigit{0}}\ finite{\isacharunderscore}{\kern0pt}F{\isadigit{1}}\ disjoint{\isacharunderscore}{\kern0pt}F{\isadigit{0}}{\isacharunderscore}{\kern0pt}F{\isadigit{1}}{\isacharbrackright}{\kern0pt}\ \isanewline
\ \ \ \ \isacommand{have}\isamarkupfalse%
\ {\isachardoublequoteopen}card\ F\ {\isacharequal}{\kern0pt}\ card\ {\isacharquery}{\kern0pt}F{\isadigit{0}}\ {\isacharplus}{\kern0pt}\ card\ {\isacharquery}{\kern0pt}F{\isadigit{1}}{\isachardoublequoteclose}\ \isacommand{by}\isamarkupfalse%
\ argo\isanewline
\ \ \ \ \isacommand{also}\isamarkupfalse%
\ \isacommand{from}\isamarkupfalse%
\ IH{\isacharunderscore}{\kern0pt}F{\isadigit{0}}\isanewline
\ \ \ \ \isacommand{have}\isamarkupfalse%
\ {\isachardoublequoteopen}{\isachardot}{\kern0pt}{\isachardot}{\kern0pt}{\isachardot}{\kern0pt}\ {\isasymle}\ card\ {\isacharparenleft}{\kern0pt}shattered{\isacharunderscore}{\kern0pt}by\ {\isacharquery}{\kern0pt}F{\isadigit{0}}{\isacharparenright}{\kern0pt}\ {\isacharplus}{\kern0pt}\ card\ {\isacharquery}{\kern0pt}F{\isadigit{1}}{\isachardoublequoteclose}\ \isacommand{by}\isamarkupfalse%
\ linarith\isanewline
\ \ \ \ \isacommand{also}\isamarkupfalse%
\ \isacommand{from}\isamarkupfalse%
\ IH{\isacharunderscore}{\kern0pt}F{\isadigit{1}}\isanewline
\ \ \ \ \isacommand{have}\isamarkupfalse%
\ {\isachardoublequoteopen}{\isachardot}{\kern0pt}{\isachardot}{\kern0pt}{\isachardot}{\kern0pt}\ {\isasymle}\ card\ {\isacharparenleft}{\kern0pt}shattered{\isacharunderscore}{\kern0pt}by\ {\isacharquery}{\kern0pt}F{\isadigit{0}}{\isacharparenright}{\kern0pt}\ {\isacharplus}{\kern0pt}\ card\ {\isacharparenleft}{\kern0pt}shattered{\isacharunderscore}{\kern0pt}by\ {\isacharquery}{\kern0pt}F{\isadigit{1}}{\isacharparenright}{\kern0pt}{\isachardoublequoteclose}\ \isacommand{by}\isamarkupfalse%
\ linarith\isanewline
\ \ \ \ \isacommand{also}\isamarkupfalse%
\ \isacommand{from}\isamarkupfalse%
\ card{\isacharunderscore}{\kern0pt}Int{\isacharunderscore}{\kern0pt}copy{\isacharbrackleft}{\kern0pt}OF\ finite{\isacharunderscore}{\kern0pt}shF\ shattered{\isacharunderscore}{\kern0pt}subset\ f{\isacharunderscore}{\kern0pt}copies{\isacharunderscore}{\kern0pt}the{\isacharunderscore}{\kern0pt}intersection{\isacharbrackright}{\kern0pt}\isanewline
\ \ \ \ \isacommand{have}\isamarkupfalse%
\ {\isachardoublequoteopen}{\isachardot}{\kern0pt}{\isachardot}{\kern0pt}{\isachardot}{\kern0pt}\ {\isasymle}\ card\ {\isacharparenleft}{\kern0pt}shattered{\isacharunderscore}{\kern0pt}by\ F{\isacharparenright}{\kern0pt}{\isachardoublequoteclose}\ \isacommand{by}\isamarkupfalse%
\ argo\isanewline
\ \ \ \ \isacommand{finally}\isamarkupfalse%
\ \isacommand{show}\isamarkupfalse%
\ {\isacharquery}{\kern0pt}thesis\ \isacommand{{\isachardot}{\kern0pt}}\isamarkupfalse%
\isanewline
\ \ \isacommand{next}\isamarkupfalse%
%
\begin{isamarkuptext}%
If \isa{F} contains less than 2 sets, the statement follows trivially%
\end{isamarkuptext}\isamarkuptrue%
\ \ \ \ \isacommand{case}\isamarkupfalse%
\ False\isanewline
\ \ \ \ \isacommand{hence}\isamarkupfalse%
\ {\isachardoublequoteopen}card\ F\ {\isacharequal}{\kern0pt}\ {\isadigit{0}}\ {\isasymor}\ card\ F\ {\isacharequal}{\kern0pt}\ {\isadigit{1}}{\isachardoublequoteclose}\ \isacommand{by}\isamarkupfalse%
\ force\isanewline
\ \ \ \ \isacommand{thus}\isamarkupfalse%
\ {\isacharquery}{\kern0pt}thesis\isanewline
\ \ \ \ \isacommand{proof}\isamarkupfalse%
\isanewline
\ \ \ \ \ \ \isacommand{assume}\isamarkupfalse%
\ {\isachardoublequoteopen}card\ F\ {\isacharequal}{\kern0pt}\ {\isadigit{0}}{\isachardoublequoteclose}\isanewline
\ \ \ \ \ \ \isacommand{thus}\isamarkupfalse%
\ {\isacharquery}{\kern0pt}thesis\ \isacommand{by}\isamarkupfalse%
\ auto\isanewline
\ \ \ \ \isacommand{next}\isamarkupfalse%
\isanewline
\ \ \ \ \ \ \isacommand{assume}\isamarkupfalse%
\ asm{\isacharcolon}{\kern0pt}\ {\isachardoublequoteopen}card\ F\ {\isacharequal}{\kern0pt}\ {\isadigit{1}}{\isachardoublequoteclose}\isanewline
\ \ \ \ \ \ \isacommand{hence}\isamarkupfalse%
\ F{\isacharunderscore}{\kern0pt}not{\isacharunderscore}{\kern0pt}empty{\isacharcolon}{\kern0pt}\ {\isachardoublequoteopen}F\ {\isasymnoteq}\ {\isacharbraceleft}{\kern0pt}{\isacharbraceright}{\kern0pt}{\isachardoublequoteclose}\ \isacommand{by}\isamarkupfalse%
\ fastforce\isanewline
\ \ \ \ \ \ \isacommand{from}\isamarkupfalse%
\ shatters{\isacharunderscore}{\kern0pt}empty{\isacharbrackleft}{\kern0pt}OF\ F{\isacharunderscore}{\kern0pt}not{\isacharunderscore}{\kern0pt}empty{\isacharbrackright}{\kern0pt}\ \isacommand{have}\isamarkupfalse%
\ {\isachardoublequoteopen}{\isacharbraceleft}{\kern0pt}{\isacharbraceleft}{\kern0pt}{\isacharbraceright}{\kern0pt}{\isacharbraceright}{\kern0pt}\ {\isasymsubseteq}\ shattered{\isacharunderscore}{\kern0pt}by\ F{\isachardoublequoteclose}\ \isacommand{unfolding}\isamarkupfalse%
\ shattered{\isacharunderscore}{\kern0pt}by{\isacharunderscore}{\kern0pt}def\ \isacommand{by}\isamarkupfalse%
\ fastforce\isanewline
\ \ \ \ \ \ \isacommand{from}\isamarkupfalse%
\ card{\isacharunderscore}{\kern0pt}mono{\isacharbrackleft}{\kern0pt}OF\ finite{\isacharunderscore}{\kern0pt}shF\ this{\isacharbrackright}{\kern0pt}\ asm\ \isacommand{show}\isamarkupfalse%
\ {\isacharquery}{\kern0pt}thesis\ \isacommand{by}\isamarkupfalse%
\ fastforce\isanewline
\ \ \ \ \isacommand{qed}\isamarkupfalse%
\isanewline
\ \ \isacommand{qed}\isamarkupfalse%
\isanewline
\isacommand{qed}\isamarkupfalse%
%
\endisatagproof
{\isafoldproof}%
%
\isadelimproof
%
\endisadelimproof
%
\isadelimdocument
%
\endisadelimdocument
%
\isatagdocument
%
\isamarkupsubsection{Sauer-Shelah Lemma%
}
\isamarkuptrue%
%
\endisatagdocument
{\isafolddocument}%
%
\isadelimdocument
%
\endisadelimdocument
%
\begin{isamarkuptext}%
The generalized version immediately implies the Sauer–Shelah Lemma,
      because only \isa{{\isacharparenleft}{\kern0pt}{\isasymSum}i{\isasymle}k{\isachardot}{\kern0pt}\ n\ choose\ i{\isacharparenright}{\kern0pt}} of the subsets of an \isa{n}-item universe have cardinality less than \isa{k\ {\isacharplus}{\kern0pt}\ {\isadigit{1}}}.
      Thus, when \isa{{\isacharparenleft}{\kern0pt}{\isasymSum}i{\isasymle}k{\isachardot}{\kern0pt}\ n\ choose\ i{\isacharparenright}{\kern0pt}\ {\isacharless}{\kern0pt}\ card\ F}, there are not enough sets to be shattered, 
      so one of the shattered sets must have cardinality at least \isa{k\ {\isacharplus}{\kern0pt}\ {\isadigit{1}}}%
\end{isamarkuptext}\isamarkuptrue%
\isacommand{corollary}\isamarkupfalse%
\ sauer{\isacharunderscore}{\kern0pt}shelah{\isacharcolon}{\kern0pt}\isanewline
\ \ \isakeyword{fixes}\ F\ {\isacharcolon}{\kern0pt}{\isacharcolon}{\kern0pt}\ {\isachardoublequoteopen}{\isacharprime}{\kern0pt}a\ set\ set{\isachardoublequoteclose}\isanewline
\ \ \isakeyword{assumes}\ {\isachardoublequoteopen}finite\ {\isacharparenleft}{\kern0pt}{\isasymUnion}F{\isacharparenright}{\kern0pt}{\isachardoublequoteclose}\ \isakeyword{and}\ {\isachardoublequoteopen}{\isacharparenleft}{\kern0pt}{\isasymSum}i{\isasymle}k{\isachardot}{\kern0pt}\ card\ {\isacharparenleft}{\kern0pt}{\isasymUnion}F{\isacharparenright}{\kern0pt}\ choose\ i{\isacharparenright}{\kern0pt}\ {\isacharless}{\kern0pt}\ card\ F{\isachardoublequoteclose}\isanewline
\ \ \isakeyword{shows}\ {\isachardoublequoteopen}{\isasymexists}S{\isachardot}{\kern0pt}\ {\isacharparenleft}{\kern0pt}F\ shatters\ S\ {\isasymand}\ card\ S\ {\isacharequal}{\kern0pt}\ k\ {\isacharplus}{\kern0pt}\ {\isadigit{1}}{\isacharparenright}{\kern0pt}{\isachardoublequoteclose}\isanewline
%
\isadelimproof
%
\endisadelimproof
%
\isatagproof
\isacommand{proof}\isamarkupfalse%
\ {\isacharminus}{\kern0pt}\isanewline
\ \ \isacommand{let}\isamarkupfalse%
\ {\isacharquery}{\kern0pt}K\ {\isacharequal}{\kern0pt}\ {\isachardoublequoteopen}{\isacharbraceleft}{\kern0pt}S{\isachardot}{\kern0pt}\ S\ {\isasymsubseteq}\ {\isasymUnion}F\ {\isasymand}\ card\ S\ {\isasymle}\ k{\isacharbraceright}{\kern0pt}{\isachardoublequoteclose}\isanewline
\ \ \isacommand{from}\isamarkupfalse%
\ finite{\isacharunderscore}{\kern0pt}Pow{\isacharunderscore}{\kern0pt}iff{\isacharbrackleft}{\kern0pt}of\ F{\isacharbrackright}{\kern0pt}\ assms{\isacharparenleft}{\kern0pt}{\isadigit{1}}{\isacharparenright}{\kern0pt}\ \isacommand{have}\isamarkupfalse%
\ finite{\isacharunderscore}{\kern0pt}Pow{\isacharunderscore}{\kern0pt}Un{\isacharcolon}{\kern0pt}\ {\isachardoublequoteopen}finite\ {\isacharparenleft}{\kern0pt}Pow\ {\isacharparenleft}{\kern0pt}{\isasymUnion}F{\isacharparenright}{\kern0pt}{\isacharparenright}{\kern0pt}{\isachardoublequoteclose}\ \isacommand{by}\isamarkupfalse%
\ fast\isanewline
\isanewline
\ \ \isacommand{from}\isamarkupfalse%
\ sauer{\isacharunderscore}{\kern0pt}shelah{\isacharunderscore}{\kern0pt}{\isadigit{0}}{\isacharbrackleft}{\kern0pt}OF\ assms{\isacharparenleft}{\kern0pt}{\isadigit{1}}{\isacharparenright}{\kern0pt}{\isacharbrackright}{\kern0pt}\ assms{\isacharparenleft}{\kern0pt}{\isadigit{2}}{\isacharparenright}{\kern0pt}\ \isacommand{have}\isamarkupfalse%
\ {\isachardoublequoteopen}{\isacharparenleft}{\kern0pt}{\isasymSum}i{\isasymle}k{\isachardot}{\kern0pt}\ card\ {\isacharparenleft}{\kern0pt}{\isasymUnion}F{\isacharparenright}{\kern0pt}\ choose\ i{\isacharparenright}{\kern0pt}\ {\isacharless}{\kern0pt}\ card\ {\isacharparenleft}{\kern0pt}shattered{\isacharunderscore}{\kern0pt}by\ F{\isacharparenright}{\kern0pt}{\isachardoublequoteclose}\ \isacommand{by}\isamarkupfalse%
\ linarith\isanewline
\ \ \isacommand{with}\isamarkupfalse%
\ choose{\isacharunderscore}{\kern0pt}row{\isacharunderscore}{\kern0pt}sum{\isacharunderscore}{\kern0pt}set{\isacharbrackleft}{\kern0pt}OF\ assms{\isacharparenleft}{\kern0pt}{\isadigit{1}}{\isacharparenright}{\kern0pt}{\isacharcomma}{\kern0pt}\ of\ k{\isacharbrackright}{\kern0pt}\ \isacommand{have}\isamarkupfalse%
\ {\isachardoublequoteopen}card\ {\isacharquery}{\kern0pt}K\ {\isacharless}{\kern0pt}\ card\ {\isacharparenleft}{\kern0pt}shattered{\isacharunderscore}{\kern0pt}by\ F{\isacharparenright}{\kern0pt}{\isachardoublequoteclose}\ \isacommand{by}\isamarkupfalse%
\ presburger\isanewline
\isanewline
\ \ \isacommand{from}\isamarkupfalse%
\ finite{\isacharunderscore}{\kern0pt}diff{\isacharunderscore}{\kern0pt}not{\isacharunderscore}{\kern0pt}empty{\isacharbrackleft}{\kern0pt}OF\ finite{\isacharunderscore}{\kern0pt}subset{\isacharbrackleft}{\kern0pt}OF\ {\isacharunderscore}{\kern0pt}\ finite{\isacharunderscore}{\kern0pt}Pow{\isacharunderscore}{\kern0pt}Un{\isacharbrackright}{\kern0pt}\ this{\isacharbrackright}{\kern0pt}\ \isanewline
\ \ \isacommand{obtain}\isamarkupfalse%
\ S\ \isakeyword{where}\ {\isachardoublequoteopen}S\ {\isasymin}\ shattered{\isacharunderscore}{\kern0pt}by\ F\ {\isacharminus}{\kern0pt}\ {\isacharquery}{\kern0pt}K{\isachardoublequoteclose}\ \isacommand{by}\isamarkupfalse%
\ blast\isanewline
\ \ \isacommand{then}\isamarkupfalse%
\ \isacommand{have}\isamarkupfalse%
\ F{\isacharunderscore}{\kern0pt}shatters{\isacharunderscore}{\kern0pt}S{\isacharcolon}{\kern0pt}\ {\isachardoublequoteopen}F\ shatters\ S{\isachardoublequoteclose}\ \isakeyword{and}\ {\isachardoublequoteopen}S\ {\isasymsubseteq}\ {\isasymUnion}F{\isachardoublequoteclose}\ \isakeyword{and}\ {\isachardoublequoteopen}{\isasymnot}{\isacharparenleft}{\kern0pt}S\ {\isasymsubseteq}\ {\isasymUnion}F\ {\isasymand}\ card\ S\ {\isasymle}\ k{\isacharparenright}{\kern0pt}{\isachardoublequoteclose}\ \isacommand{unfolding}\isamarkupfalse%
\ shattered{\isacharunderscore}{\kern0pt}by{\isacharunderscore}{\kern0pt}def\ \isacommand{by}\isamarkupfalse%
\ blast{\isacharplus}{\kern0pt}\isanewline
\ \ \isacommand{then}\isamarkupfalse%
\ \isacommand{have}\isamarkupfalse%
\ card{\isacharunderscore}{\kern0pt}S{\isacharunderscore}{\kern0pt}ge{\isacharunderscore}{\kern0pt}Suc{\isacharunderscore}{\kern0pt}k{\isacharcolon}{\kern0pt}\ {\isachardoublequoteopen}k\ {\isacharplus}{\kern0pt}\ {\isadigit{1}}\ {\isasymle}\ card\ S{\isachardoublequoteclose}\ \isacommand{by}\isamarkupfalse%
\ simp\isanewline
\ \ \isacommand{from}\isamarkupfalse%
\ obtain{\isacharunderscore}{\kern0pt}subset{\isacharunderscore}{\kern0pt}with{\isacharunderscore}{\kern0pt}card{\isacharunderscore}{\kern0pt}n{\isacharbrackleft}{\kern0pt}OF\ card{\isacharunderscore}{\kern0pt}S{\isacharunderscore}{\kern0pt}ge{\isacharunderscore}{\kern0pt}Suc{\isacharunderscore}{\kern0pt}k{\isacharbrackright}{\kern0pt}\ \isacommand{obtain}\isamarkupfalse%
\ S{\isacharprime}{\kern0pt}\ \isakeyword{where}\ {\isachardoublequoteopen}card\ S{\isacharprime}{\kern0pt}\ {\isacharequal}{\kern0pt}\ k\ {\isacharplus}{\kern0pt}\ {\isadigit{1}}{\isachardoublequoteclose}\ \isakeyword{and}\ {\isachardoublequoteopen}S{\isacharprime}{\kern0pt}\ {\isasymsubseteq}\ S{\isachardoublequoteclose}\ \isacommand{by}\isamarkupfalse%
\ blast\isanewline
\ \ \isacommand{from}\isamarkupfalse%
\ this{\isacharparenleft}{\kern0pt}{\isadigit{1}}{\isacharparenright}{\kern0pt}\ supset{\isacharunderscore}{\kern0pt}shatters{\isacharbrackleft}{\kern0pt}OF\ this{\isacharparenleft}{\kern0pt}{\isadigit{2}}{\isacharparenright}{\kern0pt}\ F{\isacharunderscore}{\kern0pt}shatters{\isacharunderscore}{\kern0pt}S{\isacharbrackright}{\kern0pt}\ \isacommand{show}\isamarkupfalse%
\ {\isacharquery}{\kern0pt}thesis\ \isacommand{by}\isamarkupfalse%
\ blast\isanewline
\isacommand{qed}\isamarkupfalse%
%
\endisatagproof
{\isafoldproof}%
%
\isadelimproof
%
\endisadelimproof
%
\isadelimdocument
%
\endisadelimdocument
%
\isatagdocument
%
\isamarkupsubsection{Sauer-Shelah Lemma for hypergraphs%
}
\isamarkuptrue%
%
\endisatagdocument
{\isafolddocument}%
%
\isadelimdocument
%
\endisadelimdocument
\isacommand{corollary}\isamarkupfalse%
\ sauer{\isacharunderscore}{\kern0pt}shelah{\isacharunderscore}{\kern0pt}{\isadigit{2}}{\isacharcolon}{\kern0pt}\isanewline
\ \ \isakeyword{fixes}\ X\ {\isacharcolon}{\kern0pt}{\isacharcolon}{\kern0pt}\ {\isachardoublequoteopen}{\isacharprime}{\kern0pt}a\ set\ set{\isachardoublequoteclose}\ \isakeyword{and}\ S\ {\isacharcolon}{\kern0pt}{\isacharcolon}{\kern0pt}\ {\isachardoublequoteopen}{\isacharprime}{\kern0pt}a\ set{\isachardoublequoteclose}\isanewline
\ \ \isakeyword{assumes}\ {\isachardoublequoteopen}finite\ S{\isachardoublequoteclose}\ \isakeyword{and}\ {\isachardoublequoteopen}X\ {\isasymsubseteq}\ Pow\ S{\isachardoublequoteclose}\ \isakeyword{and}\ {\isachardoublequoteopen}{\isacharparenleft}{\kern0pt}{\isasymSum}i{\isasymle}k{\isachardot}{\kern0pt}\ card\ S\ choose\ i{\isacharparenright}{\kern0pt}\ {\isacharless}{\kern0pt}\ card\ X{\isachardoublequoteclose}\isanewline
\ \ \isakeyword{shows}\ {\isachardoublequoteopen}{\isasymexists}Y{\isachardot}{\kern0pt}\ {\isacharparenleft}{\kern0pt}X\ shatters\ Y\ {\isasymand}\ card\ Y\ {\isacharequal}{\kern0pt}\ k\ {\isacharplus}{\kern0pt}\ {\isadigit{1}}{\isacharparenright}{\kern0pt}{\isachardoublequoteclose}\isanewline
%
\isadelimproof
%
\endisadelimproof
%
\isatagproof
\isacommand{proof}\isamarkupfalse%
\ {\isacharminus}{\kern0pt}\isanewline
\ \ \isacommand{from}\isamarkupfalse%
\ assms{\isacharparenleft}{\kern0pt}{\isadigit{2}}{\isacharparenright}{\kern0pt}\ \isacommand{have}\isamarkupfalse%
\ {\isadigit{0}}{\isacharcolon}{\kern0pt}\ {\isachardoublequoteopen}{\isasymUnion}X\ {\isasymsubseteq}\ S{\isachardoublequoteclose}\ \isacommand{by}\isamarkupfalse%
\ blast\isanewline
\ \ \isacommand{from}\isamarkupfalse%
\ sum{\isacharunderscore}{\kern0pt}mono{\isacharbrackleft}{\kern0pt}OF\ choose{\isacharunderscore}{\kern0pt}mono{\isacharbrackleft}{\kern0pt}OF\ card{\isacharunderscore}{\kern0pt}mono{\isacharbrackleft}{\kern0pt}OF\ assms{\isacharparenleft}{\kern0pt}{\isadigit{1}}{\isacharparenright}{\kern0pt}\ {\isadigit{0}}{\isacharbrackright}{\kern0pt}{\isacharbrackright}{\kern0pt}{\isacharbrackright}{\kern0pt}\ \isacommand{have}\isamarkupfalse%
\ {\isachardoublequoteopen}{\isacharparenleft}{\kern0pt}{\isasymSum}i{\isasymle}k{\isachardot}{\kern0pt}\ card\ {\isacharparenleft}{\kern0pt}{\isasymUnion}X{\isacharparenright}{\kern0pt}\ choose\ i{\isacharparenright}{\kern0pt}\ {\isasymle}\ {\isacharparenleft}{\kern0pt}{\isasymSum}i{\isasymle}k{\isachardot}{\kern0pt}\ card\ S\ choose\ i{\isacharparenright}{\kern0pt}{\isachardoublequoteclose}\ \isacommand{by}\isamarkupfalse%
\ fast\isanewline
\ \ \isacommand{with}\isamarkupfalse%
\ sauer{\isacharunderscore}{\kern0pt}shelah{\isacharbrackleft}{\kern0pt}OF\ finite{\isacharunderscore}{\kern0pt}subset{\isacharbrackleft}{\kern0pt}OF\ {\isadigit{0}}\ assms{\isacharparenleft}{\kern0pt}{\isadigit{1}}{\isacharparenright}{\kern0pt}{\isacharbrackright}{\kern0pt}{\isacharbrackright}{\kern0pt}\ assms{\isacharparenleft}{\kern0pt}{\isadigit{3}}{\isacharparenright}{\kern0pt}\ \isacommand{show}\isamarkupfalse%
\ {\isacharquery}{\kern0pt}thesis\ \isacommand{by}\isamarkupfalse%
\ simp\isanewline
\isacommand{qed}\isamarkupfalse%
%
\endisatagproof
{\isafoldproof}%
%
\isadelimproof
%
\endisadelimproof
%
\isadelimdocument
%
\endisadelimdocument
%
\isatagdocument
%
\isamarkupsubsection{Alternative statement of the Sauer-Shelah Lemma%
}
\isamarkuptrue%
%
\endisatagdocument
{\isafolddocument}%
%
\isadelimdocument
%
\endisadelimdocument
%
\begin{isamarkuptext}%
We can also state the Sauer–Shelah Lemma in terms of the \isa{VC{\isacharunderscore}{\kern0pt}dim}. If the VC dimension of \isa{F} is \isa{k}, then \isa{F}
      can consist at most of \isa{{\isacharparenleft}{\kern0pt}{\isasymSum}i{\isasymle}k{\isachardot}{\kern0pt}\ card\ {\isacharparenleft}{\kern0pt}{\isasymUnion}F{\isacharparenright}{\kern0pt}\ choose\ i{\isacharparenright}{\kern0pt}} sets, which is in \isa{{\isasymO}{\isacharparenleft}{\kern0pt}card\ {\isacharparenleft}{\kern0pt}{\isasymUnion}F{\isacharparenright}{\kern0pt}{\isacharcircum}{\kern0pt}k{\isacharparenright}{\kern0pt}}%
\end{isamarkuptext}\isamarkuptrue%
\isacommand{corollary}\isamarkupfalse%
\ sauer{\isacharunderscore}{\kern0pt}shelah{\isacharunderscore}{\kern0pt}alt{\isacharcolon}{\kern0pt}\isanewline
\ \ \isakeyword{assumes}\ {\isachardoublequoteopen}finite\ {\isacharparenleft}{\kern0pt}{\isasymUnion}F{\isacharparenright}{\kern0pt}{\isachardoublequoteclose}\ \isakeyword{and}\ {\isachardoublequoteopen}VC{\isacharunderscore}{\kern0pt}dim\ F\ {\isacharequal}{\kern0pt}\ k{\isachardoublequoteclose}\isanewline
\ \ \isakeyword{shows}\ {\isachardoublequoteopen}card\ F\ {\isasymle}\ {\isacharparenleft}{\kern0pt}{\isasymSum}i{\isasymle}k{\isachardot}{\kern0pt}\ card\ {\isacharparenleft}{\kern0pt}{\isasymUnion}F{\isacharparenright}{\kern0pt}\ choose\ i{\isacharparenright}{\kern0pt}{\isachardoublequoteclose}\isanewline
%
\isadelimproof
%
\endisadelimproof
%
\isatagproof
\isacommand{proof}\isamarkupfalse%
\ {\isacharparenleft}{\kern0pt}rule\ ccontr{\isacharparenright}{\kern0pt}\isanewline
\ \ \isacommand{assume}\isamarkupfalse%
\ {\isachardoublequoteopen}{\isasymnot}\ card\ F\ {\isasymle}\ {\isacharparenleft}{\kern0pt}{\isasymSum}i{\isasymle}k{\isachardot}{\kern0pt}\ card\ {\isacharparenleft}{\kern0pt}{\isasymUnion}F{\isacharparenright}{\kern0pt}\ choose\ i{\isacharparenright}{\kern0pt}{\isachardoublequoteclose}\ \isacommand{hence}\isamarkupfalse%
\ {\isachardoublequoteopen}{\isacharparenleft}{\kern0pt}{\isasymSum}i{\isasymle}k{\isachardot}{\kern0pt}\ card\ {\isacharparenleft}{\kern0pt}{\isasymUnion}F{\isacharparenright}{\kern0pt}\ choose\ i{\isacharparenright}{\kern0pt}\ {\isacharless}{\kern0pt}\ card\ F{\isachardoublequoteclose}\ \isacommand{by}\isamarkupfalse%
\ linarith\isanewline
\ \ \isacommand{from}\isamarkupfalse%
\ sauer{\isacharunderscore}{\kern0pt}shelah{\isacharbrackleft}{\kern0pt}OF\ assms{\isacharparenleft}{\kern0pt}{\isadigit{1}}{\isacharparenright}{\kern0pt}\ this{\isacharbrackright}{\kern0pt}\ \isacommand{obtain}\isamarkupfalse%
\ S\ \isakeyword{where}\ {\isachardoublequoteopen}F\ shatters\ S{\isachardoublequoteclose}\ \isakeyword{and}\ {\isachardoublequoteopen}card\ S\ {\isacharequal}{\kern0pt}\ k\ {\isacharplus}{\kern0pt}\ {\isadigit{1}}{\isachardoublequoteclose}\ \isacommand{by}\isamarkupfalse%
\ blast\isanewline
\ \ \isacommand{from}\isamarkupfalse%
\ this{\isacharparenleft}{\kern0pt}{\isadigit{1}}{\isacharparenright}{\kern0pt}\ this{\isacharparenleft}{\kern0pt}{\isadigit{2}}{\isacharparenright}{\kern0pt}{\isacharbrackleft}{\kern0pt}symmetric{\isacharbrackright}{\kern0pt}\ \isacommand{have}\isamarkupfalse%
\ {\isachardoublequoteopen}k\ {\isacharplus}{\kern0pt}\ {\isadigit{1}}\ {\isasymin}\ {\isacharbraceleft}{\kern0pt}card\ S\ {\isacharbar}{\kern0pt}\ S{\isachardot}{\kern0pt}\ F\ shatters\ S{\isacharbraceright}{\kern0pt}{\isachardoublequoteclose}\ \isacommand{by}\isamarkupfalse%
\ blast\isanewline
\ \ \isacommand{from}\isamarkupfalse%
\ cSup{\isacharunderscore}{\kern0pt}upper{\isacharbrackleft}{\kern0pt}OF\ this\ bdd{\isacharunderscore}{\kern0pt}above{\isacharunderscore}{\kern0pt}finite{\isacharbrackleft}{\kern0pt}OF\ finite{\isacharunderscore}{\kern0pt}image{\isacharunderscore}{\kern0pt}set{\isacharbrackleft}{\kern0pt}OF\ finite{\isacharunderscore}{\kern0pt}shattered{\isacharunderscore}{\kern0pt}by{\isacharbrackleft}{\kern0pt}unfolded\ shattered{\isacharunderscore}{\kern0pt}by{\isacharunderscore}{\kern0pt}def{\isacharcomma}{\kern0pt}\ OF\ assms{\isacharparenleft}{\kern0pt}{\isadigit{1}}{\isacharparenright}{\kern0pt}{\isacharbrackright}{\kern0pt}{\isacharbrackright}{\kern0pt}{\isacharbrackright}{\kern0pt}{\isacharcomma}{\kern0pt}\ folded\ VC{\isacharunderscore}{\kern0pt}dim{\isacharunderscore}{\kern0pt}def{\isacharbrackright}{\kern0pt}\ \isanewline
\ \ \ \ \ \ \ assms{\isacharparenleft}{\kern0pt}{\isadigit{2}}{\isacharparenright}{\kern0pt}\ \isacommand{show}\isamarkupfalse%
\ False\ \isacommand{by}\isamarkupfalse%
\ force\isanewline
\isacommand{qed}\isamarkupfalse%
%
\endisatagproof
{\isafoldproof}%
%
\isadelimproof
\isanewline
%
\endisadelimproof
%
\isadelimtheory
\isanewline
%
\endisadelimtheory
%
\isatagtheory
\isacommand{end}\isamarkupfalse%
%
\endisatagtheory
{\isafoldtheory}%
%
\isadelimtheory
%
\endisadelimtheory
%
\end{isabellebody}%
\endinput
%:%file=Sauer_Shelah_Lemma.tex%:%
%:%11=5%:%
%:%27=7%:%
%:%28=7%:%
%:%29=8%:%
%:%30=9%:%
%:%44=11%:%
%:%56=13%:%
%:%57=14%:%
%:%58=15%:%
%:%59=16%:%
%:%60=17%:%
%:%61=18%:%
%:%62=19%:%
%:%63=20%:%
%:%64=21%:%
%:%65=22%:%
%:%66=23%:%
%:%67=24%:%
%:%69=26%:%
%:%70=26%:%
%:%71=27%:%
%:%72=28%:%
%:%79=29%:%
%:%80=29%:%
%:%81=30%:%
%:%82=30%:%
%:%83=31%:%
%:%84=31%:%
%:%85=32%:%
%:%86=32%:%
%:%87=33%:%
%:%88=33%:%
%:%89=34%:%
%:%90=34%:%
%:%91=35%:%
%:%92=35%:%
%:%93=36%:%
%:%94=36%:%
%:%95=37%:%
%:%96=37%:%
%:%97=38%:%
%:%98=38%:%
%:%100=40%:%
%:%102=41%:%
%:%103=41%:%
%:%104=42%:%
%:%105=42%:%
%:%106=42%:%
%:%107=42%:%
%:%108=43%:%
%:%109=43%:%
%:%110=43%:%
%:%111=43%:%
%:%112=44%:%
%:%113=44%:%
%:%114=44%:%
%:%115=44%:%
%:%116=45%:%
%:%117=45%:%
%:%118=45%:%
%:%119=45%:%
%:%121=47%:%
%:%123=48%:%
%:%124=48%:%
%:%126=50%:%
%:%128=51%:%
%:%129=51%:%
%:%130=52%:%
%:%131=52%:%
%:%132=52%:%
%:%133=52%:%
%:%134=53%:%
%:%135=53%:%
%:%136=53%:%
%:%137=53%:%
%:%138=54%:%
%:%139=54%:%
%:%140=54%:%
%:%141=54%:%
%:%142=55%:%
%:%143=55%:%
%:%144=55%:%
%:%145=55%:%
%:%147=57%:%
%:%149=58%:%
%:%150=58%:%
%:%151=59%:%
%:%152=60%:%
%:%153=60%:%
%:%154=61%:%
%:%155=61%:%
%:%156=61%:%
%:%158=63%:%
%:%159=64%:%
%:%160=65%:%
%:%162=66%:%
%:%163=66%:%
%:%164=67%:%
%:%166=69%:%
%:%167=70%:%
%:%168=70%:%
%:%169=71%:%
%:%170=71%:%
%:%171=71%:%
%:%172=71%:%
%:%174=73%:%
%:%176=74%:%
%:%177=74%:%
%:%178=75%:%
%:%179=75%:%
%:%180=76%:%
%:%181=76%:%
%:%182=77%:%
%:%183=77%:%
%:%184=78%:%
%:%185=78%:%
%:%186=79%:%
%:%187=80%:%
%:%188=80%:%
%:%189=80%:%
%:%190=80%:%
%:%191=81%:%
%:%192=81%:%
%:%193=81%:%
%:%194=81%:%
%:%195=81%:%
%:%196=82%:%
%:%197=82%:%
%:%198=82%:%
%:%199=82%:%
%:%200=82%:%
%:%201=83%:%
%:%202=83%:%
%:%203=83%:%
%:%204=83%:%
%:%205=84%:%
%:%208=86%:%
%:%210=87%:%
%:%211=87%:%
%:%212=88%:%
%:%213=88%:%
%:%214=89%:%
%:%215=89%:%
%:%216=90%:%
%:%217=90%:%
%:%218=90%:%
%:%219=91%:%
%:%220=91%:%
%:%221=92%:%
%:%222=92%:%
%:%223=92%:%
%:%224=92%:%
%:%225=93%:%
%:%226=93%:%
%:%227=93%:%
%:%228=93%:%
%:%229=94%:%
%:%232=96%:%
%:%234=97%:%
%:%235=97%:%
%:%236=98%:%
%:%237=98%:%
%:%238=99%:%
%:%239=99%:%
%:%240=100%:%
%:%241=100%:%
%:%242=101%:%
%:%243=101%:%
%:%244=101%:%
%:%245=102%:%
%:%246=103%:%
%:%247=103%:%
%:%248=104%:%
%:%249=104%:%
%:%250=104%:%
%:%251=104%:%
%:%252=105%:%
%:%253=106%:%
%:%254=106%:%
%:%255=106%:%
%:%256=106%:%
%:%257=107%:%
%:%258=107%:%
%:%259=107%:%
%:%260=107%:%
%:%261=107%:%
%:%262=107%:%
%:%263=108%:%
%:%264=108%:%
%:%265=108%:%
%:%266=108%:%
%:%267=108%:%
%:%268=108%:%
%:%269=109%:%
%:%270=109%:%
%:%271=109%:%
%:%272=109%:%
%:%273=109%:%
%:%274=110%:%
%:%275=110%:%
%:%276=110%:%
%:%277=110%:%
%:%278=110%:%
%:%279=111%:%
%:%280=111%:%
%:%281=111%:%
%:%282=111%:%
%:%283=111%:%
%:%284=112%:%
%:%285=112%:%
%:%286=112%:%
%:%287=112%:%
%:%288=112%:%
%:%289=113%:%
%:%290=113%:%
%:%291=113%:%
%:%292=113%:%
%:%293=114%:%
%:%294=114%:%
%:%295=114%:%
%:%296=114%:%
%:%297=115%:%
%:%298=115%:%
%:%299=116%:%
%:%300=117%:%
%:%301=117%:%
%:%302=117%:%
%:%304=119%:%
%:%305=119%:%
%:%306=120%:%
%:%307=120%:%
%:%308=121%:%
%:%309=122%:%
%:%310=122%:%
%:%311=122%:%
%:%312=123%:%
%:%313=123%:%
%:%314=123%:%
%:%315=123%:%
%:%316=124%:%
%:%317=124%:%
%:%318=124%:%
%:%320=126%:%
%:%322=127%:%
%:%323=127%:%
%:%324=128%:%
%:%325=128%:%
%:%326=128%:%
%:%327=129%:%
%:%328=129%:%
%:%329=129%:%
%:%330=130%:%
%:%331=130%:%
%:%332=130%:%
%:%333=131%:%
%:%334=131%:%
%:%335=131%:%
%:%336=132%:%
%:%337=132%:%
%:%338=132%:%
%:%339=133%:%
%:%340=133%:%
%:%341=133%:%
%:%342=134%:%
%:%343=134%:%
%:%344=134%:%
%:%345=135%:%
%:%346=135%:%
%:%347=135%:%
%:%348=135%:%
%:%349=136%:%
%:%352=137%:%
%:%354=138%:%
%:%355=138%:%
%:%356=139%:%
%:%357=139%:%
%:%358=139%:%
%:%359=140%:%
%:%360=140%:%
%:%361=141%:%
%:%362=141%:%
%:%363=142%:%
%:%364=142%:%
%:%365=143%:%
%:%366=143%:%
%:%367=143%:%
%:%368=144%:%
%:%369=144%:%
%:%370=145%:%
%:%371=145%:%
%:%372=146%:%
%:%373=146%:%
%:%374=146%:%
%:%375=147%:%
%:%376=147%:%
%:%377=147%:%
%:%378=147%:%
%:%379=147%:%
%:%380=148%:%
%:%381=148%:%
%:%382=148%:%
%:%383=148%:%
%:%384=149%:%
%:%385=149%:%
%:%386=150%:%
%:%387=150%:%
%:%388=151%:%
%:%403=153%:%
%:%415=155%:%
%:%416=156%:%
%:%417=157%:%
%:%418=158%:%
%:%420=160%:%
%:%421=160%:%
%:%422=161%:%
%:%423=162%:%
%:%424=163%:%
%:%431=164%:%
%:%432=164%:%
%:%433=165%:%
%:%434=165%:%
%:%435=166%:%
%:%436=166%:%
%:%437=166%:%
%:%438=166%:%
%:%439=167%:%
%:%440=168%:%
%:%441=168%:%
%:%442=168%:%
%:%443=168%:%
%:%444=169%:%
%:%445=169%:%
%:%446=169%:%
%:%447=169%:%
%:%448=170%:%
%:%449=171%:%
%:%450=171%:%
%:%451=172%:%
%:%452=172%:%
%:%453=172%:%
%:%454=173%:%
%:%455=173%:%
%:%456=173%:%
%:%457=173%:%
%:%458=173%:%
%:%459=174%:%
%:%460=174%:%
%:%461=174%:%
%:%462=174%:%
%:%463=175%:%
%:%464=175%:%
%:%465=175%:%
%:%466=175%:%
%:%467=176%:%
%:%468=176%:%
%:%469=176%:%
%:%470=176%:%
%:%471=177%:%
%:%486=179%:%
%:%496=180%:%
%:%497=180%:%
%:%498=181%:%
%:%499=182%:%
%:%500=183%:%
%:%507=184%:%
%:%508=184%:%
%:%509=185%:%
%:%510=185%:%
%:%511=185%:%
%:%512=185%:%
%:%513=186%:%
%:%514=186%:%
%:%515=186%:%
%:%516=186%:%
%:%517=187%:%
%:%518=187%:%
%:%519=187%:%
%:%520=187%:%
%:%521=188%:%
%:%536=190%:%
%:%548=192%:%
%:%549=193%:%
%:%551=195%:%
%:%552=195%:%
%:%553=196%:%
%:%554=197%:%
%:%561=198%:%
%:%562=198%:%
%:%563=199%:%
%:%564=199%:%
%:%565=199%:%
%:%566=199%:%
%:%567=200%:%
%:%568=200%:%
%:%569=200%:%
%:%570=200%:%
%:%571=201%:%
%:%572=201%:%
%:%573=201%:%
%:%574=201%:%
%:%575=202%:%
%:%576=202%:%
%:%577=203%:%
%:%578=203%:%
%:%579=203%:%
%:%580=204%:%
%:%586=204%:%
%:%591=205%:%
%:%596=206%:%



\bibliographystyle{abbrv}
\bibliography{root}

\end{document}
